\graphicspath{ {images/} }

\section{OpenCAPI Command and Configure Sequencer (C3S)} \label{section_c3s}

\subsection{Overview}
The OpenCAPI Command and Configure Sequencer (C3S) is a set of
controls capable of stepping and looping through a configurable array
of commands. This provides a simple and flexible method of driving any
interface, which can be useful for debugging and testing as well as
initialization. Unlike a functional exerciser, C3S is exclusively a
transmitter, and can't react to any responses, such as
handshakes. While these responses can have no functional purpose, they
are captured for later manual analysis.

C3S is designed to provide a core architecture with a command array of
configurable width and depth in order to allow reuse in multiple
applications. The logic block produces a command vector and a select
signal as outputs. Each application will use this command vector as
desired, and the select signal can be used to select C3S traffic
instead of normal functional traffic.

The core C3S macro consists of five parts: a data array to store the
C3S instructions sent, an instruction flow array that holds the
control instructions such as loop count and next instruction, a
response array to capture any responses for future analysis,
configuration registers to hold control information and AXI-Lite
slave, and the control block with the central state machine. This
chapter details these five generic blocks, which are shown in
Figure~\ref{fig:c3s_overview}. Additionally, later sections explain
specific applications of C3S in the Apollo FPGA and how the core
architecture is implemented in each case.

\begin{figure}[h]
  \begin{center}
    % Graphic for TeX using PGF
% Title: /home/rpking42/proj/chips/astra/doc/images/c3s_overview.dia
% Creator: Dia v0.97.3
% CreationDate: Mon Mar 12 09:57:37 2018
% For: rpking42
% \usepackage{tikz}
% The following commands are not supported in PSTricks at present
% We define them conditionally, so when they are implemented,
% this pgf file will use them.
\ifx\du\undefined
  \newlength{\du}
\fi
\setlength{\du}{15\unitlength}
\begin{tikzpicture}
\pgftransformxscale{1.000000}
\pgftransformyscale{-1.000000}
\definecolor{dialinecolor}{rgb}{0.000000, 0.000000, 0.000000}
\pgfsetstrokecolor{dialinecolor}
\definecolor{dialinecolor}{rgb}{1.000000, 1.000000, 1.000000}
\pgfsetfillcolor{dialinecolor}
\pgfsetlinewidth{0.000000\du}
\pgfsetdash{}{0pt}
\pgfsetdash{}{0pt}
\pgfsetmiterjoin
\definecolor{dialinecolor}{rgb}{0.000000, 0.000000, 0.000000}
\pgfsetstrokecolor{dialinecolor}
\draw (21.000000\du,17.000000\du)--(21.000000\du,26.000000\du)--(33.000000\du,26.000000\du)--(33.000000\du,17.000000\du)--cycle;
% setfont left to latex
\definecolor{dialinecolor}{rgb}{0.000000, 0.000000, 0.000000}
\pgfsetstrokecolor{dialinecolor}
\node[anchor=west] at (21.500000\du,18.095000\du){c3s\_data};
\pgfsetlinewidth{0.000000\du}
\pgfsetdash{}{0pt}
\pgfsetdash{}{0pt}
\pgfsetmiterjoin
\definecolor{dialinecolor}{rgb}{0.000000, 0.000000, 0.000000}
\pgfsetstrokecolor{dialinecolor}
\draw (43.500000\du,28.000000\du)--(43.500000\du,32.000000\du)--(48.000000\du,32.000000\du)--(48.000000\du,28.000000\du)--cycle;
% setfont left to latex
\definecolor{dialinecolor}{rgb}{0.000000, 0.000000, 0.000000}
\pgfsetstrokecolor{dialinecolor}
\node at (45.750000\du,30.221250\du){c3s\_cfg};
\pgfsetlinewidth{0.000000\du}
\pgfsetdash{}{0pt}
\pgfsetdash{}{0pt}
\pgfsetmiterjoin
\definecolor{dialinecolor}{rgb}{0.000000, 0.000000, 0.000000}
\pgfsetstrokecolor{dialinecolor}
\draw (34.000000\du,28.000000\du)--(34.000000\du,32.000000\du)--(38.500000\du,32.000000\du)--(38.500000\du,28.000000\du)--cycle;
% setfont left to latex
\definecolor{dialinecolor}{rgb}{0.000000, 0.000000, 0.000000}
\pgfsetstrokecolor{dialinecolor}
\node at (36.250000\du,30.221250\du){c3s\_cntl};
\pgfsetlinewidth{0.000000\du}
\pgfsetdash{}{0pt}
\pgfsetdash{}{0pt}
\pgfsetmiterjoin
\definecolor{dialinecolor}{rgb}{0.000000, 0.000000, 0.000000}
\pgfsetstrokecolor{dialinecolor}
\draw (23.500000\du,19.000000\du)--(23.500000\du,25.000000\du)--(32.000000\du,25.000000\du)--(32.000000\du,19.000000\du)--cycle;
% setfont left to latex
\definecolor{dialinecolor}{rgb}{0.000000, 0.000000, 0.000000}
\pgfsetstrokecolor{dialinecolor}
\node at (27.750000\du,20.621050\du){BRAM};
% setfont left to latex
\definecolor{dialinecolor}{rgb}{0.000000, 0.000000, 0.000000}
\pgfsetstrokecolor{dialinecolor}
\node at (27.750000\du,21.421150\du){};
% setfont left to latex
\definecolor{dialinecolor}{rgb}{0.000000, 0.000000, 0.000000}
\pgfsetstrokecolor{dialinecolor}
\node at (27.750000\du,22.221250\du){C\_COMMAND\_WIDTH};
% setfont left to latex
\definecolor{dialinecolor}{rgb}{0.000000, 0.000000, 0.000000}
\pgfsetstrokecolor{dialinecolor}
\node at (27.750000\du,23.021350\du){x};
% setfont left to latex
\definecolor{dialinecolor}{rgb}{0.000000, 0.000000, 0.000000}
\pgfsetstrokecolor{dialinecolor}
\node at (27.750000\du,23.821450\du){2\^{}C\_ADDRESS\_WIDTH};
\pgfsetlinewidth{0.000000\du}
\pgfsetdash{}{0pt}
\pgfsetdash{}{0pt}
\pgfsetmiterjoin
\definecolor{dialinecolor}{rgb}{0.000000, 0.000000, 0.000000}
\pgfsetstrokecolor{dialinecolor}
\draw (21.000000\du,34.000000\du)--(21.000000\du,43.000000\du)--(33.000000\du,43.000000\du)--(33.000000\du,34.000000\du)--cycle;
% setfont left to latex
\definecolor{dialinecolor}{rgb}{0.000000, 0.000000, 0.000000}
\pgfsetstrokecolor{dialinecolor}
\node[anchor=west] at (21.500000\du,35.095000\du){c3s\_flow};
\pgfsetlinewidth{0.000000\du}
\pgfsetdash{}{0pt}
\pgfsetdash{}{0pt}
\pgfsetmiterjoin
\definecolor{dialinecolor}{rgb}{0.000000, 0.000000, 0.000000}
\pgfsetstrokecolor{dialinecolor}
\draw (23.500000\du,36.000000\du)--(23.500000\du,42.000000\du)--(32.000000\du,42.000000\du)--(32.000000\du,36.000000\du)--cycle;
% setfont left to latex
\definecolor{dialinecolor}{rgb}{0.000000, 0.000000, 0.000000}
\pgfsetstrokecolor{dialinecolor}
\node at (27.750000\du,37.621050\du){BRAM};
% setfont left to latex
\definecolor{dialinecolor}{rgb}{0.000000, 0.000000, 0.000000}
\pgfsetstrokecolor{dialinecolor}
\node at (27.750000\du,38.421150\du){};
% setfont left to latex
\definecolor{dialinecolor}{rgb}{0.000000, 0.000000, 0.000000}
\pgfsetstrokecolor{dialinecolor}
\node at (27.750000\du,39.221250\du){32};
% setfont left to latex
\definecolor{dialinecolor}{rgb}{0.000000, 0.000000, 0.000000}
\pgfsetstrokecolor{dialinecolor}
\node at (27.750000\du,40.021350\du){x};
% setfont left to latex
\definecolor{dialinecolor}{rgb}{0.000000, 0.000000, 0.000000}
\pgfsetstrokecolor{dialinecolor}
\node at (27.750000\du,40.821450\du){2\^{}C\_ADDRESS\_WIDTH};
\pgfsetlinewidth{0.100000\du}
\pgfsetdash{}{0pt}
\pgfsetdash{}{0pt}
\pgfsetmiterjoin
\pgfsetbuttcap
{
\definecolor{dialinecolor}{rgb}{0.000000, 0.000000, 0.000000}
\pgfsetfillcolor{dialinecolor}
% was here!!!
\pgfsetarrowsend{latex}
{\pgfsetcornersarced{\pgfpoint{0.000000\du}{0.000000\du}}\definecolor{dialinecolor}{rgb}{0.000000, 0.000000, 0.000000}
\pgfsetstrokecolor{dialinecolor}
\draw (32.000000\du,38.000000\du)--(35.500000\du,38.000000\du)--(35.500000\du,32.000000\du);
}}
\pgfsetlinewidth{0.100000\du}
\pgfsetdash{}{0pt}
\pgfsetdash{}{0pt}
\pgfsetmiterjoin
\pgfsetbuttcap
{
\definecolor{dialinecolor}{rgb}{0.000000, 0.000000, 0.000000}
\pgfsetfillcolor{dialinecolor}
% was here!!!
\pgfsetarrowsend{latex}
{\pgfsetcornersarced{\pgfpoint{0.000000\du}{0.000000\du}}\definecolor{dialinecolor}{rgb}{0.000000, 0.000000, 0.000000}
\pgfsetstrokecolor{dialinecolor}
\draw (32.000000\du,23.000000\du)--(36.250000\du,23.000000\du)--(36.250000\du,28.000000\du);
}}
\pgfsetlinewidth{0.000000\du}
\pgfsetdash{}{0pt}
\pgfsetdash{}{0pt}
\pgfsetbuttcap
{
\definecolor{dialinecolor}{rgb}{0.000000, 0.000000, 0.000000}
\pgfsetfillcolor{dialinecolor}
% was here!!!
\definecolor{dialinecolor}{rgb}{0.000000, 0.000000, 0.000000}
\pgfsetstrokecolor{dialinecolor}
\draw (32.000000\du,19.000000\du)--(32.000000\du,25.000000\du);
}
\pgfsetlinewidth{0.000000\du}
\pgfsetdash{}{0pt}
\pgfsetdash{}{0pt}
\pgfsetbuttcap
{
\definecolor{dialinecolor}{rgb}{0.000000, 0.000000, 0.000000}
\pgfsetfillcolor{dialinecolor}
% was here!!!
\definecolor{dialinecolor}{rgb}{0.000000, 0.000000, 0.000000}
\pgfsetstrokecolor{dialinecolor}
\draw (32.000000\du,36.000000\du)--(32.000000\du,42.000000\du);
}
\pgfsetlinewidth{0.100000\du}
\pgfsetdash{}{0pt}
\pgfsetdash{}{0pt}
\pgfsetmiterjoin
\pgfsetbuttcap
{
\definecolor{dialinecolor}{rgb}{0.000000, 0.000000, 0.000000}
\pgfsetfillcolor{dialinecolor}
% was here!!!
\pgfsetarrowsstart{latex}
\pgfsetarrowsend{latex}
{\pgfsetcornersarced{\pgfpoint{0.000000\du}{0.000000\du}}\definecolor{dialinecolor}{rgb}{0.000000, 0.000000, 0.000000}
\pgfsetstrokecolor{dialinecolor}
\draw (32.000000\du,21.000000\du)--(45.750000\du,21.000000\du)--(45.750000\du,28.000000\du);
}}
% setfont left to latex
\definecolor{dialinecolor}{rgb}{0.000000, 0.000000, 0.000000}
\pgfsetstrokecolor{dialinecolor}
\node[anchor=west] at (33.400000\du,20.600000\du){R/W Port B};
% setfont left to latex
\definecolor{dialinecolor}{rgb}{0.000000, 0.000000, 0.000000}
\pgfsetstrokecolor{dialinecolor}
\node[anchor=west] at (33.400000\du,22.600000\du){R Port A};
% setfont left to latex
\definecolor{dialinecolor}{rgb}{0.000000, 0.000000, 0.000000}
\pgfsetstrokecolor{dialinecolor}
\node[anchor=west] at (33.400000\du,40.595000\du){R/W Port B};
% setfont left to latex
\definecolor{dialinecolor}{rgb}{0.000000, 0.000000, 0.000000}
\pgfsetstrokecolor{dialinecolor}
\node[anchor=west] at (33.400000\du,38.595000\du){R Port A};
\pgfsetlinewidth{0.100000\du}
\pgfsetdash{}{0pt}
\pgfsetdash{}{0pt}
\pgfsetmiterjoin
\pgfsetbuttcap
{
\definecolor{dialinecolor}{rgb}{0.000000, 0.000000, 0.000000}
\pgfsetfillcolor{dialinecolor}
% was here!!!
\pgfsetarrowsstart{latex}
\pgfsetarrowsend{latex}
{\pgfsetcornersarced{\pgfpoint{0.000000\du}{0.000000\du}}\definecolor{dialinecolor}{rgb}{0.000000, 0.000000, 0.000000}
\pgfsetstrokecolor{dialinecolor}
\draw (32.000000\du,40.000000\du)--(45.000000\du,40.000000\du)--(45.000000\du,32.000000\du);
}}
% setfont left to latex
\definecolor{dialinecolor}{rgb}{0.000000, 0.000000, 0.000000}
\pgfsetstrokecolor{dialinecolor}
\node at (41.000000\du,31.395000\du){Start/Stop};
\pgfsetlinewidth{0.000000\du}
\pgfsetdash{}{0pt}
\pgfsetdash{}{0pt}
\pgfsetbuttcap
{
\definecolor{dialinecolor}{rgb}{0.000000, 0.000000, 0.000000}
\pgfsetfillcolor{dialinecolor}
% was here!!!
\definecolor{dialinecolor}{rgb}{0.000000, 0.000000, 0.000000}
\pgfsetstrokecolor{dialinecolor}
\draw (38.500000\du,28.000000\du)--(38.500000\du,32.000000\du);
}
\pgfsetlinewidth{0.000000\du}
\pgfsetdash{}{0pt}
\pgfsetdash{}{0pt}
\pgfsetbuttcap
{
\definecolor{dialinecolor}{rgb}{0.000000, 0.000000, 0.000000}
\pgfsetfillcolor{dialinecolor}
% was here!!!
\definecolor{dialinecolor}{rgb}{0.000000, 0.000000, 0.000000}
\pgfsetstrokecolor{dialinecolor}
\draw (43.500000\du,28.000000\du)--(43.500000\du,32.000000\du);
}
% setfont left to latex
\definecolor{dialinecolor}{rgb}{0.000000, 0.000000, 0.000000}
\pgfsetstrokecolor{dialinecolor}
\node at (41.000000\du,29.047500\du){In Progress};
% setfont left to latex
\definecolor{dialinecolor}{rgb}{0.000000, 0.000000, 0.000000}
\pgfsetstrokecolor{dialinecolor}
\node[anchor=west] at (22.000000\du,31.095000\du){c3s\_command};
% setfont left to latex
\definecolor{dialinecolor}{rgb}{0.000000, 0.000000, 0.000000}
\pgfsetstrokecolor{dialinecolor}
\node[anchor=west] at (22.000000\du,31.895000\du){      (C\_COMMAND\_WIDTH - 1 : 0)};
% setfont left to latex
\definecolor{dialinecolor}{rgb}{0.000000, 0.000000, 0.000000}
\pgfsetstrokecolor{dialinecolor}
\node[anchor=west] at (48.500000\du,30.595000\du){AXI4-Lite};
\pgfsetlinewidth{0.000000\du}
\pgfsetdash{}{0pt}
\pgfsetdash{}{0pt}
\pgfsetbuttcap
{
\definecolor{dialinecolor}{rgb}{0.000000, 0.000000, 0.000000}
\pgfsetfillcolor{dialinecolor}
% was here!!!
\pgfsetarrowsend{latex}
\definecolor{dialinecolor}{rgb}{0.000000, 0.000000, 0.000000}
\pgfsetstrokecolor{dialinecolor}
\draw (38.500000\du,29.333300\du)--(43.500000\du,29.333300\du);
}
\pgfsetlinewidth{0.000000\du}
\pgfsetdash{}{0pt}
\pgfsetdash{}{0pt}
\pgfsetbuttcap
{
\definecolor{dialinecolor}{rgb}{0.000000, 0.000000, 0.000000}
\pgfsetfillcolor{dialinecolor}
% was here!!!
\pgfsetarrowsend{latex}
\definecolor{dialinecolor}{rgb}{0.000000, 0.000000, 0.000000}
\pgfsetstrokecolor{dialinecolor}
\draw (43.500000\du,30.666700\du)--(38.500000\du,30.666700\du);
}
\pgfsetlinewidth{0.200000\du}
\pgfsetdash{}{0pt}
\pgfsetdash{}{0pt}
\pgfsetbuttcap
{
\definecolor{dialinecolor}{rgb}{0.000000, 0.000000, 0.000000}
\pgfsetfillcolor{dialinecolor}
% was here!!!
\pgfsetarrowsend{latex}
\definecolor{dialinecolor}{rgb}{0.000000, 0.000000, 0.000000}
\pgfsetstrokecolor{dialinecolor}
\draw (52.000000\du,30.000000\du)--(48.000000\du,30.000000\du);
}
\pgfsetlinewidth{0.200000\du}
\pgfsetdash{}{0pt}
\pgfsetdash{}{0pt}
\pgfsetbuttcap
{
\definecolor{dialinecolor}{rgb}{0.000000, 0.000000, 0.000000}
\pgfsetfillcolor{dialinecolor}
% was here!!!
\pgfsetarrowsend{latex}
\definecolor{dialinecolor}{rgb}{0.000000, 0.000000, 0.000000}
\pgfsetstrokecolor{dialinecolor}
\draw (34.000000\du,30.666667\du)--(21.000000\du,30.666667\du);
}
\pgfsetlinewidth{0.000000\du}
\pgfsetdash{}{0pt}
\pgfsetdash{}{0pt}
\pgfsetbuttcap
{
\definecolor{dialinecolor}{rgb}{0.000000, 0.000000, 0.000000}
\pgfsetfillcolor{dialinecolor}
% was here!!!
\definecolor{dialinecolor}{rgb}{0.000000, 0.000000, 0.000000}
\pgfsetstrokecolor{dialinecolor}
\draw (34.000000\du,32.000000\du)--(38.500000\du,32.000000\du);
}
\pgfsetlinewidth{0.000000\du}
\pgfsetdash{}{0pt}
\pgfsetdash{}{0pt}
\pgfsetbuttcap
{
\definecolor{dialinecolor}{rgb}{0.000000, 0.000000, 0.000000}
\pgfsetfillcolor{dialinecolor}
% was here!!!
\definecolor{dialinecolor}{rgb}{0.000000, 0.000000, 0.000000}
\pgfsetstrokecolor{dialinecolor}
\draw (34.000000\du,28.000000\du)--(34.000000\du,32.000000\du);
}
\pgfsetlinewidth{0.200000\du}
\pgfsetdash{}{0pt}
\pgfsetdash{}{0pt}
\pgfsetbuttcap
{
\definecolor{dialinecolor}{rgb}{0.000000, 0.000000, 0.000000}
\pgfsetfillcolor{dialinecolor}
% was here!!!
\pgfsetarrowsend{latex}
\definecolor{dialinecolor}{rgb}{0.000000, 0.000000, 0.000000}
\pgfsetstrokecolor{dialinecolor}
\draw (21.000000\du,29.333333\du)--(34.000000\du,29.333333\du);
}
% setfont left to latex
\definecolor{dialinecolor}{rgb}{0.000000, 0.000000, 0.000000}
\pgfsetstrokecolor{dialinecolor}
\node[anchor=west] at (22.000000\du,28.010125\du){c3s\_response};
% setfont left to latex
\definecolor{dialinecolor}{rgb}{0.000000, 0.000000, 0.000000}
\pgfsetstrokecolor{dialinecolor}
\node[anchor=west] at (22.000000\du,28.810125\du){      (C\_COMMAND\_WIDTH - 1 : 0)};
\pgfsetlinewidth{0.000000\du}
\pgfsetdash{}{0pt}
\pgfsetdash{}{0pt}
\pgfsetmiterjoin
\definecolor{dialinecolor}{rgb}{0.000000, 0.000000, 0.000000}
\pgfsetstrokecolor{dialinecolor}
\draw (21.000000\du,45.000000\du)--(21.000000\du,54.000000\du)--(33.000000\du,54.000000\du)--(33.000000\du,45.000000\du)--cycle;
% setfont left to latex
\definecolor{dialinecolor}{rgb}{0.000000, 0.000000, 0.000000}
\pgfsetstrokecolor{dialinecolor}
\node[anchor=west] at (21.500000\du,46.113687\du){c3s\_resp};
\pgfsetlinewidth{0.000000\du}
\pgfsetdash{}{0pt}
\pgfsetdash{}{0pt}
\pgfsetmiterjoin
\definecolor{dialinecolor}{rgb}{0.000000, 0.000000, 0.000000}
\pgfsetstrokecolor{dialinecolor}
\draw (23.500000\du,47.000000\du)--(23.500000\du,53.000000\du)--(32.000000\du,53.000000\du)--(32.000000\du,47.000000\du)--cycle;
% setfont left to latex
\definecolor{dialinecolor}{rgb}{0.000000, 0.000000, 0.000000}
\pgfsetstrokecolor{dialinecolor}
\node at (27.750000\du,48.621050\du){BRAM};
% setfont left to latex
\definecolor{dialinecolor}{rgb}{0.000000, 0.000000, 0.000000}
\pgfsetstrokecolor{dialinecolor}
\node at (27.750000\du,49.421150\du){};
% setfont left to latex
\definecolor{dialinecolor}{rgb}{0.000000, 0.000000, 0.000000}
\pgfsetstrokecolor{dialinecolor}
\node at (27.750000\du,50.221250\du){32};
% setfont left to latex
\definecolor{dialinecolor}{rgb}{0.000000, 0.000000, 0.000000}
\pgfsetstrokecolor{dialinecolor}
\node at (27.750000\du,51.021350\du){x};
% setfont left to latex
\definecolor{dialinecolor}{rgb}{0.000000, 0.000000, 0.000000}
\pgfsetstrokecolor{dialinecolor}
\node at (27.750000\du,51.821450\du){2\^{}C\_ADDRESS\_WIDTH};
\pgfsetlinewidth{0.100000\du}
\pgfsetdash{}{0pt}
\pgfsetdash{}{0pt}
\pgfsetmiterjoin
\pgfsetbuttcap
{
\definecolor{dialinecolor}{rgb}{0.000000, 0.000000, 0.000000}
\pgfsetfillcolor{dialinecolor}
% was here!!!
\pgfsetarrowsend{latex}
{\pgfsetcornersarced{\pgfpoint{0.000000\du}{0.000000\du}}\definecolor{dialinecolor}{rgb}{0.000000, 0.000000, 0.000000}
\pgfsetstrokecolor{dialinecolor}
\draw (37.000000\du,32.000000\du)--(37.000000\du,35.000000\du)--(39.500000\du,35.000000\du)--(39.500000\du,49.000000\du)--(32.000000\du,49.000000\du);
}}
\pgfsetlinewidth{0.000000\du}
\pgfsetdash{}{0pt}
\pgfsetdash{}{0pt}
\pgfsetbuttcap
{
\definecolor{dialinecolor}{rgb}{0.000000, 0.000000, 0.000000}
\pgfsetfillcolor{dialinecolor}
% was here!!!
\definecolor{dialinecolor}{rgb}{0.000000, 0.000000, 0.000000}
\pgfsetstrokecolor{dialinecolor}
\draw (32.000000\du,47.000000\du)--(32.000000\du,53.000000\du);
}
% setfont left to latex
\definecolor{dialinecolor}{rgb}{0.000000, 0.000000, 0.000000}
\pgfsetstrokecolor{dialinecolor}
\node[anchor=west] at (33.400000\du,51.595000\du){R/W Port B};
% setfont left to latex
\definecolor{dialinecolor}{rgb}{0.000000, 0.000000, 0.000000}
\pgfsetstrokecolor{dialinecolor}
\node[anchor=west] at (33.400000\du,49.613687\du){W Port A};
\pgfsetlinewidth{0.100000\du}
\pgfsetdash{}{0pt}
\pgfsetdash{}{0pt}
\pgfsetmiterjoin
\pgfsetbuttcap
{
\definecolor{dialinecolor}{rgb}{0.000000, 0.000000, 0.000000}
\pgfsetfillcolor{dialinecolor}
% was here!!!
\pgfsetarrowsstart{latex}
\pgfsetarrowsend{latex}
{\pgfsetcornersarced{\pgfpoint{0.000000\du}{0.000000\du}}\definecolor{dialinecolor}{rgb}{0.000000, 0.000000, 0.000000}
\pgfsetstrokecolor{dialinecolor}
\draw (32.000000\du,51.000000\du)--(46.500000\du,51.000000\du)--(46.500000\du,32.000000\du);
}}
\pgfsetlinewidth{0.000000\du}
\pgfsetdash{}{0pt}
\pgfsetdash{}{0pt}
\pgfsetbuttcap
{
\definecolor{dialinecolor}{rgb}{0.000000, 0.000000, 0.000000}
\pgfsetfillcolor{dialinecolor}
% was here!!!
\definecolor{dialinecolor}{rgb}{0.000000, 0.000000, 0.000000}
\pgfsetstrokecolor{dialinecolor}
\draw (43.500000\du,32.000000\du)--(48.000000\du,32.000000\du);
}
\end{tikzpicture}

  \end{center}
  \caption[Overview of c3s\_rlm.vhdl]{\label{fig:c3s_overview}Overview of c3s\_rlm.vhdl
  }
\end{figure}

\subsection{Interfaces}
VHDL Name: c3s\_rlm (c3s\_rlm.vhdl)

\subsubsection{Generics}
\begin{center}
  \begin{tabular}{ l | c | c | c | c | p{.3\linewidth} }
    Generic Name      & Type     & Min & Max  & Default & Description \\
    \hline
    C\_COMMAND\_WIDTH & Positive & 1   & 8192 & 512     & Number of bits in command driven by C3S. A multiple of 32 is required for efficient implementation. \\
    \hline
    C\_ADDRESS\_WIDTH & Positive & 1   & 8    & 5       & Command array will contain 2\textsuperscript{C\_ADDRESS\_WIDTH} commands. \\
  \end{tabular}
\end{center}

\subsubsection{Ports}
\begin{center}
  \begin{tabular}{ l | c | c | p{.3\linewidth} }
    Signal Name          & I/O & Size              & Description \\
    \hline
    \multicolumn{4}{c}{Clocking} \\
    \hline
    clock                & I   & 1                 & Positive clock \\
    \hline
    reset                & I   & 1                 & Active-high reset \\
    \hline
    \multicolumn{4}{c}{Command} \\
    \hline
    c3s\_command         & O   & C\_COMMAND\_WIDTH & C3S command vector \\
    \hline
    c3s\_command\_valid  & O   & 1                 & C3S command is valid, optional and custom use \\
    \hline
    c3s\_command\_sel    & O   & 1                 & Asserted high when C3S is running and commands are valid \\
    \hline
    \multicolumn{4}{c}{Response} \\
    \hline
    c3s\_response        & I   & C\_COMMAND\_WIDTH & C3S response vector \\
    \hline
    c3s\_response\_valid & I   & 1                 & Response is valid and should be stored in response array \\
  \end{tabular}
\end{center}

\textbf{Note}: Signals are little-endian ($N$ \textbf{downto} 0)

\subsection{Data Array}
VHDL Name: c3s\_data(bram\_wrap.vhdl)

The data array is simply a C\_COMMAND\_WIDTH wide by
2\textsuperscript{C\_ADDRESS\_WIDTH} deep array that contains the
instructions. Each bit of a row corresponds one-to-one with each bit
of c3s\_command, and each row is a discrete command. The Instruction
Flow Array controls exactly how these commands are sent.

The array is implemented as a BRAM (Block RAM) primitive with 2 R/W
ports. Port A is connected to the control block which does reads
during C3S execution. The control block does not do writes. Port B is
connected to the configuration block for read and write configurations
over AXI4-Lite, where each row of the data array is one or more unique
register(s).

\begin{emulation}
In an emulation environment, BRAM primitives are not available in the
hardware, as they are specific for Xilinx FPGAs like that used on
Apollo. The BRAM wrapper, bram\_wrap.vhdl, is designed to infer
latches if it's in an environment without BRAM devices.
\end{emulation}

\subsection{Response Capture Array}
VHDL Name: c3s\_resp(bram\_wrap.vhdl)

The response capture array provides a method to see the responses
coming across an interface. No checking is done; rather the responses
are inserted into the array for manual analysis. The array is written
to by an incrementing write pointer. When the maximum value of the
write pointer is reached, it resets to 0 and starts overwriting
previous entries. This ensures that the most recent responses received
are found in the array. When a response is stored in the array, there
is also a timestamp stored. This measures the time from the start of
C3S execution to when the response was received by the Response
Capture Array.

The C3S\_RESP\_CNTL provides a method to see where in the array we are
writing, if we have wrapper around. There is also a way to reset the
write address to start counting from 0 in order to provide a higher
chance of a repeatable test.

This array is the same size as the data array: C\_COMMAND\_WIDTH wide
by 2\textsuperscript{C\_ADDRESS\_WIDTH} deep.

There is a wrap path inside the control block that allows c3s\_command
to be written into the response array rather than c3s\_response. This
allows debug of a C3S sequence to verify the commands being sent
without relying on the interface. Additionally, it is possible to
disable c3s\_command\_sel in order to capture the command without
actually sending anything across the interface, in order to not
interfere with real traffic. Both of these settings are configurable
in C3S\_CNTL.

\subsection{Instruction Flow Array}
VHDL Name: c3s\_flow(bram\_wrap.vhdl)

The instruction flow array is the meta information associated with
each data entry. The array is a constant 32 bits wide by
2\textsuperscript{C\_ADDRESS\_WIDTH} deep. This flow entry contains a
repeat flag, a next instruction pointer, and a last entry flag. The
repeat flag indicates that a command repeats a number of times. Each
command is run one more than than the number gives, e.g. 0 means the
command will run only once, 1 means the command is repeated twice. The
next instruction pointer points to the command that will run next. If
set to the next address, execution will serially go through the
array. This also allows infinite loops. Finally, if execution sees a
command with the last entry flag, then C3S execution will
stop. Currently, the command containing the last entry flag is not
executed, so the final command executed is the command before the one
with the last entry flag set.

\subsection{Control Block}
VHDL Name: c3s\_cntl(c3s\_cntl.vhdl)

The control block is the core of C3S that performs a few functions. It
contains the central state machine, the read controls for port A of
the flow and data, and write controls for port A of the response. The
state machine, outlined in Figure~\ref{fig:c3s_fsm}, steps through the
C3S array as programmed by the Instruction Flow Array and forwards the
command in the Data Array to the command bus. States are encode in a
one-hot signal, with c3s\_fsm\_q(0) being set for state 0, and so
on. Additionally, the control block contains the write logic for the
response array, writing to the response array only when the response
is valid.

\begin{figure}[h]
  \begin{center}
    % Graphic for TeX using PGF
% Title: /afs/apd/func/vlsi/eclipz/c14/usr/rpking42/cbdd1/astra-sim/doc/images/c3s_fsm.dia
% Creator: Dia v0.97.3
% CreationDate: Tue Jun 12 17:56:29 2018
% For: rpking42
% \usepackage{tikz}
% The following commands are not supported in PSTricks at present
% We define them conditionally, so when they are implemented,
% this pgf file will use them.
\ifx\du\undefined
  \newlength{\du}
\fi
\setlength{\du}{15\unitlength}
\begin{tikzpicture}
\pgftransformxscale{1.000000}
\pgftransformyscale{-1.000000}
\definecolor{dialinecolor}{rgb}{0.000000, 0.000000, 0.000000}
\pgfsetstrokecolor{dialinecolor}
\definecolor{dialinecolor}{rgb}{1.000000, 1.000000, 1.000000}
\pgfsetfillcolor{dialinecolor}
\definecolor{dialinecolor}{rgb}{1.000000, 1.000000, 1.000000}
\pgfsetfillcolor{dialinecolor}
\pgfpathellipse{\pgfpoint{14.500000\du}{11.500000\du}}{\pgfpoint{2.500000\du}{0\du}}{\pgfpoint{0\du}{2.500000\du}}
\pgfusepath{fill}
\pgfsetlinewidth{0.000000\du}
\pgfsetdash{}{0pt}
\pgfsetdash{}{0pt}
\definecolor{dialinecolor}{rgb}{0.000000, 0.000000, 0.000000}
\pgfsetstrokecolor{dialinecolor}
\pgfpathellipse{\pgfpoint{14.500000\du}{11.500000\du}}{\pgfpoint{2.500000\du}{0\du}}{\pgfpoint{0\du}{2.500000\du}}
\pgfusepath{stroke}
\definecolor{dialinecolor}{rgb}{1.000000, 1.000000, 1.000000}
\pgfsetfillcolor{dialinecolor}
\pgfpathellipse{\pgfpoint{24.500000\du}{11.500000\du}}{\pgfpoint{2.500000\du}{0\du}}{\pgfpoint{0\du}{2.500000\du}}
\pgfusepath{fill}
\pgfsetlinewidth{0.000000\du}
\pgfsetdash{}{0pt}
\pgfsetdash{}{0pt}
\definecolor{dialinecolor}{rgb}{0.000000, 0.000000, 0.000000}
\pgfsetstrokecolor{dialinecolor}
\pgfpathellipse{\pgfpoint{24.500000\du}{11.500000\du}}{\pgfpoint{2.500000\du}{0\du}}{\pgfpoint{0\du}{2.500000\du}}
\pgfusepath{stroke}
\definecolor{dialinecolor}{rgb}{1.000000, 1.000000, 1.000000}
\pgfsetfillcolor{dialinecolor}
\pgfpathellipse{\pgfpoint{14.500000\du}{21.500000\du}}{\pgfpoint{2.500000\du}{0\du}}{\pgfpoint{0\du}{2.500000\du}}
\pgfusepath{fill}
\pgfsetlinewidth{0.000000\du}
\pgfsetdash{}{0pt}
\pgfsetdash{}{0pt}
\definecolor{dialinecolor}{rgb}{0.000000, 0.000000, 0.000000}
\pgfsetstrokecolor{dialinecolor}
\pgfpathellipse{\pgfpoint{14.500000\du}{21.500000\du}}{\pgfpoint{2.500000\du}{0\du}}{\pgfpoint{0\du}{2.500000\du}}
\pgfusepath{stroke}
\definecolor{dialinecolor}{rgb}{1.000000, 1.000000, 1.000000}
\pgfsetfillcolor{dialinecolor}
\pgfpathellipse{\pgfpoint{24.500000\du}{21.500000\du}}{\pgfpoint{2.500000\du}{0\du}}{\pgfpoint{0\du}{2.500000\du}}
\pgfusepath{fill}
\pgfsetlinewidth{0.000000\du}
\pgfsetdash{}{0pt}
\pgfsetdash{}{0pt}
\definecolor{dialinecolor}{rgb}{0.000000, 0.000000, 0.000000}
\pgfsetstrokecolor{dialinecolor}
\pgfpathellipse{\pgfpoint{24.500000\du}{21.500000\du}}{\pgfpoint{2.500000\du}{0\du}}{\pgfpoint{0\du}{2.500000\du}}
\pgfusepath{stroke}
% setfont left to latex
\definecolor{dialinecolor}{rgb}{0.000000, 0.000000, 0.000000}
\pgfsetstrokecolor{dialinecolor}
\node at (14.500000\du,10.921250\du){State 0};
% setfont left to latex
\definecolor{dialinecolor}{rgb}{0.000000, 0.000000, 0.000000}
\pgfsetstrokecolor{dialinecolor}
\node at (14.500000\du,11.721250\du){};
% setfont left to latex
\definecolor{dialinecolor}{rgb}{0.000000, 0.000000, 0.000000}
\pgfsetstrokecolor{dialinecolor}
\node at (14.500000\du,12.521250\du){Idle};
% setfont left to latex
\definecolor{dialinecolor}{rgb}{0.000000, 0.000000, 0.000000}
\pgfsetstrokecolor{dialinecolor}
\node at (24.500000\du,10.521250\du){State 1};
% setfont left to latex
\definecolor{dialinecolor}{rgb}{0.000000, 0.000000, 0.000000}
\pgfsetstrokecolor{dialinecolor}
\node at (24.500000\du,11.321250\du){};
% setfont left to latex
\definecolor{dialinecolor}{rgb}{0.000000, 0.000000, 0.000000}
\pgfsetstrokecolor{dialinecolor}
\node at (24.500000\du,12.121250\du){Send};
% setfont left to latex
\definecolor{dialinecolor}{rgb}{0.000000, 0.000000, 0.000000}
\pgfsetstrokecolor{dialinecolor}
\node at (24.500000\du,12.921250\du){Instruction};
% setfont left to latex
\definecolor{dialinecolor}{rgb}{0.000000, 0.000000, 0.000000}
\pgfsetstrokecolor{dialinecolor}
\node at (14.500000\du,20.921250\du){State 2};
% setfont left to latex
\definecolor{dialinecolor}{rgb}{0.000000, 0.000000, 0.000000}
\pgfsetstrokecolor{dialinecolor}
\node at (14.500000\du,21.721250\du){};
% setfont left to latex
\definecolor{dialinecolor}{rgb}{0.000000, 0.000000, 0.000000}
\pgfsetstrokecolor{dialinecolor}
\node at (14.500000\du,22.521250\du){Done};
% setfont left to latex
\definecolor{dialinecolor}{rgb}{0.000000, 0.000000, 0.000000}
\pgfsetstrokecolor{dialinecolor}
\node at (24.500000\du,20.921250\du){State 3};
% setfont left to latex
\definecolor{dialinecolor}{rgb}{0.000000, 0.000000, 0.000000}
\pgfsetstrokecolor{dialinecolor}
\node at (24.500000\du,21.721250\du){};
% setfont left to latex
\definecolor{dialinecolor}{rgb}{0.000000, 0.000000, 0.000000}
\pgfsetstrokecolor{dialinecolor}
\node at (24.500000\du,22.521250\du){Repeat};
\pgfsetlinewidth{0.000000\du}
\pgfsetdash{}{0pt}
\pgfsetdash{}{0pt}
\pgfsetbuttcap
{
\definecolor{dialinecolor}{rgb}{0.000000, 0.000000, 0.000000}
\pgfsetfillcolor{dialinecolor}
% was here!!!
\pgfsetarrowsend{latex}
\definecolor{dialinecolor}{rgb}{0.000000, 0.000000, 0.000000}
\pgfsetstrokecolor{dialinecolor}
\draw (9.000000\du,11.500000\du)--(12.000000\du,11.500000\du);
}
\pgfsetlinewidth{0.000000\du}
\pgfsetdash{}{0pt}
\pgfsetdash{}{0pt}
\pgfsetbuttcap
{
\definecolor{dialinecolor}{rgb}{0.000000, 0.000000, 0.000000}
\pgfsetfillcolor{dialinecolor}
% was here!!!
\pgfsetarrowsend{latex}
\definecolor{dialinecolor}{rgb}{0.000000, 0.000000, 0.000000}
\pgfsetstrokecolor{dialinecolor}
\draw (17.000000\du,11.500000\du)--(22.000000\du,11.500000\du);
}
\pgfsetlinewidth{0.000000\du}
\pgfsetdash{}{0pt}
\pgfsetdash{}{0pt}
\pgfsetbuttcap
{
\definecolor{dialinecolor}{rgb}{0.000000, 0.000000, 0.000000}
\pgfsetfillcolor{dialinecolor}
% was here!!!
\pgfsetarrowsend{latex}
\definecolor{dialinecolor}{rgb}{0.000000, 0.000000, 0.000000}
\pgfsetstrokecolor{dialinecolor}
\pgfpathmoveto{\pgfpoint{14.499995\du}{9.000110\du}}
\pgfpatharc{3}{-227}{1.054486\du and 1.054486\du}
\pgfusepath{stroke}
}
% setfont left to latex
\definecolor{dialinecolor}{rgb}{0.000000, 0.000000, 0.000000}
\pgfsetstrokecolor{dialinecolor}
\node[anchor=west] at (9.000000\du,11.000000\du){reset};
% setfont left to latex
\definecolor{dialinecolor}{rgb}{0.000000, 0.000000, 0.000000}
\pgfsetstrokecolor{dialinecolor}
\node[anchor=west] at (17.500000\du,11.000000\du){c3s\_start};
% setfont left to latex
\definecolor{dialinecolor}{rgb}{0.000000, 0.000000, 0.000000}
\pgfsetstrokecolor{dialinecolor}
\node[anchor=west] at (11.000000\du,7.500000\du){! c3s\_start};
\pgfsetlinewidth{0.000000\du}
\pgfsetdash{}{0pt}
\pgfsetdash{}{0pt}
\pgfsetbuttcap
{
\definecolor{dialinecolor}{rgb}{0.000000, 0.000000, 0.000000}
\pgfsetfillcolor{dialinecolor}
% was here!!!
\pgfsetarrowsend{latex}
\definecolor{dialinecolor}{rgb}{0.000000, 0.000000, 0.000000}
\pgfsetstrokecolor{dialinecolor}
\draw (24.500000\du,14.000000\du)--(24.500000\du,19.000000\du);
}
\pgfsetlinewidth{0.000000\du}
\pgfsetdash{}{0pt}
\pgfsetdash{}{0pt}
\pgfsetbuttcap
{
\definecolor{dialinecolor}{rgb}{0.000000, 0.000000, 0.000000}
\pgfsetfillcolor{dialinecolor}
% was here!!!
\pgfsetarrowsend{latex}
\definecolor{dialinecolor}{rgb}{0.000000, 0.000000, 0.000000}
\pgfsetstrokecolor{dialinecolor}
\draw (22.732200\du,13.267800\du)--(16.267800\du,19.732200\du);
}
\pgfsetlinewidth{0.000000\du}
\pgfsetdash{}{0pt}
\pgfsetdash{}{0pt}
\pgfsetbuttcap
{
\definecolor{dialinecolor}{rgb}{0.000000, 0.000000, 0.000000}
\pgfsetfillcolor{dialinecolor}
% was here!!!
\pgfsetarrowsend{latex}
\definecolor{dialinecolor}{rgb}{0.000000, 0.000000, 0.000000}
\pgfsetstrokecolor{dialinecolor}
\pgfpathmoveto{\pgfpoint{26.267722\du}{9.732404\du}}
\pgfpatharc{25}{-204}{1.938014\du and 1.938014\du}
\pgfusepath{stroke}
}
% setfont left to latex
\definecolor{dialinecolor}{rgb}{0.000000, 0.000000, 0.000000}
\pgfsetstrokecolor{dialinecolor}
\node[anchor=west] at (19.000000\du,5.500000\du){! last instruction AND};
% setfont left to latex
\definecolor{dialinecolor}{rgb}{0.000000, 0.000000, 0.000000}
\pgfsetstrokecolor{dialinecolor}
\node[anchor=west] at (19.000000\du,6.300000\du){repeat\_count == 0};
% setfont left to latex
\definecolor{dialinecolor}{rgb}{0.000000, 0.000000, 0.000000}
\pgfsetstrokecolor{dialinecolor}
\node[anchor=west] at (16.500000\du,17.500000\du){last instruction};
\pgfsetlinewidth{0.000000\du}
\pgfsetdash{}{0pt}
\pgfsetdash{}{0pt}
\pgfsetbuttcap
{
\definecolor{dialinecolor}{rgb}{0.000000, 0.000000, 0.000000}
\pgfsetfillcolor{dialinecolor}
% was here!!!
\pgfsetarrowsend{latex}
\definecolor{dialinecolor}{rgb}{0.000000, 0.000000, 0.000000}
\pgfsetstrokecolor{dialinecolor}
\draw (14.500000\du,19.000000\du)--(14.500000\du,14.000000\du);
}
\pgfsetlinewidth{0.000000\du}
\pgfsetdash{}{0pt}
\pgfsetdash{}{0pt}
\pgfsetbuttcap
{
\definecolor{dialinecolor}{rgb}{0.000000, 0.000000, 0.000000}
\pgfsetfillcolor{dialinecolor}
% was here!!!
\pgfsetarrowsend{latex}
\definecolor{dialinecolor}{rgb}{0.000000, 0.000000, 0.000000}
\pgfsetstrokecolor{dialinecolor}
\pgfpathmoveto{\pgfpoint{26.267492\du}{19.732650\du}}
\pgfpatharc{35}{-34}{5.723558\du and 5.723558\du}
\pgfusepath{stroke}
}
\pgfsetlinewidth{0.000000\du}
\pgfsetdash{}{0pt}
\pgfsetdash{}{0pt}
\pgfsetbuttcap
{
\definecolor{dialinecolor}{rgb}{0.000000, 0.000000, 0.000000}
\pgfsetfillcolor{dialinecolor}
% was here!!!
\pgfsetarrowsend{latex}
\definecolor{dialinecolor}{rgb}{0.000000, 0.000000, 0.000000}
\pgfsetstrokecolor{dialinecolor}
\pgfpathmoveto{\pgfpoint{24.500002\du}{23.999945\du}}
\pgfpatharc{183}{-47}{1.054482\du and 1.054482\du}
\pgfusepath{stroke}
}
% setfont left to latex
\definecolor{dialinecolor}{rgb}{0.000000, 0.000000, 0.000000}
\pgfsetstrokecolor{dialinecolor}
\node[anchor=west] at (27.000000\du,19.000000\du){repeat\_count\_int == 0};
% setfont left to latex
\definecolor{dialinecolor}{rgb}{0.000000, 0.000000, 0.000000}
\pgfsetstrokecolor{dialinecolor}
\node[anchor=west] at (22.000000\du,26.000000\du){repeat\_count\_int != 0};
% setfont left to latex
\definecolor{dialinecolor}{rgb}{0.000000, 0.000000, 0.000000}
\pgfsetstrokecolor{dialinecolor}
\node[anchor=west] at (21.000000\du,15.500000\du){! last instruction AND};
% setfont left to latex
\definecolor{dialinecolor}{rgb}{0.000000, 0.000000, 0.000000}
\pgfsetstrokecolor{dialinecolor}
\node[anchor=west] at (21.000000\du,16.300000\du){repeat\_count != 0};
\end{tikzpicture}

  \end{center}
  \caption[C3S State Machine]{\label{fig:c3s_fsm}C3S State Machine
  }
\end{figure}

\subsection{Configuration Block}
VHDL Name: c3s\_flow(c3s\_flow.vhdl)

The configuration block contains the single configuration register in
C3S, as well as a means to access the data in each of the 3 arrays. It
functions as an AXI4-Lite slave with an offset address configurable by
application.

Inside of C3S, the local offsets are:

\begin{center}
  \begin{tabular}{l|l}
    \texttt{0x00000} - \texttt{0x0FFFF} & Data Array \\
    \texttt{0x10000} - \texttt{0x1FFFF} & Response Array \\
    \texttt{0x20000} - \texttt{0x2FF00} & Instruction Flow Array (Only addresses with (7:0) = 0 are valid) \\
    \texttt{0x30000}                    & C3S CNTL \\
    \texttt{0x30004}                    & C3S RESP CNTL \\
  \end{tabular}
\end{center}

The main configuration register, C3S\_CNTL, contains a bit that is
written to start C3S execution, as well as to stop it, if
needed. There is also a bit to indicate if C3S is running or not,
which is automatically reset when execution is complete, as well as a
bit to indicate if C3S has run and is done running.

There is also a register to monitor and manipulate where we're writing
in the response array to allow software to better parse the results.

\subsection{Registers}

\textbf{Note}: Because the C3S arrays have generic width and depth
arrays, there are multiple copies of each register. \$\{N\} enumerates
rows, whiles \$\{M\} enumerates columns, with $0 \leq \$\{N\} <
2^{\textrm{C\_ADDRESS\_WIDTH}}$ and $0 \leq \$\{M\} <
\textrm{C\_COMMAND\_WIDTH} / 32$.

Each array takes a max of 8 bits of addresses for a row, and 8 bits of
addressing for a column. If configured for a smaller size, than the
extra registers don't exist and the addresses for them are
invalid. Bits 7:0 of the local address are the column, while bits 15:8
are the row.

The register in column 0 always contains the data for
c3s\_command(31:0), column 1 contains data for c3s\_command(63:32),
etc. Row 0 has command 0, row 1 has command 1, etc. The same logic
holds for c3s\_response(31:0), etc., as well. The response is
prepended with a 32 bit timestamp indicating the capture time. Thus,
if c3s\_response is 512 bits wide, bits 543:512 are the timestamp.

% C3S Data N_M
\subsubsection{C3S Data \$\{N\}\_\$\{M\}}
\begin{tabular}{ r | p{350px} }
  Mnemonic       & C3S\_DATA\_\$\{N\}\_\$\{M\}                                                           \\
  Address Offset & $\$\{N\} * \texttt{0x100} + \$\{M\}$                                                  \\
  VHDL Name      & c3s\_data.bram\_inst.ram\_block($\$\{N\}$)($\$\{M\} * 32 + 31$ downto $\$\{M\} * 32$) \\ \hline

  Description &
  Instruction program steps for C3S to execute. \\
\end{tabular}
\\
\begin{tabularx}{\textwidth}{r|l|l|l|X}
  \hline
  Bit   & Field Mnemonic & Type & Reset & Description \\ \hline

  31:0  & COMMAND        & RW   & 0     &

  C3S command. Directly corresponds to c3s\_command($\$\{M\} * 32 +
  31$ downto $\$\{M\} * 32$). \\
\end{tabularx}

% C3S Response N_M
\subsubsection{C3S Response \$\{N\}\_\$\{M\}}
\begin{tabular}{ r | p{350px} }
  Mnemonic       & C3S\_RESPONSE\_\$\{N\}\_\$\{M\}                                                       \\
  Address Offset & $\$\{N\} * \texttt{0x100} + \$\{M\} + \texttt{0x10000}$                               \\
  VHDL Name      & c3s\_resp.bram\_inst.ram\_block($\$\{N\}$)($\$\{M\} * 32 + 31$ downto $\$\{M\} * 32$) \\ \hline

  Description &
  Captured responses received from the other side of the interface driven by C3S. \\
\end{tabular}
\\
\begin{tabularx}{\textwidth}{r|l|l|l|X}
  \hline
  Bit   & Field Mnemonic & Type & Reset & Description \\ \hline

  31:0  & RESPONSE       & ROX  & 0     &

  C3S response. Directly corresponds to c3s\_response($\$\{M\} * 32 +
  31$ downto $\$\{M\} * 32$). \\
\end{tabularx}

% C3S Response N Timestamp
\subsubsection{C3S Response \$\{N\} Timestamp}
\begin{tabular}{ r | p{350px} }
  Mnemonic       & C3S\_RESPONSE\_\$\{N\}\_TIMESTAMP                                                     \\
  Address Offset & $\$\{N\} * \texttt{0x100} + $C\_COMMAND WIDTH/32$ + \texttt{0x10000}$                 \\
  VHDL Name      & c3s\_resp.bram\_inst.ram\_block($\$\{N\}$)(C\_COMMAND WIDTH + 31 downto C\_COMMAND WIDTH) \\ \hline

  Description &
  Timestamp corresponding to response entry \$\{N\}. The address of
  this register is 1 more than the last entry of C3S
  Response \$\{N\}\_\$\{M\} of each row. \\
\end{tabular}
\\
\begin{tabularx}{\textwidth}{r|l|l|l|X}
  \hline
  Bit   & Field Mnemonic & Type & Reset & Description \\ \hline

  31:0  & TIMESTAMP      & ROX  & 0     &

  Number of cycles since the start of C3S execution that this entry
  was stored in the response array. \\
\end{tabularx}

% C3S Flow N
\subsubsection{C3S Flow \$\{N\}}
\begin{tabular}{ r | p{350px} }
  Mnemonic       & C3S\_FLOW\_\$\{N\}                            \\
  Address Offset & $\$\{N\} * \texttt{0x100} + \texttt{0x20000}$ \\
  VHDL Name      & c3s\_flow.bram\_inst.ram\_block($\$\{N\}$)    \\ \hline

  Description &
  Meta instructions to control the execution of C3S.
\end{tabular}
\\
\begin{tabularx}{\textwidth}{r|l|l|l|X}
  \hline
  Bit   & Field Mnemonic      & Type & Reset & Description \\ \hline

  31    & C3S\_COMMAND\_VALID & RW   & 0     &

  Mark a command as valid. This drives c3s\_command\_valid to
  accompany c3s\_command. The use of this signal depends on
  application, useful only if the driven interface requires it. This
  field is not used internal to C3S, and is optional if the interface
  doesn't require it. \\

  30:15 & Reserved            & RW   & 0     &

  Reserved 0. \\

  14:12  & REPEAT\_COUNT      & RW   & 0     &

  Number of times to repeat command in addition to initial time. Total
  executions of command is $\textrm{REPEAT\_COUNT} + 1.$ \\

  11:9  & Reserved            & RW   & 0     &

  Reserved 0. \\

  8     & LAST\_INSTRUCTION   & RW   & 0     &

  Indicates that this is the last command and C3S execution should
  stop. This command is not executed. \\

  7:0   & NEXT\_INSTRUCTION   & RW   & 0     &

  Indicates the next command to go to after executing this command
  (including loops). \\
\end{tabularx}

% C3S Control
\subsubsection{C3S Control}
\begin{tabular}{ r | p{350px} }
  Mnemonic       & C3S\_CNTL                          \\
  Address Offset & \texttt{0x30000}                   \\
  VHDL Name      & c3s\_cfg.c3s\_cfg\_regs.reg\_00\_q \\ \hline

  Description &
  Generic control register for C3S containing status and start \& stop controls.
\end{tabular}
\\
\begin{tabularx}{\textwidth}{r|l|l|l|X}
  \hline
  Bit   & Field Mnemonic      & Type & Reset & Description \\ \hline

  31:11 & Reserved            & RO   & 0     &

  Reserved 0. \\

  10    & C3S\_TMPL\_B\_MASK  & RW   & 0     &

  When set, only store a suspected template B if there is a non-zero
  opcode in any of the three packets or the data is marked valid. All
  other templates are captured unconditionally. Template B is detected
  by looking only at c3s\_response(465 downto 460), so it's possible
  that some data flits will not be captured. \\

  9     & C3S\_SEL\_DISABLE   & RW   & 0     &

  Set to disable c3s\_command\_sel and force signal to 0. Allows
  execution of C3S without interfering with interface, for debug. Does
  not affect wrap path. \\

  8     & C3S\_WRAP\_ENABLE   & RW   & 0     &

  Set to enable wrap path. Response array will trace c3s\_command
  rather than c3s\_response. c3s\_command output will not be
  affected. \\

  7:4   & Reserved            & RO   & 0     &

  Reserved 0. \\

  3     & C3S\_DONE           & RWX  & 0     &

  Returns 1 when C3S is done running. Returns 0 otherwise. This is set
  automatically by hardware when a C3S execution finishes, and is
  reset automatically when one starts. This bit is designed to be
  polled by software to wait until the end of a C3S run. It can also
  be reset manually to ensure that C3S actually runs. \\

  2     & C3S\_IP             & ROX  & 0     &

  Returns 1 when C3S is running (in progress). Returns 0 otherwise. \\

  1     & C3S\_STOP           & RWX  & 0     &

  Write to 1 to stop C3S. Writing to 0 has no effect. Self-resetting,
  and always returns 0 when read. If C3S is not running, writing to 1
  has no effect. \\

  0     & C3S\_.START         & RWX  & 0     &

  Write to 1 to start C3S. Writing to 0 has no effect. Self-resetting,
  and always returns 0 when read. If C3S is running, writing to 1 has
  no effect. \\
\end{tabularx}

% C3S Response Control
\subsubsection{C3S Response Control}
\begin{tabular}{ r | p{350px} }
  Mnemonic       & C3S\_RESP\_CNTL                    \\
  Address Offset & \texttt{0x30004}                   \\
  VHDL Name      & c3s\_cfg.c3s\_cfg\_regs.reg\_01\_q \\ \hline

  Description &
  Control register to monitor and manipulate writing the response array.
\end{tabular}
\\
\begin{tabularx}{\textwidth}{r|l|l|l|X}
  \hline
  Bit   & Field Mnemonic                & Type & Reset & Description \\ \hline

  31:10 & Reserved                      & RO   & 0     &

  Reserved 0. \\

  9     & C3S\_RESP\_WRITE\_ADDR\_RESET & RWX  & 0     &

  Write to 1 to reset the value of the response addr write pointer to
  0. This can be used to allow a clean slate when running a new C3S
  test, to allow parsing the response array to be a bit more
  predictable. Writing to 0 has no effect. \\

  8     & C3S\_RESP\_OVERFLOW           & RWX  & 0     &

  Indicates that the response array writes have wrapped around and are
  now overwriting previous entries. This is not an error, but
  indicates that old data is being overwritten and the valid data
  wraps around. Write to 0 to reset. This is not reset automatically
  when setting bit 9. \\

  7:0   & C3S\_RESP\_WRITE\_ADDR        & ROX  & 0     &

  Row address of the next entry in the response array to be
  written. This value minus 1 is the most recently written to row. \\
\end{tabularx}

\subsection{Implementations}

\subsubsection{TL/DL Interface}
Name: c3s\_tlx\_dlx(c3s\_rlm.vhdl) \\
C\_CONTROL\_WIDTH = 512 \\
C\_ADDRESS\_WIDTH = 5

In this implementation, C3S directly manipulates the TL/DL interface,
specifically dlx\_tlx\_flit(511 downto 0) and
dlx\_tlx\_flit\_valid. This allows C3S to send a control or data flit
as if originating from the TL. c3s\_control(511 downto 0) are assigned
to dlx\_tlx\_flit(511 downto 0). C3S\_COMMAND\_VALID is assigned to
dlx\_tlx\_flit\_valid. Note that most of the DL content in
dlx\_tlx\_flit(511 downto 448) is discarded by the DL in a control
flit. The only values in the DL content that the TL sends to the DL
are the Data Run Length, Bad Data Flit Indicator and TL Template,
which are in dlx\_tlx\_flit(465 downto 448). dlx\_tlx\_flit(511 downto
465) is not used by the DL during a control flit. See the TL
specification for interface detail. Because C3S doesn't read any
responses from upstream, there is no back pressure and credits must be
manually managed when coding up a control sequence. This can cause
complications when unexpected events happen, such as a replay, that
can invalidate a C3S sequence.

To illustrate how this implementation can be used, what follows is a
very simple example to illustrate the point.

\lstset{language=Perl}
\begin{lstlisting}
# Pseudo-functions to write/read an AXI address
# void write_reg(address, data);
# int read_reg(address);

# Control
write_reg(0x30000, 0x00000100); # Enable wrap

# Data Array
write_reg(0x0000E, 0x00000000); # Instruction 0: Template 0 (Bit 460)
write_reg(0x0010E, 0x00001000); # Instruction 1: Template 1 (Bit 460)

# Flow Array
write_reg(0x20000, 0x80000001); # Valid, GOTO 1
write_reg(0x20100, 0x80003002); # Valid, GOTO 2, repeat 3 times (4 total)
write_reg(0x20200, 0x00000100); # END

# RMW start bit
write_reg(0x30000, 0x1 | read_reg(0x30000));
\end{lstlisting}

This will cause C3S to send a single flit with Template 0, followed by
4 flits with Template 1. Because no opcodes are specified, every
single command is a NOP.

\subsubsection{Internal DL}
Name: c3s\_dlx(c3s\_rlm.vhdl) \\
C\_COMMAND\_WIDTH = 512 \\
C\_ADDRESS\_WIDTH = 5

This implementation is similar to the previous one, with the change
that instead of driving an external interface to the DL, it instead
drives an internal interface. This provides all the same controls as
earlier, with the addition of being able to specifiy the fields
previously discarded by the DL in the DL content: CRC, ACK Count, and
DL2DL ACK. See the TL and DL specification for interface detail.
