\section{Reset} \label{section_reset}

There is a single asynchronous reset pin on Fire. A block of Xilinx IP
synchronizes this reset and distributes it to all logic on the
chip. When this reset is deasserted, all Fire functions and AXI
interconnects will be idle and functional, except for the OpenCAPI
link. At this time, the GTY Transceivers will go through an automatic
reset sequence, and signal the DL when done. The DL will then
automatically start its calibration sequence. When calibration is
complete, the DL will set a register bit indicating it is complete and
OpenCAPI traffic can start.

The asynchronous reset is driven from the BMC on the Apollo board, and
can be overridden by a button on the board too. Code must poll the
register bit indicating the DL is complete before performing any
transactions across OpenCAPI or starting any traffic-generating tests,
such as FBIST or C3S. These tests can be programmed, but not started.

A normal startup procedure would look like the following:

\begin{itemize}

\item
  Assert the reset for both OCMB and Fire.

\item
  Deassert the resets, and perform the I2C operation to set
  DL0\_CONFIG0.CFG\_DL0\_ENABLE in OCMB to enable the link. The DL in
  fire will start calibration automatically.

\item
  Wait for the link to train, polling the register bits in both Fire
  and OCMB to indicate training is complete.

\item
  Configure OCMB as desired via MMIO operations, and configure Fire
  for the desired test, and run tests.

\end{itemize}

\begin{emulation}
  In the emulation model, because the PHY and Xilinx pervasive don't
  exist, the reset interface is different. The core used in emulation
  is a synchronous reset, used to reset all of the logic except for
  the DL. The DL, which is normally held in reset waiting for a signal
  from the PHY, receives these signals as inputs to the logic block. 7
  signals from the transceivers must be set to 1 to start calibration
  in the DL. These signals are tied together into a single signal,
  gtwiz\_done. See the emulation interface signal list for signal
  names.

  Finally, the signal that indicates that DL calibration is complete
  and traffic can start is an internal signal not currently accessible
  via register. In emulation, this signal (FIRE.DLX\_TLX\_LINK\_UP)
  should be accessed directly. When set, traffic is free to begin.

\end{emulation}
