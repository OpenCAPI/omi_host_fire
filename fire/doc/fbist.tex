\graphicspath{ {images/} }

\section{Functional Built-In Self Test (FBIST)} \label{section_fbist}

\subsection{Overview}
FBIST, or Function Built-In Self Test, is the primary stress test for
mainline reads and writes in Fire. It is conceptually based on the
FBIST and MCBIST design of previous generation memory buffers,
although, being on an FPGA, ``built-in'' is a misnomer. A user can
program FBIST to provide a certain profile of traffic, balancing
variables such as read/write balance, command spacing, and test
length. Once FBIST is started, commands are sent to OCMB, and any
reads are checked to ensure that correct data is returned. Writes are
checked indirectly through subsequent reads. At the end of the test, a
pass/fail bit is set, and any failing operations are logged for debug.

The main principle that guides the design of FBIST is that all data is
deterministically generated from the address and minimal
meta-information. This allows for more flexibility in tests, as data
written during the test can be read and checked at any later point
without storing a prohibitive amount of expected data in Fire. There
are five major blocks in FBIST. A command packet is assembled in a
linear flow, as the control path flows only one direction, except for
a single feedback path to stall the pipeline if needed. FBIST
execution starts with the command generator, where an engine creates
the sequence of commands to be executed. These commands then feed into
an address generator, which attaches an address and a tag to each
command. The data generator follows, which generates deterministic
write data (if needed) based on the previous address. These commands,
with address and data, are forwarded over the OpenCAPI link. The
command and address are also stored in a lookup table (LUT). The data
is not, because in order to save resources the data can be generated
from the address in the future as needed. For writes, after the write
response is received the command is marked complete. For reads, the
data that was sent is regenerated and compared against the actual
data.

This chapter details these five blocks, which are shown in
Figure~\ref{fig:fbist_overview}. In this figure, and other figures in
this chapter, a cloud symbol indicates combinational logic that
performs the function described by the text.

Fire contains multiple, but identically configured, OpenCAPI
ports. FBIST is designed to broadcast commands to all configured ports
simultaneously. Each ports has independent checking
logic. Additionally, to keep up with the bandwidth available on
OpenCAPI, multiple commands are generated in parallel. On the other
hand, data is only required once in a cycle, which is advantageous
giving the wide data bus. Figure~\ref{fig:fbist_overview} contains
four command and address generators, with one data generator, that
broadcast to all OpenCAPI ports. Each OpenCAPI port has its own FIFO
and upstream checker. Each pair of command and address generator is
referred to as a ``pool'', which refers to the ability to divide the
commands into completely independent or related groups as desired. For
example, pools can be configured with disjoint address maps to allow
each pool to access a different part of the address space. Each pool
consists of a number of ``engines'' that allow for a mixture of
commands in a pool. FBIST is designed to be able to keep up with a
throughput of one command from each pool every cycle.

\textbf{Note}: In the FBIST description, ``pseudo-random'' refers to an
Linear Feedback Shift Register (LFSR) driven generator. The specifics
vary between applications, but in general it means one of two things:
a signal is asserted when a condition in the LFSR is met, or a signal
vector takes the value of the subset of the LFSR. The length of an
LFSR can be tuned depending on desired probability of an
event. Importantly, while events driven from an LFSR appear
``random'', they are in fact deterministic and repeatable (and even
reversible). This allows apparently random tests to be run multiple
times. Care is taken to ensure that all of FBIST is deterministic. The
only possible source of arbitrary behavior is when we throttle all of
FBIST to compensate for a channel replay. In this scenario, some
commands will be out of sync as the FIFOs empty, although the commands
and results will still be logically accurate. Eventually the channels
will resync. When this occurs can be observed, but the specific timing
will not be reproducible in general.

\begin{figure}[h]
  \begin{center}
    % Graphic for TeX using PGF
% Title: /home/rpking42/proj/chips/astra/doc/images/fbist_overview.dia
% Creator: Dia v0.97.3
% CreationDate: Wed Apr 18 19:24:10 2018
% For: rpking42
% \usepackage{tikz}
% The following commands are not supported in PSTricks at present
% We define them conditionally, so when they are implemented,
% this pgf file will use them.
\ifx\du\undefined
  \newlength{\du}
\fi
\setlength{\du}{15\unitlength}
\begin{tikzpicture}
\pgftransformxscale{1.000000}
\pgftransformyscale{-1.000000}
\definecolor{dialinecolor}{rgb}{0.000000, 0.000000, 0.000000}
\pgfsetstrokecolor{dialinecolor}
\definecolor{dialinecolor}{rgb}{1.000000, 1.000000, 1.000000}
\pgfsetfillcolor{dialinecolor}
\pgfsetlinewidth{0.000000\du}
\pgfsetdash{}{0pt}
\pgfsetdash{}{0pt}
\pgfsetmiterjoin
\definecolor{dialinecolor}{rgb}{1.000000, 1.000000, 1.000000}
\pgfsetfillcolor{dialinecolor}
\fill (11.000000\du,13.000000\du)--(11.000000\du,16.000000\du)--(15.000000\du,16.000000\du)--(15.000000\du,13.000000\du)--cycle;
\definecolor{dialinecolor}{rgb}{0.000000, 0.000000, 0.000000}
\pgfsetstrokecolor{dialinecolor}
\draw (11.000000\du,13.000000\du)--(11.000000\du,16.000000\du)--(15.000000\du,16.000000\du)--(15.000000\du,13.000000\du)--cycle;
% setfont left to latex
\definecolor{dialinecolor}{rgb}{0.000000, 0.000000, 0.000000}
\pgfsetstrokecolor{dialinecolor}
\node at (13.000000\du,14.721250\du){CGEN};
\pgfsetlinewidth{0.000000\du}
\pgfsetdash{}{0pt}
\pgfsetdash{}{0pt}
\pgfsetmiterjoin
\definecolor{dialinecolor}{rgb}{1.000000, 1.000000, 1.000000}
\pgfsetfillcolor{dialinecolor}
\fill (11.000000\du,17.000000\du)--(11.000000\du,20.000000\du)--(15.000000\du,20.000000\du)--(15.000000\du,17.000000\du)--cycle;
\definecolor{dialinecolor}{rgb}{0.000000, 0.000000, 0.000000}
\pgfsetstrokecolor{dialinecolor}
\draw (11.000000\du,17.000000\du)--(11.000000\du,20.000000\du)--(15.000000\du,20.000000\du)--(15.000000\du,17.000000\du)--cycle;
% setfont left to latex
\definecolor{dialinecolor}{rgb}{0.000000, 0.000000, 0.000000}
\pgfsetstrokecolor{dialinecolor}
\node at (13.000000\du,18.721250\du){AGEN};
\pgfsetlinewidth{0.000000\du}
\pgfsetdash{}{0pt}
\pgfsetdash{}{0pt}
\pgfsetbuttcap
{
\definecolor{dialinecolor}{rgb}{0.000000, 0.000000, 0.000000}
\pgfsetfillcolor{dialinecolor}
% was here!!!
\pgfsetarrowsend{latex}
\definecolor{dialinecolor}{rgb}{0.000000, 0.000000, 0.000000}
\pgfsetstrokecolor{dialinecolor}
\draw (13.000000\du,16.000000\du)--(13.000000\du,17.000000\du);
}
\pgfsetlinewidth{0.000000\du}
\pgfsetdash{}{0pt}
\pgfsetdash{}{0pt}
\pgfsetmiterjoin
\definecolor{dialinecolor}{rgb}{1.000000, 1.000000, 1.000000}
\pgfsetfillcolor{dialinecolor}
\fill (15.500000\du,13.000000\du)--(15.500000\du,16.000000\du)--(19.500000\du,16.000000\du)--(19.500000\du,13.000000\du)--cycle;
\definecolor{dialinecolor}{rgb}{0.000000, 0.000000, 0.000000}
\pgfsetstrokecolor{dialinecolor}
\draw (15.500000\du,13.000000\du)--(15.500000\du,16.000000\du)--(19.500000\du,16.000000\du)--(19.500000\du,13.000000\du)--cycle;
% setfont left to latex
\definecolor{dialinecolor}{rgb}{0.000000, 0.000000, 0.000000}
\pgfsetstrokecolor{dialinecolor}
\node at (17.500000\du,14.721250\du){CGEN};
\pgfsetlinewidth{0.000000\du}
\pgfsetdash{}{0pt}
\pgfsetdash{}{0pt}
\pgfsetmiterjoin
\definecolor{dialinecolor}{rgb}{1.000000, 1.000000, 1.000000}
\pgfsetfillcolor{dialinecolor}
\fill (15.500000\du,17.000000\du)--(15.500000\du,20.000000\du)--(19.500000\du,20.000000\du)--(19.500000\du,17.000000\du)--cycle;
\definecolor{dialinecolor}{rgb}{0.000000, 0.000000, 0.000000}
\pgfsetstrokecolor{dialinecolor}
\draw (15.500000\du,17.000000\du)--(15.500000\du,20.000000\du)--(19.500000\du,20.000000\du)--(19.500000\du,17.000000\du)--cycle;
% setfont left to latex
\definecolor{dialinecolor}{rgb}{0.000000, 0.000000, 0.000000}
\pgfsetstrokecolor{dialinecolor}
\node at (17.500000\du,18.721250\du){AGEN};
\pgfsetlinewidth{0.000000\du}
\pgfsetdash{}{0pt}
\pgfsetdash{}{0pt}
\pgfsetbuttcap
{
\definecolor{dialinecolor}{rgb}{0.000000, 0.000000, 0.000000}
\pgfsetfillcolor{dialinecolor}
% was here!!!
\pgfsetarrowsend{latex}
\definecolor{dialinecolor}{rgb}{0.000000, 0.000000, 0.000000}
\pgfsetstrokecolor{dialinecolor}
\draw (17.500000\du,16.000000\du)--(17.500000\du,17.000000\du);
}
\pgfsetlinewidth{0.000000\du}
\pgfsetdash{}{0pt}
\pgfsetdash{}{0pt}
\pgfsetmiterjoin
\definecolor{dialinecolor}{rgb}{1.000000, 1.000000, 1.000000}
\pgfsetfillcolor{dialinecolor}
\fill (20.000000\du,13.000000\du)--(20.000000\du,16.000000\du)--(24.000000\du,16.000000\du)--(24.000000\du,13.000000\du)--cycle;
\definecolor{dialinecolor}{rgb}{0.000000, 0.000000, 0.000000}
\pgfsetstrokecolor{dialinecolor}
\draw (20.000000\du,13.000000\du)--(20.000000\du,16.000000\du)--(24.000000\du,16.000000\du)--(24.000000\du,13.000000\du)--cycle;
% setfont left to latex
\definecolor{dialinecolor}{rgb}{0.000000, 0.000000, 0.000000}
\pgfsetstrokecolor{dialinecolor}
\node at (22.000000\du,14.721250\du){CGEN};
\pgfsetlinewidth{0.000000\du}
\pgfsetdash{}{0pt}
\pgfsetdash{}{0pt}
\pgfsetmiterjoin
\definecolor{dialinecolor}{rgb}{1.000000, 1.000000, 1.000000}
\pgfsetfillcolor{dialinecolor}
\fill (20.000000\du,17.000000\du)--(20.000000\du,20.000000\du)--(24.000000\du,20.000000\du)--(24.000000\du,17.000000\du)--cycle;
\definecolor{dialinecolor}{rgb}{0.000000, 0.000000, 0.000000}
\pgfsetstrokecolor{dialinecolor}
\draw (20.000000\du,17.000000\du)--(20.000000\du,20.000000\du)--(24.000000\du,20.000000\du)--(24.000000\du,17.000000\du)--cycle;
% setfont left to latex
\definecolor{dialinecolor}{rgb}{0.000000, 0.000000, 0.000000}
\pgfsetstrokecolor{dialinecolor}
\node at (22.000000\du,18.721250\du){AGEN};
\pgfsetlinewidth{0.000000\du}
\pgfsetdash{}{0pt}
\pgfsetdash{}{0pt}
\pgfsetbuttcap
{
\definecolor{dialinecolor}{rgb}{0.000000, 0.000000, 0.000000}
\pgfsetfillcolor{dialinecolor}
% was here!!!
\pgfsetarrowsend{latex}
\definecolor{dialinecolor}{rgb}{0.000000, 0.000000, 0.000000}
\pgfsetstrokecolor{dialinecolor}
\draw (22.000000\du,16.000000\du)--(22.000000\du,17.000000\du);
}
\pgfsetlinewidth{0.000000\du}
\pgfsetdash{}{0pt}
\pgfsetdash{}{0pt}
\pgfsetmiterjoin
\definecolor{dialinecolor}{rgb}{1.000000, 1.000000, 1.000000}
\pgfsetfillcolor{dialinecolor}
\fill (24.500000\du,13.000000\du)--(24.500000\du,16.000000\du)--(28.500000\du,16.000000\du)--(28.500000\du,13.000000\du)--cycle;
\definecolor{dialinecolor}{rgb}{0.000000, 0.000000, 0.000000}
\pgfsetstrokecolor{dialinecolor}
\draw (24.500000\du,13.000000\du)--(24.500000\du,16.000000\du)--(28.500000\du,16.000000\du)--(28.500000\du,13.000000\du)--cycle;
% setfont left to latex
\definecolor{dialinecolor}{rgb}{0.000000, 0.000000, 0.000000}
\pgfsetstrokecolor{dialinecolor}
\node at (26.500000\du,14.721250\du){CGEN};
\pgfsetlinewidth{0.000000\du}
\pgfsetdash{}{0pt}
\pgfsetdash{}{0pt}
\pgfsetmiterjoin
\definecolor{dialinecolor}{rgb}{1.000000, 1.000000, 1.000000}
\pgfsetfillcolor{dialinecolor}
\fill (24.500000\du,17.000000\du)--(24.500000\du,20.000000\du)--(28.500000\du,20.000000\du)--(28.500000\du,17.000000\du)--cycle;
\definecolor{dialinecolor}{rgb}{0.000000, 0.000000, 0.000000}
\pgfsetstrokecolor{dialinecolor}
\draw (24.500000\du,17.000000\du)--(24.500000\du,20.000000\du)--(28.500000\du,20.000000\du)--(28.500000\du,17.000000\du)--cycle;
% setfont left to latex
\definecolor{dialinecolor}{rgb}{0.000000, 0.000000, 0.000000}
\pgfsetstrokecolor{dialinecolor}
\node at (26.500000\du,18.721250\du){AGEN};
\pgfsetlinewidth{0.000000\du}
\pgfsetdash{}{0pt}
\pgfsetdash{}{0pt}
\pgfsetbuttcap
{
\definecolor{dialinecolor}{rgb}{0.000000, 0.000000, 0.000000}
\pgfsetfillcolor{dialinecolor}
% was here!!!
\pgfsetarrowsend{latex}
\definecolor{dialinecolor}{rgb}{0.000000, 0.000000, 0.000000}
\pgfsetstrokecolor{dialinecolor}
\draw (26.500000\du,16.000000\du)--(26.500000\du,17.000000\du);
}
\pgfsetlinewidth{0.000000\du}
\pgfsetdash{}{0pt}
\pgfsetdash{}{0pt}
\pgfsetmiterjoin
\definecolor{dialinecolor}{rgb}{1.000000, 1.000000, 1.000000}
\pgfsetfillcolor{dialinecolor}
\fill (30.000000\du,13.000000\du)--(30.000000\du,16.000000\du)--(34.000000\du,16.000000\du)--(34.000000\du,13.000000\du)--cycle;
\definecolor{dialinecolor}{rgb}{0.000000, 0.000000, 0.000000}
\pgfsetstrokecolor{dialinecolor}
\draw (30.000000\du,13.000000\du)--(30.000000\du,16.000000\du)--(34.000000\du,16.000000\du)--(34.000000\du,13.000000\du)--cycle;
% setfont left to latex
\definecolor{dialinecolor}{rgb}{0.000000, 0.000000, 0.000000}
\pgfsetstrokecolor{dialinecolor}
\node at (32.000000\du,14.721250\du){LUT};
\pgfsetlinewidth{0.000000\du}
\pgfsetdash{}{0pt}
\pgfsetdash{}{0pt}
\pgfsetmiterjoin
\definecolor{dialinecolor}{rgb}{1.000000, 1.000000, 1.000000}
\pgfsetfillcolor{dialinecolor}
\fill (17.000000\du,24.000000\du)--(17.000000\du,27.000000\du)--(22.000000\du,27.000000\du)--(22.000000\du,24.000000\du)--cycle;
\definecolor{dialinecolor}{rgb}{0.000000, 0.000000, 0.000000}
\pgfsetstrokecolor{dialinecolor}
\draw (17.000000\du,24.000000\du)--(17.000000\du,27.000000\du)--(22.000000\du,27.000000\du)--(22.000000\du,24.000000\du)--cycle;
% setfont left to latex
\definecolor{dialinecolor}{rgb}{0.000000, 0.000000, 0.000000}
\pgfsetstrokecolor{dialinecolor}
\node at (19.500000\du,25.721250\du){DGEN};
\pgfsetlinewidth{0.000000\du}
\pgfsetdash{}{0pt}
\pgfsetdash{}{0pt}
\pgfsetmiterjoin
\pgfsetbuttcap
{
\definecolor{dialinecolor}{rgb}{0.000000, 0.000000, 0.000000}
\pgfsetfillcolor{dialinecolor}
% was here!!!
\pgfsetarrowsend{latex}
{\pgfsetcornersarced{\pgfpoint{0.000000\du}{0.000000\du}}\definecolor{dialinecolor}{rgb}{0.000000, 0.000000, 0.000000}
\pgfsetstrokecolor{dialinecolor}
\draw (13.000000\du,20.000000\du)--(13.000000\du,23.000000\du)--(18.000000\du,23.000000\du)--(18.000000\du,24.000000\du);
}}
\pgfsetlinewidth{0.000000\du}
\pgfsetdash{}{0pt}
\pgfsetdash{}{0pt}
\pgfsetmiterjoin
\pgfsetbuttcap
{
\definecolor{dialinecolor}{rgb}{0.000000, 0.000000, 0.000000}
\pgfsetfillcolor{dialinecolor}
% was here!!!
\pgfsetarrowsend{latex}
{\pgfsetcornersarced{\pgfpoint{0.000000\du}{0.000000\du}}\definecolor{dialinecolor}{rgb}{0.000000, 0.000000, 0.000000}
\pgfsetstrokecolor{dialinecolor}
\draw (17.500000\du,20.000000\du)--(17.500000\du,22.600000\du)--(19.000000\du,22.600000\du)--(19.000000\du,24.000000\du);
}}
\pgfsetlinewidth{0.000000\du}
\pgfsetdash{}{0pt}
\pgfsetdash{}{0pt}
\pgfsetmiterjoin
\pgfsetbuttcap
{
\definecolor{dialinecolor}{rgb}{0.000000, 0.000000, 0.000000}
\pgfsetfillcolor{dialinecolor}
% was here!!!
\pgfsetarrowsend{latex}
{\pgfsetcornersarced{\pgfpoint{0.000000\du}{0.000000\du}}\definecolor{dialinecolor}{rgb}{0.000000, 0.000000, 0.000000}
\pgfsetstrokecolor{dialinecolor}
\draw (22.000000\du,20.000000\du)--(22.000000\du,22.600000\du)--(20.000000\du,22.600000\du)--(20.000000\du,24.000000\du);
}}
\pgfsetlinewidth{0.000000\du}
\pgfsetdash{}{0pt}
\pgfsetdash{}{0pt}
\pgfsetmiterjoin
\pgfsetbuttcap
{
\definecolor{dialinecolor}{rgb}{0.000000, 0.000000, 0.000000}
\pgfsetfillcolor{dialinecolor}
% was here!!!
\pgfsetarrowsend{latex}
{\pgfsetcornersarced{\pgfpoint{0.000000\du}{0.000000\du}}\definecolor{dialinecolor}{rgb}{0.000000, 0.000000, 0.000000}
\pgfsetstrokecolor{dialinecolor}
\draw (26.500000\du,20.000000\du)--(26.500000\du,23.000000\du)--(21.000000\du,23.000000\du)--(21.000000\du,24.000000\du);
}}
\pgfsetlinewidth{0.000000\du}
\pgfsetdash{}{0pt}
\pgfsetdash{}{0pt}
\pgfsetbuttcap
{
\definecolor{dialinecolor}{rgb}{0.000000, 0.000000, 0.000000}
\pgfsetfillcolor{dialinecolor}
% was here!!!
\definecolor{dialinecolor}{rgb}{0.000000, 0.000000, 0.000000}
\pgfsetstrokecolor{dialinecolor}
\draw (17.000000\du,24.000000\du)--(22.000000\du,24.000000\du);
}
\pgfsetlinewidth{0.000000\du}
\pgfsetdash{}{0pt}
\pgfsetdash{}{0pt}
\pgfsetmiterjoin
\definecolor{dialinecolor}{rgb}{1.000000, 1.000000, 1.000000}
\pgfsetfillcolor{dialinecolor}
\fill (30.000000\du,9.000000\du)--(30.000000\du,12.000000\du)--(34.000000\du,12.000000\du)--(34.000000\du,9.000000\du)--cycle;
\definecolor{dialinecolor}{rgb}{0.000000, 0.000000, 0.000000}
\pgfsetstrokecolor{dialinecolor}
\draw (30.000000\du,9.000000\du)--(30.000000\du,12.000000\du)--(34.000000\du,12.000000\du)--(34.000000\du,9.000000\du)--cycle;
% setfont left to latex
\definecolor{dialinecolor}{rgb}{0.000000, 0.000000, 0.000000}
\pgfsetstrokecolor{dialinecolor}
\node at (32.000000\du,10.721250\du){DGEN};
\pgfsetlinewidth{0.000000\du}
\pgfsetdash{}{0pt}
\pgfsetdash{}{0pt}
\pgfsetbuttcap
\pgfsetmiterjoin
\pgfsetlinewidth{0.000000\du}
\pgfsetbuttcap
\pgfsetmiterjoin
\pgfsetdash{}{0pt}
\definecolor{dialinecolor}{rgb}{0.000000, 0.000000, 0.000000}
\pgfsetstrokecolor{dialinecolor}
\draw (30.800000\du,18.000000\du)--(33.200000\du,18.000000\du)--(32.600000\du,17.000000\du)--(31.400000\du,17.000000\du)--cycle;
% setfont left to latex
\definecolor{dialinecolor}{rgb}{0.000000, 0.000000, 0.000000}
\pgfsetstrokecolor{dialinecolor}
\node at (32.000000\du,17.700000\du){};
\pgfsetlinewidth{0.000000\du}
\pgfsetdash{}{0pt}
\pgfsetdash{}{0pt}
\pgfsetbuttcap
{
\definecolor{dialinecolor}{rgb}{0.000000, 0.000000, 0.000000}
\pgfsetfillcolor{dialinecolor}
% was here!!!
\definecolor{dialinecolor}{rgb}{0.000000, 0.000000, 0.000000}
\pgfsetstrokecolor{dialinecolor}
\draw (31.400000\du,18.000000\du)--(32.600000\du,18.000000\du);
}
\pgfsetlinewidth{0.000000\du}
\pgfsetdash{}{0pt}
\pgfsetdash{}{0pt}
\pgfsetbuttcap
{
\definecolor{dialinecolor}{rgb}{0.000000, 0.000000, 0.000000}
\pgfsetfillcolor{dialinecolor}
% was here!!!
\definecolor{dialinecolor}{rgb}{0.000000, 0.000000, 0.000000}
\pgfsetstrokecolor{dialinecolor}
\draw (32.000000\du,17.000000\du)--(32.000000\du,16.000000\du);
}
\pgfsetlinewidth{0.000000\du}
\pgfsetdash{}{0pt}
\pgfsetdash{}{0pt}
\pgfsetbuttcap
{
\definecolor{dialinecolor}{rgb}{0.000000, 0.000000, 0.000000}
\pgfsetfillcolor{dialinecolor}
% was here!!!
\pgfsetarrowsend{latex}
\definecolor{dialinecolor}{rgb}{0.000000, 0.000000, 0.000000}
\pgfsetstrokecolor{dialinecolor}
\draw (32.000000\du,13.000000\du)--(32.000000\du,12.000000\du);
}
\pgfsetlinewidth{0.000000\du}
\pgfsetdash{}{0pt}
\pgfsetdash{}{0pt}
\pgfsetmiterjoin
\pgfsetbuttcap
{
\definecolor{dialinecolor}{rgb}{0.000000, 0.000000, 0.000000}
\pgfsetfillcolor{dialinecolor}
% was here!!!
{\pgfsetcornersarced{\pgfpoint{0.000000\du}{0.000000\du}}\definecolor{dialinecolor}{rgb}{0.000000, 0.000000, 0.000000}
\pgfsetstrokecolor{dialinecolor}
\draw (31.400000\du,18.000000\du)--(31.400000\du,20.500000\du)--(26.500000\du,20.500000\du)--(26.500000\du,20.000000\du);
}}
\pgfsetlinewidth{0.000000\du}
\pgfsetdash{}{0pt}
\pgfsetdash{}{0pt}
\pgfsetmiterjoin
\pgfsetbuttcap
{
\definecolor{dialinecolor}{rgb}{0.000000, 0.000000, 0.000000}
\pgfsetfillcolor{dialinecolor}
% was here!!!
{\pgfsetcornersarced{\pgfpoint{0.000000\du}{0.000000\du}}\definecolor{dialinecolor}{rgb}{0.000000, 0.000000, 0.000000}
\pgfsetstrokecolor{dialinecolor}
\draw (31.800000\du,18.000000\du)--(31.800000\du,21.000000\du)--(22.000000\du,21.000000\du)--(22.000000\du,20.000000\du);
}}
\pgfsetlinewidth{0.000000\du}
\pgfsetdash{}{0pt}
\pgfsetdash{}{0pt}
\pgfsetmiterjoin
\pgfsetbuttcap
{
\definecolor{dialinecolor}{rgb}{0.000000, 0.000000, 0.000000}
\pgfsetfillcolor{dialinecolor}
% was here!!!
{\pgfsetcornersarced{\pgfpoint{0.000000\du}{0.000000\du}}\definecolor{dialinecolor}{rgb}{0.000000, 0.000000, 0.000000}
\pgfsetstrokecolor{dialinecolor}
\draw (32.200000\du,18.000000\du)--(32.200000\du,21.500000\du)--(17.500000\du,21.500000\du)--(17.500000\du,20.000000\du);
}}
\pgfsetlinewidth{0.000000\du}
\pgfsetdash{}{0pt}
\pgfsetdash{}{0pt}
\pgfsetmiterjoin
\pgfsetbuttcap
{
\definecolor{dialinecolor}{rgb}{0.000000, 0.000000, 0.000000}
\pgfsetfillcolor{dialinecolor}
% was here!!!
{\pgfsetcornersarced{\pgfpoint{0.000000\du}{0.000000\du}}\definecolor{dialinecolor}{rgb}{0.000000, 0.000000, 0.000000}
\pgfsetstrokecolor{dialinecolor}
\draw (32.600000\du,18.000000\du)--(32.600000\du,22.000000\du)--(13.000000\du,22.000000\du)--(13.000000\du,20.000000\du);
}}
\pgfsetlinewidth{0.000000\du}
\pgfsetdash{}{0pt}
\pgfsetdash{}{0pt}
\pgfsetbuttcap
\pgfsetmiterjoin
\pgfsetlinewidth{0.000000\du}
\pgfsetbuttcap
\pgfsetmiterjoin
\pgfsetdash{}{0pt}
\pgfsetbuttcap
\pgfsetmiterjoin
\pgfsetdash{}{0pt}
\definecolor{dialinecolor}{rgb}{0.000000, 0.000000, 0.000000}
\pgfsetfillcolor{dialinecolor}
\pgfpathellipse{\pgfpoint{13.000000\du}{22.000000\du}}{\pgfpoint{0.125000\du}{0\du}}{\pgfpoint{0\du}{0.125000\du}}
\pgfusepath{fill}
\definecolor{dialinecolor}{rgb}{0.000000, 0.000000, 0.000000}
\pgfsetstrokecolor{dialinecolor}
\pgfpathellipse{\pgfpoint{13.000000\du}{22.000000\du}}{\pgfpoint{0.125000\du}{0\du}}{\pgfpoint{0\du}{0.125000\du}}
\pgfusepath{stroke}
\pgfsetlinewidth{0.000000\du}
\pgfsetdash{}{0pt}
\pgfsetdash{}{0pt}
\pgfsetbuttcap
\pgfsetmiterjoin
\pgfsetlinewidth{0.000000\du}
\pgfsetbuttcap
\pgfsetmiterjoin
\pgfsetdash{}{0pt}
\pgfsetbuttcap
\pgfsetmiterjoin
\pgfsetdash{}{0pt}
\definecolor{dialinecolor}{rgb}{0.000000, 0.000000, 0.000000}
\pgfsetfillcolor{dialinecolor}
\pgfpathellipse{\pgfpoint{17.500000\du}{21.500000\du}}{\pgfpoint{0.125000\du}{0\du}}{\pgfpoint{0\du}{0.125000\du}}
\pgfusepath{fill}
\definecolor{dialinecolor}{rgb}{0.000000, 0.000000, 0.000000}
\pgfsetstrokecolor{dialinecolor}
\pgfpathellipse{\pgfpoint{17.500000\du}{21.500000\du}}{\pgfpoint{0.125000\du}{0\du}}{\pgfpoint{0\du}{0.125000\du}}
\pgfusepath{stroke}
\pgfsetlinewidth{0.000000\du}
\pgfsetdash{}{0pt}
\pgfsetdash{}{0pt}
\pgfsetbuttcap
\pgfsetmiterjoin
\pgfsetlinewidth{0.000000\du}
\pgfsetbuttcap
\pgfsetmiterjoin
\pgfsetdash{}{0pt}
\pgfsetbuttcap
\pgfsetmiterjoin
\pgfsetdash{}{0pt}
\definecolor{dialinecolor}{rgb}{0.000000, 0.000000, 0.000000}
\pgfsetfillcolor{dialinecolor}
\pgfpathellipse{\pgfpoint{22.000000\du}{21.000000\du}}{\pgfpoint{0.125000\du}{0\du}}{\pgfpoint{0\du}{0.125000\du}}
\pgfusepath{fill}
\definecolor{dialinecolor}{rgb}{0.000000, 0.000000, 0.000000}
\pgfsetstrokecolor{dialinecolor}
\pgfpathellipse{\pgfpoint{22.000000\du}{21.000000\du}}{\pgfpoint{0.125000\du}{0\du}}{\pgfpoint{0\du}{0.125000\du}}
\pgfusepath{stroke}
\pgfsetlinewidth{0.000000\du}
\pgfsetdash{}{0pt}
\pgfsetdash{}{0pt}
\pgfsetbuttcap
\pgfsetmiterjoin
\pgfsetlinewidth{0.000000\du}
\pgfsetbuttcap
\pgfsetmiterjoin
\pgfsetdash{}{0pt}
\pgfsetbuttcap
\pgfsetmiterjoin
\pgfsetdash{}{0pt}
\definecolor{dialinecolor}{rgb}{0.000000, 0.000000, 0.000000}
\pgfsetfillcolor{dialinecolor}
\pgfpathellipse{\pgfpoint{26.500000\du}{20.500000\du}}{\pgfpoint{0.125000\du}{0\du}}{\pgfpoint{0\du}{0.125000\du}}
\pgfusepath{fill}
\definecolor{dialinecolor}{rgb}{0.000000, 0.000000, 0.000000}
\pgfsetstrokecolor{dialinecolor}
\pgfpathellipse{\pgfpoint{26.500000\du}{20.500000\du}}{\pgfpoint{0.125000\du}{0\du}}{\pgfpoint{0\du}{0.125000\du}}
\pgfusepath{stroke}
\pgfsetlinewidth{0.000000\du}
\pgfsetdash{}{0pt}
\pgfsetdash{}{0pt}
\pgfsetbuttcap
\pgfsetmiterjoin
\pgfsetlinewidth{0.000000\du}
\pgfsetbuttcap
\pgfsetmiterjoin
\pgfsetdash{}{0pt}
\definecolor{dialinecolor}{rgb}{0.000000, 0.000000, 0.000000}
\pgfsetstrokecolor{dialinecolor}
\pgfpathmoveto{\pgfpoint{37.680000\du}{13.000000\du}}
\pgfpathcurveto{\pgfpoint{37.311059\du}{12.367961\du}}{\pgfpoint{36.712569\du}{12.000000\du}}{\pgfpoint{36.080000\du}{12.016294\du}}
\pgfusepath{stroke}
\pgfsetbuttcap
\pgfsetmiterjoin
\pgfsetdash{}{0pt}
\definecolor{dialinecolor}{rgb}{0.000000, 0.000000, 0.000000}
\pgfsetstrokecolor{dialinecolor}
\pgfpathmoveto{\pgfpoint{36.080000\du}{13.983706\du}}
\pgfpathcurveto{\pgfpoint{36.712569\du}{14.000000\du}}{\pgfpoint{37.311059\du}{13.632039\du}}{\pgfpoint{37.680000\du}{13.000000\du}}
\pgfusepath{stroke}
\pgfsetbuttcap
\pgfsetmiterjoin
\pgfsetdash{}{0pt}
\definecolor{dialinecolor}{rgb}{0.000000, 0.000000, 0.000000}
\pgfsetstrokecolor{dialinecolor}
\pgfpathmoveto{\pgfpoint{36.000000\du}{13.983706\du}}
\pgfpathcurveto{\pgfpoint{36.320000\du}{13.386925\du}}{\pgfpoint{36.320000\du}{12.613075\du}}{\pgfpoint{36.000000\du}{12.016294\du}}
\pgfusepath{stroke}
\pgfsetbuttcap
\pgfsetmiterjoin
\pgfsetdash{}{0pt}
\definecolor{dialinecolor}{rgb}{0.000000, 0.000000, 0.000000}
\pgfsetstrokecolor{dialinecolor}
\draw (37.680000\du,13.000000\du)--(38.000000\du,13.000000\du);
\pgfsetbuttcap
\pgfsetmiterjoin
\pgfsetdash{}{0pt}
\definecolor{dialinecolor}{rgb}{0.000000, 0.000000, 0.000000}
\pgfsetstrokecolor{dialinecolor}
\pgfpathmoveto{\pgfpoint{36.080000\du}{13.983706\du}}
\pgfpathcurveto{\pgfpoint{36.400000\du}{13.386925\du}}{\pgfpoint{36.400000\du}{12.613075\du}}{\pgfpoint{36.080000\du}{12.016294\du}}
\pgfusepath{stroke}
\pgfsetlinewidth{0.000000\du}
\pgfsetdash{}{0pt}
\pgfsetdash{}{0pt}
\pgfsetmiterjoin
\pgfsetbuttcap
{
\definecolor{dialinecolor}{rgb}{0.000000, 0.000000, 0.000000}
\pgfsetfillcolor{dialinecolor}
% was here!!!
{\pgfsetcornersarced{\pgfpoint{0.000000\du}{0.000000\du}}\definecolor{dialinecolor}{rgb}{0.000000, 0.000000, 0.000000}
\pgfsetstrokecolor{dialinecolor}
\draw (35.200000\du,17.000000\du)--(35.200000\du,17.000000\du)--(35.200000\du,13.491853\du)--(36.176000\du,13.491853\du);
}}
\pgfsetlinewidth{0.000000\du}
\pgfsetdash{}{0pt}
\pgfsetdash{}{0pt}
\pgfsetmiterjoin
\pgfsetbuttcap
{
\definecolor{dialinecolor}{rgb}{0.000000, 0.000000, 0.000000}
\pgfsetfillcolor{dialinecolor}
% was here!!!
{\pgfsetcornersarced{\pgfpoint{0.000000\du}{0.000000\du}}\definecolor{dialinecolor}{rgb}{0.000000, 0.000000, 0.000000}
\pgfsetstrokecolor{dialinecolor}
\draw (34.000000\du,10.500000\du)--(35.200000\du,10.500000\du)--(35.200000\du,12.508147\du)--(36.176000\du,12.508147\du);
}}
\pgfsetlinewidth{0.000000\du}
\pgfsetdash{}{0pt}
\pgfsetdash{}{0pt}
\pgfsetbuttcap
{
\definecolor{dialinecolor}{rgb}{0.000000, 0.000000, 0.000000}
\pgfsetfillcolor{dialinecolor}
% was here!!!
\pgfsetarrowsend{latex}
\definecolor{dialinecolor}{rgb}{0.000000, 0.000000, 0.000000}
\pgfsetstrokecolor{dialinecolor}
\draw (38.000000\du,13.000000\du)--(39.000000\du,13.000000\du);
}
% setfont left to latex
\definecolor{dialinecolor}{rgb}{0.000000, 0.000000, 0.000000}
\pgfsetstrokecolor{dialinecolor}
\node[anchor=west] at (37.500000\du,14.221250\du){ Logging};
\pgfsetlinewidth{0.000000\du}
\pgfsetdash{}{0pt}
\pgfsetdash{}{0pt}
\pgfsetmiterjoin
\definecolor{dialinecolor}{rgb}{1.000000, 1.000000, 1.000000}
\pgfsetfillcolor{dialinecolor}
\fill (17.000000\du,34.000000\du)--(17.000000\du,37.000000\du)--(22.000000\du,37.000000\du)--(22.000000\du,34.000000\du)--cycle;
\definecolor{dialinecolor}{rgb}{0.000000, 0.000000, 0.000000}
\pgfsetstrokecolor{dialinecolor}
\draw (17.000000\du,34.000000\du)--(17.000000\du,37.000000\du)--(22.000000\du,37.000000\du)--(22.000000\du,34.000000\du)--cycle;
% setfont left to latex
\definecolor{dialinecolor}{rgb}{0.000000, 0.000000, 0.000000}
\pgfsetstrokecolor{dialinecolor}
\node at (19.500000\du,35.721250\du){TLT};
\pgfsetlinewidth{0.000000\du}
\pgfsetdash{}{0pt}
\pgfsetdash{}{0pt}
\pgfsetmiterjoin
\definecolor{dialinecolor}{rgb}{1.000000, 1.000000, 1.000000}
\pgfsetfillcolor{dialinecolor}
\fill (23.000000\du,34.000000\du)--(23.000000\du,37.000000\du)--(28.000000\du,37.000000\du)--(28.000000\du,34.000000\du)--cycle;
\definecolor{dialinecolor}{rgb}{0.000000, 0.000000, 0.000000}
\pgfsetstrokecolor{dialinecolor}
\draw (23.000000\du,34.000000\du)--(23.000000\du,37.000000\du)--(28.000000\du,37.000000\du)--(28.000000\du,34.000000\du)--cycle;
% setfont left to latex
\definecolor{dialinecolor}{rgb}{0.000000, 0.000000, 0.000000}
\pgfsetstrokecolor{dialinecolor}
\node at (25.500000\du,35.721250\du){TLR};
\pgfsetlinewidth{0.000000\du}
\pgfsetdash{}{0pt}
\pgfsetdash{}{0pt}
\pgfsetmiterjoin
\definecolor{dialinecolor}{rgb}{1.000000, 1.000000, 1.000000}
\pgfsetfillcolor{dialinecolor}
\fill (17.000000\du,29.000000\du)--(17.000000\du,32.000000\du)--(22.000000\du,32.000000\du)--(22.000000\du,29.000000\du)--cycle;
\definecolor{dialinecolor}{rgb}{0.000000, 0.000000, 0.000000}
\pgfsetstrokecolor{dialinecolor}
\draw (17.000000\du,29.000000\du)--(17.000000\du,32.000000\du)--(22.000000\du,32.000000\du)--(22.000000\du,29.000000\du)--cycle;
% setfont left to latex
\definecolor{dialinecolor}{rgb}{0.000000, 0.000000, 0.000000}
\pgfsetstrokecolor{dialinecolor}
\node at (19.500000\du,30.721250\du){FIFO};
\pgfsetlinewidth{0.000000\du}
\pgfsetdash{}{0pt}
\pgfsetdash{}{0pt}
\pgfsetbuttcap
{
\definecolor{dialinecolor}{rgb}{0.000000, 0.000000, 0.000000}
\pgfsetfillcolor{dialinecolor}
% was here!!!
\definecolor{dialinecolor}{rgb}{0.000000, 0.000000, 0.000000}
\pgfsetstrokecolor{dialinecolor}
\draw (23.000000\du,34.000000\du)--(28.000000\du,34.000000\du);
}
\pgfsetlinewidth{0.000000\du}
\pgfsetdash{}{0pt}
\pgfsetdash{}{0pt}
\pgfsetmiterjoin
\pgfsetbuttcap
{
\definecolor{dialinecolor}{rgb}{0.000000, 0.000000, 0.000000}
\pgfsetfillcolor{dialinecolor}
% was here!!!
{\pgfsetcornersarced{\pgfpoint{0.000000\du}{0.000000\du}}\definecolor{dialinecolor}{rgb}{0.000000, 0.000000, 0.000000}
\pgfsetstrokecolor{dialinecolor}
\draw (24.000000\du,34.000000\du)--(24.000000\du,25.000000\du)--(34.600000\du,25.000000\du)--(34.600000\du,18.000000\du);
}}
\pgfsetlinewidth{0.000000\du}
\pgfsetdash{}{0pt}
\pgfsetdash{}{0pt}
\pgfsetmiterjoin
\pgfsetbuttcap
{
\definecolor{dialinecolor}{rgb}{0.000000, 0.000000, 0.000000}
\pgfsetfillcolor{dialinecolor}
% was here!!!
{\pgfsetcornersarced{\pgfpoint{0.000000\du}{0.000000\du}}\definecolor{dialinecolor}{rgb}{0.000000, 0.000000, 0.000000}
\pgfsetstrokecolor{dialinecolor}
\draw (25.000000\du,34.000000\du)--(25.000000\du,25.500000\du)--(35.000000\du,25.500000\du)--(35.000000\du,18.000000\du);
}}
\pgfsetlinewidth{0.000000\du}
\pgfsetdash{}{0pt}
\pgfsetdash{}{0pt}
\pgfsetmiterjoin
\pgfsetbuttcap
{
\definecolor{dialinecolor}{rgb}{0.000000, 0.000000, 0.000000}
\pgfsetfillcolor{dialinecolor}
% was here!!!
{\pgfsetcornersarced{\pgfpoint{0.000000\du}{0.000000\du}}\definecolor{dialinecolor}{rgb}{0.000000, 0.000000, 0.000000}
\pgfsetstrokecolor{dialinecolor}
\draw (26.000000\du,34.000000\du)--(26.000000\du,26.000000\du)--(35.400000\du,26.000000\du)--(35.400000\du,18.000000\du);
}}
\pgfsetlinewidth{0.000000\du}
\pgfsetdash{}{0pt}
\pgfsetdash{}{0pt}
\pgfsetmiterjoin
\pgfsetbuttcap
{
\definecolor{dialinecolor}{rgb}{0.000000, 0.000000, 0.000000}
\pgfsetfillcolor{dialinecolor}
% was here!!!
{\pgfsetcornersarced{\pgfpoint{0.000000\du}{0.000000\du}}\definecolor{dialinecolor}{rgb}{0.000000, 0.000000, 0.000000}
\pgfsetstrokecolor{dialinecolor}
\draw (27.000000\du,34.000000\du)--(27.000000\du,26.500000\du)--(35.800000\du,26.500000\du)--(35.800000\du,18.000000\du);
}}
\pgfsetlinewidth{0.000000\du}
\pgfsetdash{}{0pt}
\pgfsetdash{}{0pt}
\pgfsetbuttcap
{
\definecolor{dialinecolor}{rgb}{0.000000, 0.000000, 0.000000}
\pgfsetfillcolor{dialinecolor}
% was here!!!
\definecolor{dialinecolor}{rgb}{0.000000, 0.000000, 0.000000}
\pgfsetstrokecolor{dialinecolor}
\draw (17.000000\du,29.000000\du)--(22.000000\du,29.000000\du);
}
\pgfsetlinewidth{0.000000\du}
\pgfsetdash{}{0pt}
\pgfsetdash{}{0pt}
\pgfsetbuttcap
{
\definecolor{dialinecolor}{rgb}{0.000000, 0.000000, 0.000000}
\pgfsetfillcolor{dialinecolor}
% was here!!!
\definecolor{dialinecolor}{rgb}{0.000000, 0.000000, 0.000000}
\pgfsetstrokecolor{dialinecolor}
\draw (17.000000\du,27.000000\du)--(22.000000\du,27.000000\du);
}
\pgfsetlinewidth{0.000000\du}
\pgfsetdash{}{0pt}
\pgfsetdash{}{0pt}
\pgfsetmiterjoin
\pgfsetbuttcap
{
\definecolor{dialinecolor}{rgb}{0.000000, 0.000000, 0.000000}
\pgfsetfillcolor{dialinecolor}
% was here!!!
\pgfsetarrowsend{latex}
{\pgfsetcornersarced{\pgfpoint{0.000000\du}{0.000000\du}}\definecolor{dialinecolor}{rgb}{0.000000, 0.000000, 0.000000}
\pgfsetstrokecolor{dialinecolor}
\draw (18.000000\du,27.000000\du)--(18.000000\du,28.200000\du)--(18.000000\du,28.200000\du)--(18.000000\du,29.000000\du);
}}
\pgfsetlinewidth{0.000000\du}
\pgfsetdash{}{0pt}
\pgfsetdash{}{0pt}
\pgfsetmiterjoin
\pgfsetbuttcap
{
\definecolor{dialinecolor}{rgb}{0.000000, 0.000000, 0.000000}
\pgfsetfillcolor{dialinecolor}
% was here!!!
\pgfsetarrowsend{latex}
{\pgfsetcornersarced{\pgfpoint{0.000000\du}{0.000000\du}}\definecolor{dialinecolor}{rgb}{0.000000, 0.000000, 0.000000}
\pgfsetstrokecolor{dialinecolor}
\draw (19.000000\du,27.000000\du)--(19.000000\du,27.000000\du)--(19.000000\du,29.000000\du)--(19.000000\du,29.000000\du);
}}
\pgfsetlinewidth{0.000000\du}
\pgfsetdash{}{0pt}
\pgfsetdash{}{0pt}
\pgfsetmiterjoin
\pgfsetbuttcap
{
\definecolor{dialinecolor}{rgb}{0.000000, 0.000000, 0.000000}
\pgfsetfillcolor{dialinecolor}
% was here!!!
\pgfsetarrowsend{latex}
{\pgfsetcornersarced{\pgfpoint{0.000000\du}{0.000000\du}}\definecolor{dialinecolor}{rgb}{0.000000, 0.000000, 0.000000}
\pgfsetstrokecolor{dialinecolor}
\draw (20.000000\du,27.000000\du)--(20.000000\du,27.800000\du)--(20.000000\du,27.800000\du)--(20.000000\du,29.000000\du);
}}
\pgfsetlinewidth{0.000000\du}
\pgfsetdash{}{0pt}
\pgfsetdash{}{0pt}
\pgfsetmiterjoin
\pgfsetbuttcap
{
\definecolor{dialinecolor}{rgb}{0.000000, 0.000000, 0.000000}
\pgfsetfillcolor{dialinecolor}
% was here!!!
\pgfsetarrowsend{latex}
{\pgfsetcornersarced{\pgfpoint{0.000000\du}{0.000000\du}}\definecolor{dialinecolor}{rgb}{0.000000, 0.000000, 0.000000}
\pgfsetstrokecolor{dialinecolor}
\draw (21.000000\du,27.000000\du)--(21.000000\du,27.600000\du)--(21.000000\du,27.600000\du)--(21.000000\du,29.000000\du);
}}
\pgfsetlinewidth{0.000000\du}
\pgfsetdash{}{0pt}
\pgfsetdash{}{0pt}
\pgfsetbuttcap
{
\definecolor{dialinecolor}{rgb}{0.000000, 0.000000, 0.000000}
\pgfsetfillcolor{dialinecolor}
% was here!!!
\definecolor{dialinecolor}{rgb}{0.000000, 0.000000, 0.000000}
\pgfsetstrokecolor{dialinecolor}
\draw (17.000000\du,32.000000\du)--(22.000000\du,32.000000\du);
}
\pgfsetlinewidth{0.000000\du}
\pgfsetdash{}{0pt}
\pgfsetdash{}{0pt}
\pgfsetbuttcap
{
\definecolor{dialinecolor}{rgb}{0.000000, 0.000000, 0.000000}
\pgfsetfillcolor{dialinecolor}
% was here!!!
\definecolor{dialinecolor}{rgb}{0.000000, 0.000000, 0.000000}
\pgfsetstrokecolor{dialinecolor}
\draw (17.000000\du,34.000000\du)--(22.000000\du,34.000000\du);
}
\pgfsetlinewidth{0.000000\du}
\pgfsetdash{}{0pt}
\pgfsetdash{}{0pt}
\pgfsetmiterjoin
\pgfsetbuttcap
{
\definecolor{dialinecolor}{rgb}{0.000000, 0.000000, 0.000000}
\pgfsetfillcolor{dialinecolor}
% was here!!!
\pgfsetarrowsend{latex}
{\pgfsetcornersarced{\pgfpoint{0.000000\du}{0.000000\du}}\definecolor{dialinecolor}{rgb}{0.000000, 0.000000, 0.000000}
\pgfsetstrokecolor{dialinecolor}
\draw (18.000000\du,32.000000\du)--(18.000000\du,32.000000\du)--(18.000000\du,34.000000\du)--(18.000000\du,34.000000\du);
}}
\pgfsetlinewidth{0.000000\du}
\pgfsetdash{}{0pt}
\pgfsetdash{}{0pt}
\pgfsetmiterjoin
\pgfsetbuttcap
{
\definecolor{dialinecolor}{rgb}{0.000000, 0.000000, 0.000000}
\pgfsetfillcolor{dialinecolor}
% was here!!!
\pgfsetarrowsend{latex}
{\pgfsetcornersarced{\pgfpoint{0.000000\du}{0.000000\du}}\definecolor{dialinecolor}{rgb}{0.000000, 0.000000, 0.000000}
\pgfsetstrokecolor{dialinecolor}
\draw (19.000000\du,32.000000\du)--(19.000000\du,32.000000\du)--(19.000000\du,34.000000\du)--(19.000000\du,34.000000\du);
}}
\pgfsetlinewidth{0.000000\du}
\pgfsetdash{}{0pt}
\pgfsetdash{}{0pt}
\pgfsetmiterjoin
\pgfsetbuttcap
{
\definecolor{dialinecolor}{rgb}{0.000000, 0.000000, 0.000000}
\pgfsetfillcolor{dialinecolor}
% was here!!!
\pgfsetarrowsend{latex}
{\pgfsetcornersarced{\pgfpoint{0.000000\du}{0.000000\du}}\definecolor{dialinecolor}{rgb}{0.000000, 0.000000, 0.000000}
\pgfsetstrokecolor{dialinecolor}
\draw (20.000000\du,32.000000\du)--(20.000000\du,32.000000\du)--(20.000000\du,34.000000\du)--(20.000000\du,34.000000\du);
}}
\pgfsetlinewidth{0.000000\du}
\pgfsetdash{}{0pt}
\pgfsetdash{}{0pt}
\pgfsetmiterjoin
\pgfsetbuttcap
{
\definecolor{dialinecolor}{rgb}{0.000000, 0.000000, 0.000000}
\pgfsetfillcolor{dialinecolor}
% was here!!!
\pgfsetarrowsend{latex}
{\pgfsetcornersarced{\pgfpoint{0.000000\du}{0.000000\du}}\definecolor{dialinecolor}{rgb}{0.000000, 0.000000, 0.000000}
\pgfsetstrokecolor{dialinecolor}
\draw (21.000000\du,32.000000\du)--(21.000000\du,32.000000\du)--(21.000000\du,34.000000\du)--(21.000000\du,34.000000\du);
}}
\pgfsetlinewidth{0.000000\du}
\pgfsetdash{}{0pt}
\pgfsetdash{}{0pt}
\pgfsetmiterjoin
\definecolor{dialinecolor}{rgb}{1.000000, 1.000000, 1.000000}
\pgfsetfillcolor{dialinecolor}
\fill (31.000000\du,34.000000\du)--(31.000000\du,37.000000\du)--(36.000000\du,37.000000\du)--(36.000000\du,34.000000\du)--cycle;
\definecolor{dialinecolor}{rgb}{0.000000, 0.000000, 0.000000}
\pgfsetstrokecolor{dialinecolor}
\draw (31.000000\du,34.000000\du)--(31.000000\du,37.000000\du)--(36.000000\du,37.000000\du)--(36.000000\du,34.000000\du)--cycle;
\pgfsetlinewidth{0.000000\du}
\pgfsetdash{}{0pt}
\pgfsetdash{}{0pt}
\pgfsetmiterjoin
\definecolor{dialinecolor}{rgb}{1.000000, 1.000000, 1.000000}
\pgfsetfillcolor{dialinecolor}
\fill (31.000000\du,29.000000\du)--(31.000000\du,32.000000\du)--(36.000000\du,32.000000\du)--(36.000000\du,29.000000\du)--cycle;
\definecolor{dialinecolor}{rgb}{0.000000, 0.000000, 0.000000}
\pgfsetstrokecolor{dialinecolor}
\draw (31.000000\du,29.000000\du)--(31.000000\du,32.000000\du)--(36.000000\du,32.000000\du)--(36.000000\du,29.000000\du)--cycle;
% setfont left to latex
\definecolor{dialinecolor}{rgb}{0.000000, 0.000000, 0.000000}
\pgfsetstrokecolor{dialinecolor}
\node at (33.500000\du,30.721250\du){FIFO};
\pgfsetlinewidth{0.000000\du}
\pgfsetdash{}{0pt}
\pgfsetdash{}{0pt}
\pgfsetbuttcap
{
\definecolor{dialinecolor}{rgb}{0.000000, 0.000000, 0.000000}
\pgfsetfillcolor{dialinecolor}
% was here!!!
\definecolor{dialinecolor}{rgb}{0.000000, 0.000000, 0.000000}
\pgfsetstrokecolor{dialinecolor}
\draw (31.000000\du,29.000000\du)--(36.000000\du,29.000000\du);
}
\pgfsetlinewidth{0.000000\du}
\pgfsetdash{}{0pt}
\pgfsetdash{}{0pt}
\pgfsetbuttcap
{
\definecolor{dialinecolor}{rgb}{0.000000, 0.000000, 0.000000}
\pgfsetfillcolor{dialinecolor}
% was here!!!
\definecolor{dialinecolor}{rgb}{0.000000, 0.000000, 0.000000}
\pgfsetstrokecolor{dialinecolor}
\draw (31.000000\du,32.000000\du)--(36.000000\du,32.000000\du);
}
\pgfsetlinewidth{0.000000\du}
\pgfsetdash{}{0pt}
\pgfsetdash{}{0pt}
\pgfsetbuttcap
{
\definecolor{dialinecolor}{rgb}{0.000000, 0.000000, 0.000000}
\pgfsetfillcolor{dialinecolor}
% was here!!!
\definecolor{dialinecolor}{rgb}{0.000000, 0.000000, 0.000000}
\pgfsetstrokecolor{dialinecolor}
\draw (31.000000\du,34.000000\du)--(36.000000\du,34.000000\du);
}
\pgfsetlinewidth{0.000000\du}
\pgfsetdash{}{0pt}
\pgfsetdash{}{0pt}
\pgfsetmiterjoin
\pgfsetbuttcap
{
\definecolor{dialinecolor}{rgb}{0.000000, 0.000000, 0.000000}
\pgfsetfillcolor{dialinecolor}
% was here!!!
\pgfsetarrowsend{latex}
{\pgfsetcornersarced{\pgfpoint{0.000000\du}{0.000000\du}}\definecolor{dialinecolor}{rgb}{0.000000, 0.000000, 0.000000}
\pgfsetstrokecolor{dialinecolor}
\draw (32.000000\du,32.000000\du)--(32.000000\du,32.000000\du)--(32.000000\du,34.000000\du)--(32.000000\du,34.000000\du);
}}
\pgfsetlinewidth{0.000000\du}
\pgfsetdash{}{0pt}
\pgfsetdash{}{0pt}
\pgfsetmiterjoin
\pgfsetbuttcap
{
\definecolor{dialinecolor}{rgb}{0.000000, 0.000000, 0.000000}
\pgfsetfillcolor{dialinecolor}
% was here!!!
\pgfsetarrowsend{latex}
{\pgfsetcornersarced{\pgfpoint{0.000000\du}{0.000000\du}}\definecolor{dialinecolor}{rgb}{0.000000, 0.000000, 0.000000}
\pgfsetstrokecolor{dialinecolor}
\draw (33.000000\du,32.000000\du)--(33.000000\du,32.000000\du)--(33.000000\du,34.000000\du)--(33.000000\du,34.000000\du);
}}
\pgfsetlinewidth{0.000000\du}
\pgfsetdash{}{0pt}
\pgfsetdash{}{0pt}
\pgfsetmiterjoin
\pgfsetbuttcap
{
\definecolor{dialinecolor}{rgb}{0.000000, 0.000000, 0.000000}
\pgfsetfillcolor{dialinecolor}
% was here!!!
\pgfsetarrowsend{latex}
{\pgfsetcornersarced{\pgfpoint{0.000000\du}{0.000000\du}}\definecolor{dialinecolor}{rgb}{0.000000, 0.000000, 0.000000}
\pgfsetstrokecolor{dialinecolor}
\draw (34.000000\du,32.000000\du)--(34.000000\du,32.000000\du)--(34.000000\du,34.000000\du)--(34.000000\du,34.000000\du);
}}
\pgfsetlinewidth{0.000000\du}
\pgfsetdash{}{0pt}
\pgfsetdash{}{0pt}
\pgfsetmiterjoin
\pgfsetbuttcap
{
\definecolor{dialinecolor}{rgb}{0.000000, 0.000000, 0.000000}
\pgfsetfillcolor{dialinecolor}
% was here!!!
\pgfsetarrowsend{latex}
{\pgfsetcornersarced{\pgfpoint{0.000000\du}{0.000000\du}}\definecolor{dialinecolor}{rgb}{0.000000, 0.000000, 0.000000}
\pgfsetstrokecolor{dialinecolor}
\draw (35.000000\du,32.000000\du)--(35.000000\du,32.000000\du)--(35.000000\du,34.000000\du)--(35.000000\du,34.000000\du);
}}
% setfont left to latex
\definecolor{dialinecolor}{rgb}{0.000000, 0.000000, 0.000000}
\pgfsetstrokecolor{dialinecolor}
\node at (33.500000\du,35.721250\du){TLT};
\pgfsetlinewidth{0.000000\du}
\pgfsetdash{}{0pt}
\pgfsetdash{}{0pt}
\pgfsetmiterjoin
\pgfsetbuttcap
{
\definecolor{dialinecolor}{rgb}{0.000000, 0.000000, 0.000000}
\pgfsetfillcolor{dialinecolor}
% was here!!!
\pgfsetarrowsend{latex}
{\pgfsetcornersarced{\pgfpoint{0.000000\du}{0.000000\du}}\definecolor{dialinecolor}{rgb}{0.000000, 0.000000, 0.000000}
\pgfsetstrokecolor{dialinecolor}
\draw (18.000000\du,27.000000\du)--(18.000000\du,28.200000\du)--(32.000000\du,28.200000\du)--(32.000000\du,29.000000\du);
}}
\pgfsetlinewidth{0.000000\du}
\pgfsetdash{}{0pt}
\pgfsetdash{}{0pt}
\pgfsetmiterjoin
\pgfsetbuttcap
{
\definecolor{dialinecolor}{rgb}{0.000000, 0.000000, 0.000000}
\pgfsetfillcolor{dialinecolor}
% was here!!!
\pgfsetarrowsend{latex}
{\pgfsetcornersarced{\pgfpoint{0.000000\du}{0.000000\du}}\definecolor{dialinecolor}{rgb}{0.000000, 0.000000, 0.000000}
\pgfsetstrokecolor{dialinecolor}
\draw (19.000000\du,27.000000\du)--(19.000000\du,28.000000\du)--(33.000000\du,28.000000\du)--(33.000000\du,29.000000\du);
}}
\pgfsetlinewidth{0.000000\du}
\pgfsetdash{}{0pt}
\pgfsetdash{}{0pt}
\pgfsetmiterjoin
\pgfsetbuttcap
{
\definecolor{dialinecolor}{rgb}{0.000000, 0.000000, 0.000000}
\pgfsetfillcolor{dialinecolor}
% was here!!!
\pgfsetarrowsend{latex}
{\pgfsetcornersarced{\pgfpoint{0.000000\du}{0.000000\du}}\definecolor{dialinecolor}{rgb}{0.000000, 0.000000, 0.000000}
\pgfsetstrokecolor{dialinecolor}
\draw (20.000000\du,27.000000\du)--(20.000000\du,27.800000\du)--(34.000000\du,27.800000\du)--(34.000000\du,29.000000\du);
}}
\pgfsetlinewidth{0.000000\du}
\pgfsetdash{}{0pt}
\pgfsetdash{}{0pt}
\pgfsetmiterjoin
\pgfsetbuttcap
{
\definecolor{dialinecolor}{rgb}{0.000000, 0.000000, 0.000000}
\pgfsetfillcolor{dialinecolor}
% was here!!!
\pgfsetarrowsend{latex}
{\pgfsetcornersarced{\pgfpoint{0.000000\du}{0.000000\du}}\definecolor{dialinecolor}{rgb}{0.000000, 0.000000, 0.000000}
\pgfsetstrokecolor{dialinecolor}
\draw (21.000000\du,27.000000\du)--(21.000000\du,27.600000\du)--(35.000000\du,27.600000\du)--(35.000000\du,29.000000\du);
}}
\pgfsetlinewidth{0.000000\du}
\pgfsetdash{}{0pt}
\pgfsetdash{}{0pt}
\pgfsetbuttcap
\pgfsetmiterjoin
\pgfsetlinewidth{0.000000\du}
\pgfsetbuttcap
\pgfsetmiterjoin
\pgfsetdash{}{0pt}
\pgfsetbuttcap
\pgfsetmiterjoin
\pgfsetdash{}{0pt}
\definecolor{dialinecolor}{rgb}{0.000000, 0.000000, 0.000000}
\pgfsetfillcolor{dialinecolor}
\pgfpathellipse{\pgfpoint{21.000000\du}{27.600000\du}}{\pgfpoint{0.050000\du}{0\du}}{\pgfpoint{0\du}{0.050000\du}}
\pgfusepath{fill}
\definecolor{dialinecolor}{rgb}{0.000000, 0.000000, 0.000000}
\pgfsetstrokecolor{dialinecolor}
\pgfpathellipse{\pgfpoint{21.000000\du}{27.600000\du}}{\pgfpoint{0.050000\du}{0\du}}{\pgfpoint{0\du}{0.050000\du}}
\pgfusepath{stroke}
\pgfsetlinewidth{0.000000\du}
\pgfsetdash{}{0pt}
\pgfsetdash{}{0pt}
\pgfsetbuttcap
\pgfsetmiterjoin
\pgfsetlinewidth{0.000000\du}
\pgfsetbuttcap
\pgfsetmiterjoin
\pgfsetdash{}{0pt}
\pgfsetbuttcap
\pgfsetmiterjoin
\pgfsetdash{}{0pt}
\definecolor{dialinecolor}{rgb}{0.000000, 0.000000, 0.000000}
\pgfsetfillcolor{dialinecolor}
\pgfpathellipse{\pgfpoint{20.000000\du}{27.800000\du}}{\pgfpoint{0.050000\du}{0\du}}{\pgfpoint{0\du}{0.050000\du}}
\pgfusepath{fill}
\definecolor{dialinecolor}{rgb}{0.000000, 0.000000, 0.000000}
\pgfsetstrokecolor{dialinecolor}
\pgfpathellipse{\pgfpoint{20.000000\du}{27.800000\du}}{\pgfpoint{0.050000\du}{0\du}}{\pgfpoint{0\du}{0.050000\du}}
\pgfusepath{stroke}
\pgfsetlinewidth{0.000000\du}
\pgfsetdash{}{0pt}
\pgfsetdash{}{0pt}
\pgfsetbuttcap
\pgfsetmiterjoin
\pgfsetlinewidth{0.000000\du}
\pgfsetbuttcap
\pgfsetmiterjoin
\pgfsetdash{}{0pt}
\pgfsetbuttcap
\pgfsetmiterjoin
\pgfsetdash{}{0pt}
\definecolor{dialinecolor}{rgb}{0.000000, 0.000000, 0.000000}
\pgfsetfillcolor{dialinecolor}
\pgfpathellipse{\pgfpoint{19.000000\du}{28.000000\du}}{\pgfpoint{0.050000\du}{0\du}}{\pgfpoint{0\du}{0.050000\du}}
\pgfusepath{fill}
\definecolor{dialinecolor}{rgb}{0.000000, 0.000000, 0.000000}
\pgfsetstrokecolor{dialinecolor}
\pgfpathellipse{\pgfpoint{19.000000\du}{28.000000\du}}{\pgfpoint{0.050000\du}{0\du}}{\pgfpoint{0\du}{0.050000\du}}
\pgfusepath{stroke}
\pgfsetlinewidth{0.000000\du}
\pgfsetdash{}{0pt}
\pgfsetdash{}{0pt}
\pgfsetbuttcap
\pgfsetmiterjoin
\pgfsetlinewidth{0.000000\du}
\pgfsetbuttcap
\pgfsetmiterjoin
\pgfsetdash{}{0pt}
\pgfsetbuttcap
\pgfsetmiterjoin
\pgfsetdash{}{0pt}
\definecolor{dialinecolor}{rgb}{0.000000, 0.000000, 0.000000}
\pgfsetfillcolor{dialinecolor}
\pgfpathellipse{\pgfpoint{18.000000\du}{28.200000\du}}{\pgfpoint{0.050000\du}{0\du}}{\pgfpoint{0\du}{0.050000\du}}
\pgfusepath{fill}
\definecolor{dialinecolor}{rgb}{0.000000, 0.000000, 0.000000}
\pgfsetstrokecolor{dialinecolor}
\pgfpathellipse{\pgfpoint{18.000000\du}{28.200000\du}}{\pgfpoint{0.050000\du}{0\du}}{\pgfpoint{0\du}{0.050000\du}}
\pgfusepath{stroke}
\pgfsetlinewidth{0.000000\du}
\pgfsetdash{}{0pt}
\pgfsetdash{}{0pt}
\pgfsetbuttcap
\pgfsetmiterjoin
\pgfsetlinewidth{0.000000\du}
\pgfsetbuttcap
\pgfsetmiterjoin
\pgfsetdash{}{0pt}
\definecolor{dialinecolor}{rgb}{0.000000, 0.000000, 0.000000}
\pgfsetstrokecolor{dialinecolor}
\draw (34.000000\du,18.000000\du)--(36.400000\du,18.000000\du)--(35.800000\du,17.000000\du)--(34.600000\du,17.000000\du)--cycle;
% setfont left to latex
\definecolor{dialinecolor}{rgb}{0.000000, 0.000000, 0.000000}
\pgfsetstrokecolor{dialinecolor}
\node at (35.200000\du,17.700000\du){};
\pgfsetlinewidth{0.000000\du}
\pgfsetdash{}{0pt}
\pgfsetdash{}{0pt}
\pgfsetbuttcap
{
\definecolor{dialinecolor}{rgb}{0.000000, 0.000000, 0.000000}
\pgfsetfillcolor{dialinecolor}
% was here!!!
\definecolor{dialinecolor}{rgb}{0.000000, 0.000000, 0.000000}
\pgfsetstrokecolor{dialinecolor}
\draw (34.600000\du,18.000000\du)--(35.800000\du,18.000000\du);
}
\pgfsetlinewidth{0.000000\du}
\pgfsetdash{}{0pt}
\pgfsetdash{}{0pt}
\pgfsetmiterjoin
\pgfsetbuttcap
{
\definecolor{dialinecolor}{rgb}{0.000000, 0.000000, 0.000000}
\pgfsetfillcolor{dialinecolor}
% was here!!!
\pgfsetarrowsend{latex}
{\pgfsetcornersarced{\pgfpoint{0.000000\du}{0.000000\du}}\definecolor{dialinecolor}{rgb}{0.000000, 0.000000, 0.000000}
\pgfsetstrokecolor{dialinecolor}
\draw (35.200000\du,17.000000\du)--(35.200000\du,14.500000\du)--(34.000000\du,14.500000\du);
}}
% setfont left to latex
\definecolor{dialinecolor}{rgb}{0.000000, 0.000000, 0.000000}
\pgfsetstrokecolor{dialinecolor}
\node[anchor=west] at (15.500000\du,33.000000\du){AXI x4};
% setfont left to latex
\definecolor{dialinecolor}{rgb}{0.000000, 0.000000, 0.000000}
\pgfsetstrokecolor{dialinecolor}
\node[anchor=west] at (29.500000\du,33.000000\du){AXI x4};
\pgfsetlinewidth{0.000000\du}
\pgfsetdash{}{0pt}
\pgfsetdash{}{0pt}
\pgfsetbuttcap
\pgfsetmiterjoin
\pgfsetlinewidth{0.000000\du}
\pgfsetbuttcap
\pgfsetmiterjoin
\pgfsetdash{}{0pt}
\definecolor{dialinecolor}{rgb}{0.000000, 0.000000, 0.000000}
\pgfsetstrokecolor{dialinecolor}
\pgfpathmoveto{\pgfpoint{13.620000\du}{36.000000\du}}
\pgfpathcurveto{\pgfpoint{13.178172\du}{36.000000\du}}{\pgfpoint{12.820000\du}{35.552285\du}}{\pgfpoint{12.820000\du}{35.000000\du}}
\pgfpathcurveto{\pgfpoint{12.820000\du}{34.447715\du}}{\pgfpoint{13.178172\du}{34.000000\du}}{\pgfpoint{13.620000\du}{34.000000\du}}
\pgfusepath{stroke}
\pgfsetbuttcap
\pgfsetmiterjoin
\pgfsetdash{}{0pt}
\definecolor{dialinecolor}{rgb}{0.000000, 0.000000, 0.000000}
\pgfsetstrokecolor{dialinecolor}
\draw (14.500000\du,34.000000\du)--(14.500000\du,36.000000\du);
\pgfsetbuttcap
\pgfsetmiterjoin
\pgfsetdash{}{0pt}
\definecolor{dialinecolor}{rgb}{0.000000, 0.000000, 0.000000}
\pgfsetstrokecolor{dialinecolor}
\draw (14.500000\du,34.000000\du)--(13.620000\du,34.000000\du);
\pgfsetbuttcap
\pgfsetmiterjoin
\pgfsetdash{}{0pt}
\definecolor{dialinecolor}{rgb}{0.000000, 0.000000, 0.000000}
\pgfsetstrokecolor{dialinecolor}
\draw (14.500000\du,36.000000\du)--(13.620000\du,36.000000\du);
\pgfsetbuttcap
\pgfsetmiterjoin
\pgfsetdash{}{0pt}
\definecolor{dialinecolor}{rgb}{0.000000, 0.000000, 0.000000}
\pgfsetstrokecolor{dialinecolor}
\draw (12.820000\du,35.000000\du)--(12.500000\du,35.000000\du);
\pgfsetlinewidth{0.000000\du}
\pgfsetdash{}{0pt}
\pgfsetdash{}{0pt}
\pgfsetmiterjoin
\pgfsetbuttcap
{
\definecolor{dialinecolor}{rgb}{0.000000, 0.000000, 0.000000}
\pgfsetfillcolor{dialinecolor}
% was here!!!
{\pgfsetcornersarced{\pgfpoint{0.000000\du}{0.000000\du}}\definecolor{dialinecolor}{rgb}{0.000000, 0.000000, 0.000000}
\pgfsetstrokecolor{dialinecolor}
\draw (17.000000\du,30.500000\du)--(15.000000\du,30.500000\du)--(15.000000\du,34.500000\du)--(14.500000\du,34.500000\du);
}}
\pgfsetlinewidth{0.000000\du}
\pgfsetdash{}{0pt}
\pgfsetdash{}{0pt}
\pgfsetmiterjoin
\pgfsetbuttcap
{
\definecolor{dialinecolor}{rgb}{0.000000, 0.000000, 0.000000}
\pgfsetfillcolor{dialinecolor}
% was here!!!
{\pgfsetcornersarced{\pgfpoint{0.000000\du}{0.000000\du}}\definecolor{dialinecolor}{rgb}{0.000000, 0.000000, 0.000000}
\pgfsetstrokecolor{dialinecolor}
\draw (31.000000\du,30.500000\du)--(29.000000\du,30.500000\du)--(29.000000\du,40.000000\du)--(15.000000\du,40.000000\du)--(15.000000\du,35.500000\du)--(14.500000\du,35.500000\du);
}}
\pgfsetlinewidth{0.000000\du}
\pgfsetdash{}{0pt}
\pgfsetdash{}{0pt}
\pgfsetmiterjoin
\pgfsetbuttcap
{
\definecolor{dialinecolor}{rgb}{0.000000, 0.000000, 0.000000}
\pgfsetfillcolor{dialinecolor}
% was here!!!
{\pgfsetcornersarced{\pgfpoint{0.000000\du}{0.000000\du}}\definecolor{dialinecolor}{rgb}{0.000000, 0.000000, 0.000000}
\pgfsetstrokecolor{dialinecolor}
\draw (12.500000\du,35.000000\du)--(12.000000\du,35.000000\du)--(12.000000\du,33.000000\du)--(12.000000\du,33.000000\du);
}}
% setfont left to latex
\definecolor{dialinecolor}{rgb}{0.000000, 0.000000, 0.000000}
\pgfsetstrokecolor{dialinecolor}
\node at (12.000000\du,30.247500\du){Global};
% setfont left to latex
\definecolor{dialinecolor}{rgb}{0.000000, 0.000000, 0.000000}
\pgfsetstrokecolor{dialinecolor}
\node at (12.000000\du,31.047500\du){Clock};
% setfont left to latex
\definecolor{dialinecolor}{rgb}{0.000000, 0.000000, 0.000000}
\pgfsetstrokecolor{dialinecolor}
\node at (12.000000\du,31.847500\du){Enable};
% setfont left to latex
\definecolor{dialinecolor}{rgb}{0.000000, 0.000000, 0.000000}
\pgfsetstrokecolor{dialinecolor}
\node[anchor=east] at (17.000000\du,29.847500\du){empty };
% setfont left to latex
\definecolor{dialinecolor}{rgb}{0.000000, 0.000000, 0.000000}
\pgfsetstrokecolor{dialinecolor}
\node[anchor=east] at (31.000000\du,29.847500\du){empty };
\pgfsetlinewidth{0.010000\du}
\pgfsetdash{}{0pt}
\pgfsetdash{}{0pt}
\pgfsetmiterjoin
\pgfsetbuttcap
{
\definecolor{dialinecolor}{rgb}{0.000000, 0.000000, 0.000000}
\pgfsetfillcolor{dialinecolor}
% was here!!!
\pgfsetarrowsend{latex}
{\pgfsetcornersarced{\pgfpoint{0.000000\du}{0.000000\du}}\definecolor{dialinecolor}{rgb}{0.000000, 0.000000, 0.000000}
\pgfsetstrokecolor{dialinecolor}
\draw (26.500000\du,20.000000\du)--(26.500000\du,20.500000\du)--(37.500000\du,20.500000\du)--(37.500000\du,20.500000\du);
}}
\pgfsetlinewidth{0.010000\du}
\pgfsetdash{}{0pt}
\pgfsetdash{}{0pt}
\pgfsetmiterjoin
\pgfsetbuttcap
{
\definecolor{dialinecolor}{rgb}{0.000000, 0.000000, 0.000000}
\pgfsetfillcolor{dialinecolor}
% was here!!!
\pgfsetarrowsend{latex}
{\pgfsetcornersarced{\pgfpoint{0.000000\du}{0.000000\du}}\definecolor{dialinecolor}{rgb}{0.000000, 0.000000, 0.000000}
\pgfsetstrokecolor{dialinecolor}
\draw (20.500000\du,20.000000\du)--(20.500000\du,21.000000\du)--(37.500000\du,21.000000\du)--(37.500000\du,21.000000\du);
}}
\pgfsetlinewidth{0.010000\du}
\pgfsetdash{}{0pt}
\pgfsetdash{}{0pt}
\pgfsetmiterjoin
\pgfsetbuttcap
{
\definecolor{dialinecolor}{rgb}{0.000000, 0.000000, 0.000000}
\pgfsetfillcolor{dialinecolor}
% was here!!!
\pgfsetarrowsend{latex}
{\pgfsetcornersarced{\pgfpoint{0.000000\du}{0.000000\du}}\definecolor{dialinecolor}{rgb}{0.000000, 0.000000, 0.000000}
\pgfsetstrokecolor{dialinecolor}
\draw (17.500000\du,20.000000\du)--(17.500000\du,21.500000\du)--(37.500000\du,21.500000\du)--(37.500000\du,21.500000\du);
}}
\pgfsetlinewidth{0.010000\du}
\pgfsetdash{}{0pt}
\pgfsetdash{}{0pt}
\pgfsetmiterjoin
\pgfsetbuttcap
{
\definecolor{dialinecolor}{rgb}{0.000000, 0.000000, 0.000000}
\pgfsetfillcolor{dialinecolor}
% was here!!!
\pgfsetarrowsend{latex}
{\pgfsetcornersarced{\pgfpoint{0.000000\du}{0.000000\du}}\definecolor{dialinecolor}{rgb}{0.000000, 0.000000, 0.000000}
\pgfsetstrokecolor{dialinecolor}
\draw (13.000000\du,20.000000\du)--(13.000000\du,22.000000\du)--(37.500000\du,22.000000\du)--(37.500000\du,22.000000\du);
}}
% setfont left to latex
\definecolor{dialinecolor}{rgb}{0.000000, 0.000000, 0.000000}
\pgfsetstrokecolor{dialinecolor}
\node[anchor=west] at (36.000000\du,23.321250\du){ To Port 1+};
% setfont left to latex
\definecolor{dialinecolor}{rgb}{0.000000, 0.000000, 0.000000}
\pgfsetstrokecolor{dialinecolor}
\node[anchor=west] at (36.000000\du,24.121250\du){    Checking};
\pgfsetlinewidth{0.000000\du}
\pgfsetdash{}{0pt}
\pgfsetdash{}{0pt}
\pgfsetbuttcap
\pgfsetmiterjoin
\pgfsetlinewidth{0.000000\du}
\pgfsetbuttcap
\pgfsetmiterjoin
\pgfsetdash{}{0pt}
\pgfsetbuttcap
\pgfsetmiterjoin
\pgfsetdash{}{0pt}
\definecolor{dialinecolor}{rgb}{0.000000, 0.000000, 0.000000}
\pgfsetfillcolor{dialinecolor}
\pgfpathellipse{\pgfpoint{31.400000\du}{20.500000\du}}{\pgfpoint{0.125000\du}{0\du}}{\pgfpoint{0\du}{0.125000\du}}
\pgfusepath{fill}
\definecolor{dialinecolor}{rgb}{0.000000, 0.000000, 0.000000}
\pgfsetstrokecolor{dialinecolor}
\pgfpathellipse{\pgfpoint{31.400000\du}{20.500000\du}}{\pgfpoint{0.125000\du}{0\du}}{\pgfpoint{0\du}{0.125000\du}}
\pgfusepath{stroke}
\pgfsetlinewidth{0.000000\du}
\pgfsetdash{}{0pt}
\pgfsetdash{}{0pt}
\pgfsetbuttcap
\pgfsetmiterjoin
\pgfsetlinewidth{0.000000\du}
\pgfsetbuttcap
\pgfsetmiterjoin
\pgfsetdash{}{0pt}
\pgfsetbuttcap
\pgfsetmiterjoin
\pgfsetdash{}{0pt}
\definecolor{dialinecolor}{rgb}{0.000000, 0.000000, 0.000000}
\pgfsetfillcolor{dialinecolor}
\pgfpathellipse{\pgfpoint{31.800000\du}{21.000000\du}}{\pgfpoint{0.125000\du}{0\du}}{\pgfpoint{0\du}{0.125000\du}}
\pgfusepath{fill}
\definecolor{dialinecolor}{rgb}{0.000000, 0.000000, 0.000000}
\pgfsetstrokecolor{dialinecolor}
\pgfpathellipse{\pgfpoint{31.800000\du}{21.000000\du}}{\pgfpoint{0.125000\du}{0\du}}{\pgfpoint{0\du}{0.125000\du}}
\pgfusepath{stroke}
\pgfsetlinewidth{0.000000\du}
\pgfsetdash{}{0pt}
\pgfsetdash{}{0pt}
\pgfsetbuttcap
\pgfsetmiterjoin
\pgfsetlinewidth{0.000000\du}
\pgfsetbuttcap
\pgfsetmiterjoin
\pgfsetdash{}{0pt}
\pgfsetbuttcap
\pgfsetmiterjoin
\pgfsetdash{}{0pt}
\definecolor{dialinecolor}{rgb}{0.000000, 0.000000, 0.000000}
\pgfsetfillcolor{dialinecolor}
\pgfpathellipse{\pgfpoint{32.200000\du}{21.500000\du}}{\pgfpoint{0.125000\du}{0\du}}{\pgfpoint{0\du}{0.125000\du}}
\pgfusepath{fill}
\definecolor{dialinecolor}{rgb}{0.000000, 0.000000, 0.000000}
\pgfsetstrokecolor{dialinecolor}
\pgfpathellipse{\pgfpoint{32.200000\du}{21.500000\du}}{\pgfpoint{0.125000\du}{0\du}}{\pgfpoint{0\du}{0.125000\du}}
\pgfusepath{stroke}
\pgfsetlinewidth{0.000000\du}
\pgfsetdash{}{0pt}
\pgfsetdash{}{0pt}
\pgfsetbuttcap
\pgfsetmiterjoin
\pgfsetlinewidth{0.000000\du}
\pgfsetbuttcap
\pgfsetmiterjoin
\pgfsetdash{}{0pt}
\pgfsetbuttcap
\pgfsetmiterjoin
\pgfsetdash{}{0pt}
\definecolor{dialinecolor}{rgb}{0.000000, 0.000000, 0.000000}
\pgfsetfillcolor{dialinecolor}
\pgfpathellipse{\pgfpoint{32.600000\du}{22.000000\du}}{\pgfpoint{0.125000\du}{0\du}}{\pgfpoint{0\du}{0.125000\du}}
\pgfusepath{fill}
\definecolor{dialinecolor}{rgb}{0.000000, 0.000000, 0.000000}
\pgfsetstrokecolor{dialinecolor}
\pgfpathellipse{\pgfpoint{32.600000\du}{22.000000\du}}{\pgfpoint{0.125000\du}{0\du}}{\pgfpoint{0\du}{0.125000\du}}
\pgfusepath{stroke}
\pgfsetlinewidth{0.200000\du}
\pgfsetdash{}{0pt}
\pgfsetdash{}{0pt}
\pgfsetbuttcap
{
\definecolor{dialinecolor}{rgb}{0.000000, 0.000000, 0.000000}
\pgfsetfillcolor{dialinecolor}
% was here!!!
\pgfsetarrowsend{latex}
\definecolor{dialinecolor}{rgb}{0.000000, 0.000000, 0.000000}
\pgfsetstrokecolor{dialinecolor}
\draw (19.500000\du,37.000000\du)--(19.500000\du,41.500000\du);
}
\pgfsetlinewidth{0.200000\du}
\pgfsetdash{}{0pt}
\pgfsetdash{}{0pt}
\pgfsetbuttcap
{
\definecolor{dialinecolor}{rgb}{0.000000, 0.000000, 0.000000}
\pgfsetfillcolor{dialinecolor}
% was here!!!
\pgfsetarrowsend{latex}
\definecolor{dialinecolor}{rgb}{0.000000, 0.000000, 0.000000}
\pgfsetstrokecolor{dialinecolor}
\draw (25.500000\du,41.500000\du)--(25.500000\du,37.000000\du);
}
\pgfsetlinewidth{0.200000\du}
\pgfsetdash{}{0pt}
\pgfsetdash{}{0pt}
\pgfsetbuttcap
{
\definecolor{dialinecolor}{rgb}{0.000000, 0.000000, 0.000000}
\pgfsetfillcolor{dialinecolor}
% was here!!!
\pgfsetarrowsend{latex}
\definecolor{dialinecolor}{rgb}{0.000000, 0.000000, 0.000000}
\pgfsetstrokecolor{dialinecolor}
\draw (33.500000\du,37.000000\du)--(33.500000\du,41.500000\du);
}
% setfont left to latex
\definecolor{dialinecolor}{rgb}{0.000000, 0.000000, 0.000000}
\pgfsetstrokecolor{dialinecolor}
\node at (22.500000\du,43.000000\du){OpenCAPI Port 0};
% setfont left to latex
\definecolor{dialinecolor}{rgb}{0.000000, 0.000000, 0.000000}
\pgfsetstrokecolor{dialinecolor}
\node at (33.500000\du,43.000000\du){OpenCAPI Port 1};
\end{tikzpicture}

  \end{center}
  \caption[Overview of fbist\_rlm.vhdl]{\label{fig:fbist_overview}Overview of fbist\_rlm.vhdl
  }
\end{figure}

\subsection{Interfaces}

\subsection{Command Generator}
The FBIST pipeline begins with the command generator. At this point,
there is no address or data; only the command is generated. The
command generator is made up of 3 major parts: command engine
configuration, command arbitration, and command spacing. There are 4
independent command generators in FBIST, one in each parallel
pool. The command generator in each pool has 8 engines that are
arbitrated between. This command generator is illustrated in
Figure~\ref{fig:fbist_cgen}.

A single command engine is programmed with a single command type that
is static during test execution. Any variety of operations within a
test is due to variation in commands programmed in different
engines. For example, to achieve an (approximately) 60/40 mix of reads
and writes, 60\% of the command engines should be programmed for
reads, with the rest being programmed for writes. Primary focus in on
reads and writes, but the interrupt command exists to allow custom
code in the MicroBlaze to perform custom operations. This path allows
a great deal of flexibility without relying on logic for operations
that are not critical to performance, such as writes to configuration
registers on either OCMB or the FPGA. It is also possible to disable a
command engine. Valid commands are:
\begin{itemize}
  \item Read
  \begin{itemize}
    \item 128-Byte
    \item 64-Byte
    \item 32-Byte
  \end{itemize}
  \item Write
  \begin{itemize}
    \item 128-Byte
    \item 64-Byte
    \item 32-Byte
  \end{itemize}
  \item MicroBlaze Interrupt
  \begin{itemize}
    \item 8 bit interrupt ID for custom handling
  \end{itemize}
\end{itemize}

Command arbitration chooses one command a cycle, selecting between the
8 programmed commands. If a certain sequence of commands is required,
the arbitration algorithm can work in a Round-Robin algorithm,
selecting each engine sequentially. If more variety is required, it
can also select in a pseudo-random order, using an LFSR with a
programmed seed.

The arbiter is throttled by the command spacing. The command spacing
allows commands to be sent out at a certain rate. This rate can be
fixed, sending a command every $X$ cycles, or pseudo-random in a
programmed range, with either a uniform or geometric distribution.

The MicroBlaze interrupts do not continue down the pipeline to the
address generator. Rather, they leave the command generator and
proceed directly to the MicroBlaze interrupt handling.

\begin{figure}[h]
  \begin{center}
    % Graphic for TeX using PGF
% Title: /home/rpking42/proj/chips/astra/doc/images/fbist_cgen.dia
% Creator: Dia v0.97.3
% CreationDate: Thu Apr 19 10:44:41 2018
% For: rpking42
% \usepackage{tikz}
% The following commands are not supported in PSTricks at present
% We define them conditionally, so when they are implemented,
% this pgf file will use them.
\ifx\du\undefined
  \newlength{\du}
\fi
\setlength{\du}{15\unitlength}
\begin{tikzpicture}
\pgftransformxscale{1.000000}
\pgftransformyscale{-1.000000}
\definecolor{dialinecolor}{rgb}{0.000000, 0.000000, 0.000000}
\pgfsetstrokecolor{dialinecolor}
\definecolor{dialinecolor}{rgb}{1.000000, 1.000000, 1.000000}
\pgfsetfillcolor{dialinecolor}
\pgfsetlinewidth{0.000000\du}
\pgfsetdash{}{0pt}
\pgfsetdash{}{0pt}
\pgfsetbuttcap
\pgfsetmiterjoin
\pgfsetlinewidth{0.000000\du}
\pgfsetbuttcap
\pgfsetmiterjoin
\pgfsetdash{}{0pt}
\definecolor{dialinecolor}{rgb}{0.000000, 0.000000, 0.000000}
\pgfsetstrokecolor{dialinecolor}
\draw (19.000000\du,-11.000000\du)--(26.000000\du,-11.000000\du)--(26.000000\du,-9.000000\du)--(19.000000\du,-9.000000\du)--cycle;
\pgfsetbuttcap
\pgfsetmiterjoin
\pgfsetdash{}{0pt}
\definecolor{dialinecolor}{rgb}{0.000000, 0.000000, 0.000000}
\pgfsetstrokecolor{dialinecolor}
\draw (19.000000\du,-10.400000\du)--(20.400000\du,-10.000000\du);
\pgfsetbuttcap
\pgfsetmiterjoin
\pgfsetdash{}{0pt}
\definecolor{dialinecolor}{rgb}{0.000000, 0.000000, 0.000000}
\pgfsetstrokecolor{dialinecolor}
\draw (20.400000\du,-10.000000\du)--(19.000000\du,-9.600000\du);
% setfont left to latex
\definecolor{dialinecolor}{rgb}{0.000000, 0.000000, 0.000000}
\pgfsetstrokecolor{dialinecolor}
\node at (23.200000\du,-9.800000\du){CFG\_COMMAND};
\pgfsetlinewidth{0.000000\du}
\pgfsetdash{}{0pt}
\pgfsetdash{}{0pt}
\pgfsetbuttcap
\pgfsetmiterjoin
\pgfsetlinewidth{0.000000\du}
\pgfsetbuttcap
\pgfsetmiterjoin
\pgfsetdash{}{0pt}
\definecolor{dialinecolor}{rgb}{0.000000, 0.000000, 0.000000}
\pgfsetstrokecolor{dialinecolor}
\draw (19.000000\du,-7.000000\du)--(26.000000\du,-7.000000\du)--(26.000000\du,-5.000000\du)--(19.000000\du,-5.000000\du)--cycle;
\pgfsetbuttcap
\pgfsetmiterjoin
\pgfsetdash{}{0pt}
\definecolor{dialinecolor}{rgb}{0.000000, 0.000000, 0.000000}
\pgfsetstrokecolor{dialinecolor}
\draw (19.000000\du,-6.400000\du)--(20.400000\du,-6.000000\du);
\pgfsetbuttcap
\pgfsetmiterjoin
\pgfsetdash{}{0pt}
\definecolor{dialinecolor}{rgb}{0.000000, 0.000000, 0.000000}
\pgfsetstrokecolor{dialinecolor}
\draw (20.400000\du,-6.000000\du)--(19.000000\du,-5.600000\du);
% setfont left to latex
\definecolor{dialinecolor}{rgb}{0.000000, 0.000000, 0.000000}
\pgfsetstrokecolor{dialinecolor}
\node at (23.200000\du,-5.800000\du){CFG\_COMMAND};
\pgfsetlinewidth{0.000000\du}
\pgfsetdash{}{0pt}
\pgfsetdash{}{0pt}
\pgfsetbuttcap
\pgfsetmiterjoin
\pgfsetlinewidth{0.000000\du}
\pgfsetbuttcap
\pgfsetmiterjoin
\pgfsetdash{}{0pt}
\definecolor{dialinecolor}{rgb}{0.000000, 0.000000, 0.000000}
\pgfsetstrokecolor{dialinecolor}
\draw (19.000000\du,-3.000000\du)--(26.000000\du,-3.000000\du)--(26.000000\du,-1.000000\du)--(19.000000\du,-1.000000\du)--cycle;
\pgfsetbuttcap
\pgfsetmiterjoin
\pgfsetdash{}{0pt}
\definecolor{dialinecolor}{rgb}{0.000000, 0.000000, 0.000000}
\pgfsetstrokecolor{dialinecolor}
\draw (19.000000\du,-2.400000\du)--(20.400000\du,-2.000000\du);
\pgfsetbuttcap
\pgfsetmiterjoin
\pgfsetdash{}{0pt}
\definecolor{dialinecolor}{rgb}{0.000000, 0.000000, 0.000000}
\pgfsetstrokecolor{dialinecolor}
\draw (20.400000\du,-2.000000\du)--(19.000000\du,-1.600000\du);
% setfont left to latex
\definecolor{dialinecolor}{rgb}{0.000000, 0.000000, 0.000000}
\pgfsetstrokecolor{dialinecolor}
\node at (23.200000\du,-1.800000\du){CFG\_COMMAND};
\pgfsetlinewidth{0.000000\du}
\pgfsetdash{}{0pt}
\pgfsetdash{}{0pt}
\pgfsetbuttcap
\pgfsetmiterjoin
\pgfsetlinewidth{0.000000\du}
\pgfsetbuttcap
\pgfsetmiterjoin
\pgfsetdash{}{0pt}
\definecolor{dialinecolor}{rgb}{0.000000, 0.000000, 0.000000}
\pgfsetstrokecolor{dialinecolor}
\draw (19.000000\du,1.000000\du)--(26.000000\du,1.000000\du)--(26.000000\du,3.000000\du)--(19.000000\du,3.000000\du)--cycle;
\pgfsetbuttcap
\pgfsetmiterjoin
\pgfsetdash{}{0pt}
\definecolor{dialinecolor}{rgb}{0.000000, 0.000000, 0.000000}
\pgfsetstrokecolor{dialinecolor}
\draw (19.000000\du,1.600000\du)--(20.400000\du,2.000000\du);
\pgfsetbuttcap
\pgfsetmiterjoin
\pgfsetdash{}{0pt}
\definecolor{dialinecolor}{rgb}{0.000000, 0.000000, 0.000000}
\pgfsetstrokecolor{dialinecolor}
\draw (20.400000\du,2.000000\du)--(19.000000\du,2.400000\du);
% setfont left to latex
\definecolor{dialinecolor}{rgb}{0.000000, 0.000000, 0.000000}
\pgfsetstrokecolor{dialinecolor}
\node at (23.200000\du,2.200000\du){CFG\_COMMAND};
\pgfsetlinewidth{0.000000\du}
\pgfsetdash{}{0pt}
\pgfsetdash{}{0pt}
\pgfsetbuttcap
\pgfsetmiterjoin
\pgfsetlinewidth{0.000000\du}
\pgfsetbuttcap
\pgfsetmiterjoin
\pgfsetdash{}{0pt}
\definecolor{dialinecolor}{rgb}{0.000000, 0.000000, 0.000000}
\pgfsetstrokecolor{dialinecolor}
\draw (30.000000\du,-11.000000\du)--(37.000000\du,-11.000000\du)--(37.000000\du,-9.000000\du)--(30.000000\du,-9.000000\du)--cycle;
\pgfsetbuttcap
\pgfsetmiterjoin
\pgfsetdash{}{0pt}
\definecolor{dialinecolor}{rgb}{0.000000, 0.000000, 0.000000}
\pgfsetstrokecolor{dialinecolor}
\draw (30.000000\du,-10.400000\du)--(31.400000\du,-10.000000\du);
\pgfsetbuttcap
\pgfsetmiterjoin
\pgfsetdash{}{0pt}
\definecolor{dialinecolor}{rgb}{0.000000, 0.000000, 0.000000}
\pgfsetstrokecolor{dialinecolor}
\draw (31.400000\du,-10.000000\du)--(30.000000\du,-9.600000\du);
% setfont left to latex
\definecolor{dialinecolor}{rgb}{0.000000, 0.000000, 0.000000}
\pgfsetstrokecolor{dialinecolor}
\node at (34.200000\du,-9.800000\du){CFG\_COMMAND};
\pgfsetlinewidth{0.000000\du}
\pgfsetdash{}{0pt}
\pgfsetdash{}{0pt}
\pgfsetbuttcap
\pgfsetmiterjoin
\pgfsetlinewidth{0.000000\du}
\pgfsetbuttcap
\pgfsetmiterjoin
\pgfsetdash{}{0pt}
\definecolor{dialinecolor}{rgb}{0.000000, 0.000000, 0.000000}
\pgfsetstrokecolor{dialinecolor}
\draw (30.000000\du,-7.000000\du)--(37.000000\du,-7.000000\du)--(37.000000\du,-5.000000\du)--(30.000000\du,-5.000000\du)--cycle;
\pgfsetbuttcap
\pgfsetmiterjoin
\pgfsetdash{}{0pt}
\definecolor{dialinecolor}{rgb}{0.000000, 0.000000, 0.000000}
\pgfsetstrokecolor{dialinecolor}
\draw (30.000000\du,-6.400000\du)--(31.400000\du,-6.000000\du);
\pgfsetbuttcap
\pgfsetmiterjoin
\pgfsetdash{}{0pt}
\definecolor{dialinecolor}{rgb}{0.000000, 0.000000, 0.000000}
\pgfsetstrokecolor{dialinecolor}
\draw (31.400000\du,-6.000000\du)--(30.000000\du,-5.600000\du);
% setfont left to latex
\definecolor{dialinecolor}{rgb}{0.000000, 0.000000, 0.000000}
\pgfsetstrokecolor{dialinecolor}
\node at (34.200000\du,-5.800000\du){CFG\_COMMAND};
\pgfsetlinewidth{0.000000\du}
\pgfsetdash{}{0pt}
\pgfsetdash{}{0pt}
\pgfsetbuttcap
\pgfsetmiterjoin
\pgfsetlinewidth{0.000000\du}
\pgfsetbuttcap
\pgfsetmiterjoin
\pgfsetdash{}{0pt}
\definecolor{dialinecolor}{rgb}{0.000000, 0.000000, 0.000000}
\pgfsetstrokecolor{dialinecolor}
\draw (30.000000\du,-3.000000\du)--(37.000000\du,-3.000000\du)--(37.000000\du,-1.000000\du)--(30.000000\du,-1.000000\du)--cycle;
\pgfsetbuttcap
\pgfsetmiterjoin
\pgfsetdash{}{0pt}
\definecolor{dialinecolor}{rgb}{0.000000, 0.000000, 0.000000}
\pgfsetstrokecolor{dialinecolor}
\draw (30.000000\du,-2.400000\du)--(31.400000\du,-2.000000\du);
\pgfsetbuttcap
\pgfsetmiterjoin
\pgfsetdash{}{0pt}
\definecolor{dialinecolor}{rgb}{0.000000, 0.000000, 0.000000}
\pgfsetstrokecolor{dialinecolor}
\draw (31.400000\du,-2.000000\du)--(30.000000\du,-1.600000\du);
% setfont left to latex
\definecolor{dialinecolor}{rgb}{0.000000, 0.000000, 0.000000}
\pgfsetstrokecolor{dialinecolor}
\node at (34.200000\du,-1.800000\du){CFG\_COMMAND};
\pgfsetlinewidth{0.000000\du}
\pgfsetdash{}{0pt}
\pgfsetdash{}{0pt}
\pgfsetbuttcap
\pgfsetmiterjoin
\pgfsetlinewidth{0.000000\du}
\pgfsetbuttcap
\pgfsetmiterjoin
\pgfsetdash{}{0pt}
\definecolor{dialinecolor}{rgb}{0.000000, 0.000000, 0.000000}
\pgfsetstrokecolor{dialinecolor}
\draw (30.000000\du,1.000000\du)--(37.000000\du,1.000000\du)--(37.000000\du,3.000000\du)--(30.000000\du,3.000000\du)--cycle;
\pgfsetbuttcap
\pgfsetmiterjoin
\pgfsetdash{}{0pt}
\definecolor{dialinecolor}{rgb}{0.000000, 0.000000, 0.000000}
\pgfsetstrokecolor{dialinecolor}
\draw (30.000000\du,1.600000\du)--(31.400000\du,2.000000\du);
\pgfsetbuttcap
\pgfsetmiterjoin
\pgfsetdash{}{0pt}
\definecolor{dialinecolor}{rgb}{0.000000, 0.000000, 0.000000}
\pgfsetstrokecolor{dialinecolor}
\draw (31.400000\du,2.000000\du)--(30.000000\du,2.400000\du);
% setfont left to latex
\definecolor{dialinecolor}{rgb}{0.000000, 0.000000, 0.000000}
\pgfsetstrokecolor{dialinecolor}
\node at (34.200000\du,2.200000\du){CFG\_COMMAND};
\pgfsetlinewidth{0.000000\du}
\pgfsetdash{}{0pt}
\pgfsetdash{}{0pt}
\pgfsetbuttcap
\pgfsetmiterjoin
\pgfsetlinewidth{0.000000\du}
\pgfsetbuttcap
\pgfsetmiterjoin
\pgfsetdash{}{0pt}
\definecolor{dialinecolor}{rgb}{0.000000, 0.000000, 0.000000}
\pgfsetstrokecolor{dialinecolor}
\draw (38.200000\du,7.000000\du)--(41.800000\du,7.000000\du)--(40.900000\du,8.400000\du)--(39.100000\du,8.400000\du)--cycle;
% setfont left to latex
\definecolor{dialinecolor}{rgb}{0.000000, 0.000000, 0.000000}
\pgfsetstrokecolor{dialinecolor}
\node at (40.000000\du,7.900000\du){};
\pgfsetlinewidth{0.000000\du}
\pgfsetdash{}{0pt}
\pgfsetdash{}{0pt}
\pgfsetbuttcap
{
\definecolor{dialinecolor}{rgb}{0.000000, 0.000000, 0.000000}
\pgfsetfillcolor{dialinecolor}
% was here!!!
\definecolor{dialinecolor}{rgb}{0.000000, 0.000000, 0.000000}
\pgfsetstrokecolor{dialinecolor}
\draw (39.100000\du,7.000000\du)--(40.900000\du,7.000000\du);
}
\pgfsetlinewidth{0.000000\du}
\pgfsetdash{}{0pt}
\pgfsetdash{}{0pt}
\pgfsetmiterjoin
\pgfsetbuttcap
{
\definecolor{dialinecolor}{rgb}{0.000000, 0.000000, 0.000000}
\pgfsetfillcolor{dialinecolor}
% was here!!!
{\pgfsetcornersarced{\pgfpoint{0.000000\du}{0.000000\du}}\definecolor{dialinecolor}{rgb}{0.000000, 0.000000, 0.000000}
\pgfsetstrokecolor{dialinecolor}
\draw (22.500000\du,-9.000000\du)--(22.500000\du,-8.000000\du)--(29.000000\du,-8.000000\du)--(29.000000\du,4.000000\du)--(39.900000\du,4.000000\du)--(39.900000\du,7.000000\du)--(39.900000\du,7.000000\du);
}}
\pgfsetlinewidth{0.000000\du}
\pgfsetdash{}{0pt}
\pgfsetdash{}{0pt}
\pgfsetmiterjoin
\pgfsetbuttcap
{
\definecolor{dialinecolor}{rgb}{0.000000, 0.000000, 0.000000}
\pgfsetfillcolor{dialinecolor}
% was here!!!
{\pgfsetcornersarced{\pgfpoint{0.000000\du}{0.000000\du}}\definecolor{dialinecolor}{rgb}{0.000000, 0.000000, 0.000000}
\pgfsetstrokecolor{dialinecolor}
\draw (22.500000\du,-5.000000\du)--(22.500000\du,-4.000000\du)--(28.000000\du,-4.000000\du)--(28.000000\du,4.500000\du)--(39.700000\du,4.500000\du)--(39.700000\du,7.000000\du)--(39.700000\du,7.000000\du);
}}
\pgfsetlinewidth{0.000000\du}
\pgfsetdash{}{0pt}
\pgfsetdash{}{0pt}
\pgfsetmiterjoin
\pgfsetbuttcap
{
\definecolor{dialinecolor}{rgb}{0.000000, 0.000000, 0.000000}
\pgfsetfillcolor{dialinecolor}
% was here!!!
{\pgfsetcornersarced{\pgfpoint{0.000000\du}{0.000000\du}}\definecolor{dialinecolor}{rgb}{0.000000, 0.000000, 0.000000}
\pgfsetstrokecolor{dialinecolor}
\draw (22.700000\du,-1.000000\du)--(22.700000\du,0.000000\du)--(27.200000\du,0.000000\du)--(27.200000\du,5.000000\du)--(39.500000\du,5.000000\du)--(39.500000\du,7.000000\du)--(39.500000\du,7.000000\du);
}}
\pgfsetlinewidth{0.000000\du}
\pgfsetdash{}{0pt}
\pgfsetdash{}{0pt}
\pgfsetmiterjoin
\pgfsetbuttcap
{
\definecolor{dialinecolor}{rgb}{0.000000, 0.000000, 0.000000}
\pgfsetfillcolor{dialinecolor}
% was here!!!
{\pgfsetcornersarced{\pgfpoint{0.000000\du}{0.000000\du}}\definecolor{dialinecolor}{rgb}{0.000000, 0.000000, 0.000000}
\pgfsetstrokecolor{dialinecolor}
\draw (22.500000\du,3.000000\du)--(22.500000\du,4.000000\du)--(26.000000\du,4.000000\du)--(26.000000\du,5.500000\du)--(39.300000\du,5.500000\du)--(39.300000\du,7.000000\du)--(39.300000\du,7.000000\du);
}}
\pgfsetlinewidth{0.000000\du}
\pgfsetdash{}{0pt}
\pgfsetdash{}{0pt}
\pgfsetmiterjoin
\pgfsetbuttcap
{
\definecolor{dialinecolor}{rgb}{0.000000, 0.000000, 0.000000}
\pgfsetfillcolor{dialinecolor}
% was here!!!
{\pgfsetcornersarced{\pgfpoint{0.000000\du}{0.000000\du}}\definecolor{dialinecolor}{rgb}{0.000000, 0.000000, 0.000000}
\pgfsetstrokecolor{dialinecolor}
\draw (33.500000\du,3.000000\du)--(33.500000\du,3.500000\du)--(40.100000\du,3.500000\du)--(40.100000\du,7.000000\du);
}}
\pgfsetlinewidth{0.000000\du}
\pgfsetdash{}{0pt}
\pgfsetdash{}{0pt}
\pgfsetmiterjoin
\pgfsetbuttcap
{
\definecolor{dialinecolor}{rgb}{0.000000, 0.000000, 0.000000}
\pgfsetfillcolor{dialinecolor}
% was here!!!
{\pgfsetcornersarced{\pgfpoint{0.000000\du}{0.000000\du}}\definecolor{dialinecolor}{rgb}{0.000000, 0.000000, 0.000000}
\pgfsetstrokecolor{dialinecolor}
\draw (33.500000\du,-1.000000\du)--(33.500000\du,0.000000\du)--(38.000000\du,0.000000\du)--(38.000000\du,3.000000\du)--(40.300000\du,3.000000\du)--(40.300000\du,7.000000\du)--(40.300000\du,7.000000\du);
}}
\pgfsetlinewidth{0.000000\du}
\pgfsetdash{}{0pt}
\pgfsetdash{}{0pt}
\pgfsetmiterjoin
\pgfsetbuttcap
{
\definecolor{dialinecolor}{rgb}{0.000000, 0.000000, 0.000000}
\pgfsetfillcolor{dialinecolor}
% was here!!!
{\pgfsetcornersarced{\pgfpoint{0.000000\du}{0.000000\du}}\definecolor{dialinecolor}{rgb}{0.000000, 0.000000, 0.000000}
\pgfsetstrokecolor{dialinecolor}
\draw (33.500000\du,-5.000000\du)--(33.500000\du,-4.000000\du)--(39.000000\du,-4.000000\du)--(39.000000\du,2.500000\du)--(40.500000\du,2.500000\du)--(40.500000\du,7.000000\du);
}}
\pgfsetlinewidth{0.000000\du}
\pgfsetdash{}{0pt}
\pgfsetdash{}{0pt}
\pgfsetmiterjoin
\pgfsetbuttcap
{
\definecolor{dialinecolor}{rgb}{0.000000, 0.000000, 0.000000}
\pgfsetfillcolor{dialinecolor}
% was here!!!
{\pgfsetcornersarced{\pgfpoint{0.000000\du}{0.000000\du}}\definecolor{dialinecolor}{rgb}{0.000000, 0.000000, 0.000000}
\pgfsetstrokecolor{dialinecolor}
\draw (33.500000\du,-9.000000\du)--(33.500000\du,-8.000000\du)--(40.000000\du,-8.000000\du)--(40.000000\du,2.000000\du)--(40.700000\du,2.000000\du)--(40.700000\du,7.000000\du);
}}
% setfont left to latex
\definecolor{dialinecolor}{rgb}{0.000000, 0.000000, 0.000000}
\pgfsetstrokecolor{dialinecolor}
\node at (22.500000\du,-11.152500\du){Engine 0};
% setfont left to latex
\definecolor{dialinecolor}{rgb}{0.000000, 0.000000, 0.000000}
\pgfsetstrokecolor{dialinecolor}
\node at (22.500000\du,-7.152500\du){Engine 1};
% setfont left to latex
\definecolor{dialinecolor}{rgb}{0.000000, 0.000000, 0.000000}
\pgfsetstrokecolor{dialinecolor}
\node at (22.500000\du,-3.152500\du){Engine 2};
% setfont left to latex
\definecolor{dialinecolor}{rgb}{0.000000, 0.000000, 0.000000}
\pgfsetstrokecolor{dialinecolor}
\node at (22.500000\du,0.847500\du){Engine 3};
% setfont left to latex
\definecolor{dialinecolor}{rgb}{0.000000, 0.000000, 0.000000}
\pgfsetstrokecolor{dialinecolor}
\node at (33.500000\du,-11.152500\du){Engine 4};
% setfont left to latex
\definecolor{dialinecolor}{rgb}{0.000000, 0.000000, 0.000000}
\pgfsetstrokecolor{dialinecolor}
\node at (33.500000\du,-7.152500\du){Engine 5};
% setfont left to latex
\definecolor{dialinecolor}{rgb}{0.000000, 0.000000, 0.000000}
\pgfsetstrokecolor{dialinecolor}
\node at (33.500000\du,-3.152500\du){Engine 6};
% setfont left to latex
\definecolor{dialinecolor}{rgb}{0.000000, 0.000000, 0.000000}
\pgfsetstrokecolor{dialinecolor}
\node at (33.500000\du,0.847500\du){Engine 7};
\pgfsetlinewidth{0.000000\du}
\pgfsetdash{}{0pt}
\pgfsetdash{}{0pt}
\pgfsetbuttcap
\pgfsetmiterjoin
\pgfsetlinewidth{0.000000\du}
\pgfsetbuttcap
\pgfsetmiterjoin
\pgfsetdash{}{0pt}
\definecolor{dialinecolor}{rgb}{0.000000, 0.000000, 0.000000}
\pgfsetstrokecolor{dialinecolor}
\draw (38.000000\du,21.000000\du)--(44.000000\du,21.000000\du)--(44.000000\du,23.000000\du)--(38.000000\du,23.000000\du)--cycle;
\pgfsetbuttcap
\pgfsetmiterjoin
\pgfsetdash{}{0pt}
\definecolor{dialinecolor}{rgb}{0.000000, 0.000000, 0.000000}
\pgfsetstrokecolor{dialinecolor}
\draw (38.000000\du,21.600000\du)--(39.200000\du,22.000000\du);
\pgfsetbuttcap
\pgfsetmiterjoin
\pgfsetdash{}{0pt}
\definecolor{dialinecolor}{rgb}{0.000000, 0.000000, 0.000000}
\pgfsetstrokecolor{dialinecolor}
\draw (39.200000\du,22.000000\du)--(38.000000\du,22.400000\du);
% setfont left to latex
\definecolor{dialinecolor}{rgb}{0.000000, 0.000000, 0.000000}
\pgfsetstrokecolor{dialinecolor}
\node at (41.600000\du,22.200000\du){command};
\pgfsetlinewidth{0.000000\du}
\pgfsetdash{}{0pt}
\pgfsetdash{}{0pt}
\pgfsetbuttcap
{
\definecolor{dialinecolor}{rgb}{0.000000, 0.000000, 0.000000}
\pgfsetfillcolor{dialinecolor}
% was here!!!
\pgfsetarrowsend{latex}
\definecolor{dialinecolor}{rgb}{0.000000, 0.000000, 0.000000}
\pgfsetstrokecolor{dialinecolor}
\draw (41.000000\du,23.000000\du)--(41.000000\du,25.000000\du);
}
\pgfsetlinewidth{0.000000\du}
\pgfsetdash{}{0pt}
\pgfsetdash{}{0pt}
\pgfsetbuttcap
\pgfsetmiterjoin
\pgfsetlinewidth{0.000000\du}
\pgfsetbuttcap
\pgfsetmiterjoin
\pgfsetdash{}{0pt}
\definecolor{dialinecolor}{rgb}{0.000000, 0.000000, 0.000000}
\pgfsetstrokecolor{dialinecolor}
\draw (21.000000\du,6.000000\du)--(29.000000\du,6.000000\du)--(29.000000\du,8.000000\du)--(21.000000\du,8.000000\du)--cycle;
\pgfsetbuttcap
\pgfsetmiterjoin
\pgfsetdash{}{0pt}
\definecolor{dialinecolor}{rgb}{0.000000, 0.000000, 0.000000}
\pgfsetstrokecolor{dialinecolor}
\draw (21.000000\du,6.600000\du)--(22.600000\du,7.000000\du);
\pgfsetbuttcap
\pgfsetmiterjoin
\pgfsetdash{}{0pt}
\definecolor{dialinecolor}{rgb}{0.000000, 0.000000, 0.000000}
\pgfsetstrokecolor{dialinecolor}
\draw (22.600000\du,7.000000\du)--(21.000000\du,7.400000\du);
% setfont left to latex
\definecolor{dialinecolor}{rgb}{0.000000, 0.000000, 0.000000}
\pgfsetstrokecolor{dialinecolor}
\node at (25.800000\du,7.200000\du){CFG\_ARB\_SCHEME};
\pgfsetlinewidth{0.000000\du}
\pgfsetdash{}{0pt}
\pgfsetdash{}{0pt}
\pgfsetbuttcap
\pgfsetmiterjoin
\pgfsetlinewidth{0.000000\du}
\pgfsetbuttcap
\pgfsetmiterjoin
\pgfsetdash{}{0pt}
\definecolor{dialinecolor}{rgb}{0.000000, 0.000000, 0.000000}
\pgfsetstrokecolor{dialinecolor}
\draw (35.000000\du,8.500000\du)--(36.000000\du,8.500000\du)--(36.000000\du,10.500000\du)--(35.000000\du,10.500000\du)--cycle;
\pgfsetbuttcap
\pgfsetmiterjoin
\pgfsetdash{}{0pt}
\definecolor{dialinecolor}{rgb}{0.000000, 0.000000, 0.000000}
\pgfsetstrokecolor{dialinecolor}
\draw (35.300000\du,10.500000\du)--(35.500000\du,10.100000\du);
\pgfsetbuttcap
\pgfsetmiterjoin
\pgfsetdash{}{0pt}
\definecolor{dialinecolor}{rgb}{0.000000, 0.000000, 0.000000}
\pgfsetstrokecolor{dialinecolor}
\draw (35.500000\du,10.100000\du)--(35.700000\du,10.500000\du);
% setfont left to latex
\definecolor{dialinecolor}{rgb}{0.000000, 0.000000, 0.000000}
\pgfsetstrokecolor{dialinecolor}
\node at (35.500000\du,9.500000\du){};
\pgfsetlinewidth{0.000000\du}
\pgfsetdash{}{0pt}
\pgfsetdash{}{0pt}
\pgfsetbuttcap
\pgfsetmiterjoin
\pgfsetlinewidth{0.000000\du}
\pgfsetbuttcap
\pgfsetmiterjoin
\pgfsetdash{}{0pt}
\definecolor{dialinecolor}{rgb}{0.000000, 0.000000, 0.000000}
\pgfsetstrokecolor{dialinecolor}
\draw (32.000000\du,21.000000\du)--(36.000000\du,21.000000\du)--(36.000000\du,23.000000\du)--(32.000000\du,23.000000\du)--cycle;
\pgfsetbuttcap
\pgfsetmiterjoin
\pgfsetdash{}{0pt}
\definecolor{dialinecolor}{rgb}{0.000000, 0.000000, 0.000000}
\pgfsetstrokecolor{dialinecolor}
\draw (32.000000\du,21.600000\du)--(32.800000\du,22.000000\du);
\pgfsetbuttcap
\pgfsetmiterjoin
\pgfsetdash{}{0pt}
\definecolor{dialinecolor}{rgb}{0.000000, 0.000000, 0.000000}
\pgfsetstrokecolor{dialinecolor}
\draw (32.800000\du,22.000000\du)--(32.000000\du,22.400000\du);
% setfont left to latex
\definecolor{dialinecolor}{rgb}{0.000000, 0.000000, 0.000000}
\pgfsetstrokecolor{dialinecolor}
\node at (34.400000\du,22.200000\du){engine};
\pgfsetlinewidth{0.000000\du}
\pgfsetdash{}{0pt}
\pgfsetdash{}{0pt}
\pgfsetmiterjoin
\pgfsetbuttcap
{
\definecolor{dialinecolor}{rgb}{0.000000, 0.000000, 0.000000}
\pgfsetfillcolor{dialinecolor}
% was here!!!
{\pgfsetcornersarced{\pgfpoint{0.000000\du}{0.000000\du}}\definecolor{dialinecolor}{rgb}{0.000000, 0.000000, 0.000000}
\pgfsetstrokecolor{dialinecolor}
\draw (36.000000\du,9.500000\du)--(36.500000\du,9.500000\du)--(36.500000\du,17.000000\du)--(34.000000\du,17.000000\du)--(34.000000\du,21.000000\du);
}}
\pgfsetlinewidth{0.000000\du}
\pgfsetdash{}{0pt}
\pgfsetdash{}{0pt}
\pgfsetmiterjoin
\pgfsetbuttcap
{
\definecolor{dialinecolor}{rgb}{0.000000, 0.000000, 0.000000}
\pgfsetfillcolor{dialinecolor}
% was here!!!
\pgfsetarrowsend{latex}
{\pgfsetcornersarced{\pgfpoint{0.000000\du}{0.000000\du}}\definecolor{dialinecolor}{rgb}{0.000000, 0.000000, 0.000000}
\pgfsetstrokecolor{dialinecolor}
\draw (36.000000\du,9.500000\du)--(37.500000\du,9.500000\du)--(37.500000\du,7.700000\du)--(38.650000\du,7.700000\du);
}}
\pgfsetlinewidth{0.000000\du}
\pgfsetdash{}{0pt}
\pgfsetdash{}{0pt}
\pgfsetbuttcap
{
\definecolor{dialinecolor}{rgb}{0.000000, 0.000000, 0.000000}
\pgfsetfillcolor{dialinecolor}
% was here!!!
\pgfsetarrowsend{latex}
\definecolor{dialinecolor}{rgb}{0.000000, 0.000000, 0.000000}
\pgfsetstrokecolor{dialinecolor}
\draw (34.000000\du,23.000000\du)--(34.000000\du,25.000000\du);
}
\pgfsetlinewidth{0.000000\du}
\pgfsetdash{}{0pt}
\pgfsetdash{}{0pt}
\pgfsetbuttcap
\pgfsetmiterjoin
\pgfsetlinewidth{0.000000\du}
\pgfsetbuttcap
\pgfsetmiterjoin
\pgfsetdash{}{0pt}
\definecolor{dialinecolor}{rgb}{1.000000, 1.000000, 1.000000}
\pgfsetfillcolor{dialinecolor}
\pgfpathmoveto{\pgfpoint{29.138499\du}{9.871624\du}}
\pgfpathcurveto{\pgfpoint{28.751174\du}{9.861488\du}}{\pgfpoint{28.000000\du}{10.074326\du}}{\pgfpoint{28.105634\du}{10.530408\du}}
\pgfpathcurveto{\pgfpoint{28.211267\du}{10.986490\du}}{\pgfpoint{28.715963\du}{11.087837\du}}{\pgfpoint{28.927231\du}{10.956084\du}}
\pgfpathcurveto{\pgfpoint{29.138499\du}{10.824327\du}}{\pgfpoint{28.598592\du}{11.594594\du}}{\pgfpoint{29.631458\du}{11.797297\du}}
\pgfpathcurveto{\pgfpoint{30.664314\du}{12.000000\du}}{\pgfpoint{31.192484\du}{11.675675\du}}{\pgfpoint{31.039902\du}{11.442567\du}}
\pgfpathcurveto{\pgfpoint{30.887319\du}{11.209459\du}}{\pgfpoint{31.943660\du}{11.989865\du}}{\pgfpoint{32.436619\du}{11.543918\du}}
\pgfpathcurveto{\pgfpoint{32.929577\du}{11.097972\du}}{\pgfpoint{31.931923\du}{10.672300\du}}{\pgfpoint{32.143191\du}{10.733111\du}}
\pgfpathcurveto{\pgfpoint{32.354459\du}{10.793922\du}}{\pgfpoint{33.000000\du}{10.712841\du}}{\pgfpoint{32.788732\du}{9.952705\du}}
\pgfpathcurveto{\pgfpoint{32.577464\du}{9.192569\du}}{\pgfpoint{30.676051\du}{9.780407\du}}{\pgfpoint{30.887319\du}{9.668921\du}}
\pgfpathcurveto{\pgfpoint{31.098588\du}{9.557434\du}}{\pgfpoint{30.570417\du}{9.000000\du}}{\pgfpoint{29.913148\du}{9.111487\du}}
\pgfpathcurveto{\pgfpoint{29.255870\du}{9.222974\du}}{\pgfpoint{29.209250\du}{9.425282\du}}{\pgfpoint{29.138828\du}{9.871228\du}}
\pgfpathlineto{\pgfpoint{29.138499\du}{9.871624\du}}
\pgfusepath{fill}
\definecolor{dialinecolor}{rgb}{0.000000, 0.000000, 0.000000}
\pgfsetstrokecolor{dialinecolor}
\pgfpathmoveto{\pgfpoint{29.138499\du}{9.871624\du}}
\pgfpathcurveto{\pgfpoint{28.751174\du}{9.861488\du}}{\pgfpoint{28.000000\du}{10.074326\du}}{\pgfpoint{28.105634\du}{10.530408\du}}
\pgfpathcurveto{\pgfpoint{28.211267\du}{10.986490\du}}{\pgfpoint{28.715963\du}{11.087837\du}}{\pgfpoint{28.927231\du}{10.956084\du}}
\pgfpathcurveto{\pgfpoint{29.138499\du}{10.824327\du}}{\pgfpoint{28.598592\du}{11.594594\du}}{\pgfpoint{29.631458\du}{11.797297\du}}
\pgfpathcurveto{\pgfpoint{30.664314\du}{12.000000\du}}{\pgfpoint{31.192484\du}{11.675675\du}}{\pgfpoint{31.039902\du}{11.442567\du}}
\pgfpathcurveto{\pgfpoint{30.887319\du}{11.209459\du}}{\pgfpoint{31.943660\du}{11.989865\du}}{\pgfpoint{32.436619\du}{11.543918\du}}
\pgfpathcurveto{\pgfpoint{32.929577\du}{11.097972\du}}{\pgfpoint{31.931923\du}{10.672300\du}}{\pgfpoint{32.143191\du}{10.733111\du}}
\pgfpathcurveto{\pgfpoint{32.354459\du}{10.793922\du}}{\pgfpoint{33.000000\du}{10.712841\du}}{\pgfpoint{32.788732\du}{9.952705\du}}
\pgfpathcurveto{\pgfpoint{32.577464\du}{9.192569\du}}{\pgfpoint{30.676051\du}{9.780407\du}}{\pgfpoint{30.887319\du}{9.668921\du}}
\pgfpathcurveto{\pgfpoint{31.098588\du}{9.557434\du}}{\pgfpoint{30.570417\du}{9.000000\du}}{\pgfpoint{29.913148\du}{9.111487\du}}
\pgfpathcurveto{\pgfpoint{29.255870\du}{9.222974\du}}{\pgfpoint{29.209250\du}{9.425282\du}}{\pgfpoint{29.138828\du}{9.871228\du}}
\pgfpathlineto{\pgfpoint{29.138499\du}{9.871624\du}}
\pgfusepath{stroke}
% setfont left to latex
\definecolor{dialinecolor}{rgb}{0.000000, 0.000000, 0.000000}
\pgfsetstrokecolor{dialinecolor}
\node at (30.634033\du,10.752884\du){};
\pgfsetlinewidth{0.000000\du}
\pgfsetdash{}{0pt}
\pgfsetdash{}{0pt}
\pgfsetmiterjoin
\pgfsetbuttcap
{
\definecolor{dialinecolor}{rgb}{0.000000, 0.000000, 0.000000}
\pgfsetfillcolor{dialinecolor}
% was here!!!
\pgfsetarrowsend{latex}
{\pgfsetcornersarced{\pgfpoint{0.000000\du}{0.000000\du}}\definecolor{dialinecolor}{rgb}{0.000000, 0.000000, 0.000000}
\pgfsetstrokecolor{dialinecolor}
\draw (32.828500\du,10.237700\du)--(34.000000\du,10.237700\du)--(34.000000\du,9.500000\du)--(35.000000\du,9.500000\du)--(35.000000\du,9.500000\du);
}}
\pgfsetlinewidth{0.000000\du}
\pgfsetdash{}{0pt}
\pgfsetdash{}{0pt}
\pgfsetmiterjoin
\pgfsetbuttcap
{
\definecolor{dialinecolor}{rgb}{0.000000, 0.000000, 0.000000}
\pgfsetfillcolor{dialinecolor}
% was here!!!
\pgfsetarrowsend{latex}
{\pgfsetcornersarced{\pgfpoint{0.000000\du}{0.000000\du}}\definecolor{dialinecolor}{rgb}{0.000000, 0.000000, 0.000000}
\pgfsetstrokecolor{dialinecolor}
\draw (36.000000\du,9.500000\du)--(37.000000\du,9.500000\du)--(37.000000\du,7.500000\du)--(30.080200\du,7.500000\du)--(30.080200\du,9.097800\du);
}}
\pgfsetlinewidth{0.000000\du}
\pgfsetdash{}{0pt}
\pgfsetdash{}{0pt}
\pgfsetmiterjoin
\pgfsetbuttcap
{
\definecolor{dialinecolor}{rgb}{0.000000, 0.000000, 0.000000}
\pgfsetfillcolor{dialinecolor}
% was here!!!
\pgfsetarrowsend{latex}
{\pgfsetcornersarced{\pgfpoint{0.000000\du}{0.000000\du}}\definecolor{dialinecolor}{rgb}{0.000000, 0.000000, 0.000000}
\pgfsetstrokecolor{dialinecolor}
\draw (25.000000\du,8.000000\du)--(27.000000\du,8.000000\du)--(27.000000\du,10.455600\du)--(28.093300\du,10.455600\du);
}}
\pgfsetlinewidth{0.000000\du}
\pgfsetdash{}{0pt}
\pgfsetdash{}{0pt}
\pgfsetbuttcap
\pgfsetmiterjoin
\pgfsetlinewidth{0.000000\du}
\pgfsetbuttcap
\pgfsetmiterjoin
\pgfsetdash{}{0pt}
\definecolor{dialinecolor}{rgb}{0.000000, 0.000000, 0.000000}
\pgfsetstrokecolor{dialinecolor}
\draw (17.000000\du,19.000000\du)--(23.000000\du,19.000000\du)--(23.000000\du,21.000000\du)--(17.000000\du,21.000000\du)--cycle;
\pgfsetbuttcap
\pgfsetmiterjoin
\pgfsetdash{}{0pt}
\definecolor{dialinecolor}{rgb}{0.000000, 0.000000, 0.000000}
\pgfsetstrokecolor{dialinecolor}
\draw (17.000000\du,19.600000\du)--(18.200000\du,20.000000\du);
\pgfsetbuttcap
\pgfsetmiterjoin
\pgfsetdash{}{0pt}
\definecolor{dialinecolor}{rgb}{0.000000, 0.000000, 0.000000}
\pgfsetstrokecolor{dialinecolor}
\draw (18.200000\du,20.000000\du)--(17.000000\du,20.400000\du);
% setfont left to latex
\definecolor{dialinecolor}{rgb}{0.000000, 0.000000, 0.000000}
\pgfsetstrokecolor{dialinecolor}
\node at (20.600000\du,20.200000\du){CFG\_SPACING};
\pgfsetlinewidth{0.000000\du}
\pgfsetdash{}{0pt}
\pgfsetdash{}{0pt}
\pgfsetbuttcap
\pgfsetmiterjoin
\pgfsetlinewidth{0.000000\du}
\pgfsetbuttcap
\pgfsetmiterjoin
\pgfsetdash{}{0pt}
\definecolor{dialinecolor}{rgb}{0.000000, 0.000000, 0.000000}
\pgfsetstrokecolor{dialinecolor}
\draw (26.000000\du,21.000000\du)--(30.000000\du,21.000000\du)--(30.000000\du,23.000000\du)--(26.000000\du,23.000000\du)--cycle;
\pgfsetbuttcap
\pgfsetmiterjoin
\pgfsetdash{}{0pt}
\definecolor{dialinecolor}{rgb}{0.000000, 0.000000, 0.000000}
\pgfsetstrokecolor{dialinecolor}
\draw (26.000000\du,21.600000\du)--(26.800000\du,22.000000\du);
\pgfsetbuttcap
\pgfsetmiterjoin
\pgfsetdash{}{0pt}
\definecolor{dialinecolor}{rgb}{0.000000, 0.000000, 0.000000}
\pgfsetstrokecolor{dialinecolor}
\draw (26.800000\du,22.000000\du)--(26.000000\du,22.400000\du);
% setfont left to latex
\definecolor{dialinecolor}{rgb}{0.000000, 0.000000, 0.000000}
\pgfsetstrokecolor{dialinecolor}
\node at (28.400000\du,22.200000\du){valid};
\pgfsetlinewidth{0.000000\du}
\pgfsetdash{}{0pt}
\pgfsetdash{}{0pt}
\pgfsetmiterjoin
\pgfsetbuttcap
{
\definecolor{dialinecolor}{rgb}{0.000000, 0.000000, 0.000000}
\pgfsetfillcolor{dialinecolor}
% was here!!!
{\pgfsetcornersarced{\pgfpoint{0.000000\du}{0.000000\du}}\definecolor{dialinecolor}{rgb}{0.000000, 0.000000, 0.000000}
\pgfsetstrokecolor{dialinecolor}
\draw (27.000000\du,14.000000\du)--(28.000000\du,14.000000\du)--(28.000000\du,19.000000\du);
}}
\pgfsetlinewidth{0.000000\du}
\pgfsetdash{}{0pt}
\pgfsetdash{}{0pt}
\pgfsetmiterjoin
\pgfsetbuttcap
{
\definecolor{dialinecolor}{rgb}{0.000000, 0.000000, 0.000000}
\pgfsetfillcolor{dialinecolor}
% was here!!!
\pgfsetarrowsend{latex}
{\pgfsetcornersarced{\pgfpoint{0.000000\du}{0.000000\du}}\definecolor{dialinecolor}{rgb}{0.000000, 0.000000, 0.000000}
\pgfsetstrokecolor{dialinecolor}
\draw (20.000000\du,19.000000\du)--(20.000000\du,18.000000\du)--(18.600000\du,18.000000\du)--(18.600000\du,15.552700\du)--(20.293300\du,15.552700\du);
}}
\pgfsetlinewidth{0.000000\du}
\pgfsetdash{}{0pt}
\pgfsetdash{}{0pt}
\pgfsetbuttcap
{
\definecolor{dialinecolor}{rgb}{0.000000, 0.000000, 0.000000}
\pgfsetfillcolor{dialinecolor}
% was here!!!
\pgfsetarrowsend{latex}
\definecolor{dialinecolor}{rgb}{0.000000, 0.000000, 0.000000}
\pgfsetstrokecolor{dialinecolor}
\draw (28.000000\du,23.000000\du)--(28.000000\du,25.000000\du);
}
\pgfsetlinewidth{0.000000\du}
\pgfsetdash{}{0pt}
\pgfsetdash{}{0pt}
\pgfsetmiterjoin
\pgfsetbuttcap
{
\definecolor{dialinecolor}{rgb}{0.000000, 0.000000, 0.000000}
\pgfsetfillcolor{dialinecolor}
% was here!!!
\pgfsetarrowsend{latex}
{\pgfsetcornersarced{\pgfpoint{0.000000\du}{0.000000\du}}\definecolor{dialinecolor}{rgb}{0.000000, 0.000000, 0.000000}
\pgfsetstrokecolor{dialinecolor}
\draw (27.000000\du,14.000000\du)--(30.243800\du,14.000000\du)--(30.243800\du,11.862500\du);
}}
\pgfsetlinewidth{0.000000\du}
\pgfsetdash{}{0pt}
\pgfsetdash{}{0pt}
\pgfsetbuttcap
\pgfsetmiterjoin
\pgfsetlinewidth{0.000000\du}
\pgfsetbuttcap
\pgfsetmiterjoin
\pgfsetdash{}{0pt}
\definecolor{dialinecolor}{rgb}{1.000000, 1.000000, 1.000000}
\pgfsetfillcolor{dialinecolor}
\pgfpathmoveto{\pgfpoint{21.338499\du}{14.929732\du}}
\pgfpathcurveto{\pgfpoint{20.951174\du}{14.918921\du}}{\pgfpoint{20.200000\du}{15.145948\du}}{\pgfpoint{20.305634\du}{15.632435\du}}
\pgfpathcurveto{\pgfpoint{20.411267\du}{16.118922\du}}{\pgfpoint{20.915963\du}{16.227026\du}}{\pgfpoint{21.127231\du}{16.086490\du}}
\pgfpathcurveto{\pgfpoint{21.338499\du}{15.945949\du}}{\pgfpoint{20.798592\du}{16.767567\du}}{\pgfpoint{21.831458\du}{16.983784\du}}
\pgfpathcurveto{\pgfpoint{22.864314\du}{17.200000\du}}{\pgfpoint{23.392484\du}{16.854054\du}}{\pgfpoint{23.239902\du}{16.605405\du}}
\pgfpathcurveto{\pgfpoint{23.087319\du}{16.356756\du}}{\pgfpoint{24.143660\du}{17.189189\du}}{\pgfpoint{24.636619\du}{16.713513\du}}
\pgfpathcurveto{\pgfpoint{25.129577\du}{16.237837\du}}{\pgfpoint{24.131923\du}{15.783787\du}}{\pgfpoint{24.343191\du}{15.848652\du}}
\pgfpathcurveto{\pgfpoint{24.554459\du}{15.913517\du}}{\pgfpoint{25.200000\du}{15.827030\du}}{\pgfpoint{24.988732\du}{15.016218\du}}
\pgfpathcurveto{\pgfpoint{24.777464\du}{14.205407\du}}{\pgfpoint{22.876051\du}{14.832434\du}}{\pgfpoint{23.087319\du}{14.713515\du}}
\pgfpathcurveto{\pgfpoint{23.298588\du}{14.594596\du}}{\pgfpoint{22.770417\du}{14.000000\du}}{\pgfpoint{22.113148\du}{14.118919\du}}
\pgfpathcurveto{\pgfpoint{21.455870\du}{14.237839\du}}{\pgfpoint{21.409250\du}{14.453634\du}}{\pgfpoint{21.338828\du}{14.929310\du}}
\pgfpathlineto{\pgfpoint{21.338499\du}{14.929732\du}}
\pgfusepath{fill}
\definecolor{dialinecolor}{rgb}{0.000000, 0.000000, 0.000000}
\pgfsetstrokecolor{dialinecolor}
\pgfpathmoveto{\pgfpoint{21.338499\du}{14.929732\du}}
\pgfpathcurveto{\pgfpoint{20.951174\du}{14.918921\du}}{\pgfpoint{20.200000\du}{15.145948\du}}{\pgfpoint{20.305634\du}{15.632435\du}}
\pgfpathcurveto{\pgfpoint{20.411267\du}{16.118922\du}}{\pgfpoint{20.915963\du}{16.227026\du}}{\pgfpoint{21.127231\du}{16.086490\du}}
\pgfpathcurveto{\pgfpoint{21.338499\du}{15.945949\du}}{\pgfpoint{20.798592\du}{16.767567\du}}{\pgfpoint{21.831458\du}{16.983784\du}}
\pgfpathcurveto{\pgfpoint{22.864314\du}{17.200000\du}}{\pgfpoint{23.392484\du}{16.854054\du}}{\pgfpoint{23.239902\du}{16.605405\du}}
\pgfpathcurveto{\pgfpoint{23.087319\du}{16.356756\du}}{\pgfpoint{24.143660\du}{17.189189\du}}{\pgfpoint{24.636619\du}{16.713513\du}}
\pgfpathcurveto{\pgfpoint{25.129577\du}{16.237837\du}}{\pgfpoint{24.131923\du}{15.783787\du}}{\pgfpoint{24.343191\du}{15.848652\du}}
\pgfpathcurveto{\pgfpoint{24.554459\du}{15.913517\du}}{\pgfpoint{25.200000\du}{15.827030\du}}{\pgfpoint{24.988732\du}{15.016218\du}}
\pgfpathcurveto{\pgfpoint{24.777464\du}{14.205407\du}}{\pgfpoint{22.876051\du}{14.832434\du}}{\pgfpoint{23.087319\du}{14.713515\du}}
\pgfpathcurveto{\pgfpoint{23.298588\du}{14.594596\du}}{\pgfpoint{22.770417\du}{14.000000\du}}{\pgfpoint{22.113148\du}{14.118919\du}}
\pgfpathcurveto{\pgfpoint{21.455870\du}{14.237839\du}}{\pgfpoint{21.409250\du}{14.453634\du}}{\pgfpoint{21.338828\du}{14.929310\du}}
\pgfpathlineto{\pgfpoint{21.338499\du}{14.929732\du}}
\pgfusepath{stroke}
% setfont left to latex
\definecolor{dialinecolor}{rgb}{0.000000, 0.000000, 0.000000}
\pgfsetstrokecolor{dialinecolor}
\node at (22.834033\du,15.856409\du){};
\pgfsetlinewidth{0.000000\du}
\pgfsetdash{}{0pt}
\pgfsetdash{}{0pt}
\pgfsetbuttcap
\pgfsetmiterjoin
\pgfsetlinewidth{0.000000\du}
\pgfsetbuttcap
\pgfsetmiterjoin
\pgfsetdash{}{0pt}
\definecolor{dialinecolor}{rgb}{0.000000, 0.000000, 0.000000}
\pgfsetstrokecolor{dialinecolor}
\draw (26.000000\du,13.000000\du)--(27.000000\du,13.000000\du)--(27.000000\du,15.000000\du)--(26.000000\du,15.000000\du)--cycle;
\pgfsetbuttcap
\pgfsetmiterjoin
\pgfsetdash{}{0pt}
\definecolor{dialinecolor}{rgb}{0.000000, 0.000000, 0.000000}
\pgfsetstrokecolor{dialinecolor}
\draw (26.300000\du,15.000000\du)--(26.500000\du,14.600000\du);
\pgfsetbuttcap
\pgfsetmiterjoin
\pgfsetdash{}{0pt}
\definecolor{dialinecolor}{rgb}{0.000000, 0.000000, 0.000000}
\pgfsetstrokecolor{dialinecolor}
\draw (26.500000\du,14.600000\du)--(26.700000\du,15.000000\du);
% setfont left to latex
\definecolor{dialinecolor}{rgb}{0.000000, 0.000000, 0.000000}
\pgfsetstrokecolor{dialinecolor}
\node at (26.500000\du,14.000000\du){};
\pgfsetlinewidth{0.000000\du}
\pgfsetdash{}{0pt}
\pgfsetdash{}{0pt}
\pgfsetmiterjoin
\pgfsetbuttcap
{
\definecolor{dialinecolor}{rgb}{0.000000, 0.000000, 0.000000}
\pgfsetfillcolor{dialinecolor}
% was here!!!
\pgfsetarrowsend{latex}
{\pgfsetcornersarced{\pgfpoint{0.000000\du}{0.000000\du}}\definecolor{dialinecolor}{rgb}{0.000000, 0.000000, 0.000000}
\pgfsetstrokecolor{dialinecolor}
\draw (25.028500\du,15.320300\du)--(25.200000\du,15.320300\du)--(25.200000\du,14.000000\du)--(26.000000\du,14.000000\du);
}}
\pgfsetlinewidth{0.000000\du}
\pgfsetdash{}{0pt}
\pgfsetdash{}{0pt}
\pgfsetmiterjoin
\pgfsetbuttcap
{
\definecolor{dialinecolor}{rgb}{0.000000, 0.000000, 0.000000}
\pgfsetfillcolor{dialinecolor}
% was here!!!
\pgfsetarrowsend{latex}
{\pgfsetcornersarced{\pgfpoint{0.000000\du}{0.000000\du}}\definecolor{dialinecolor}{rgb}{0.000000, 0.000000, 0.000000}
\pgfsetstrokecolor{dialinecolor}
\draw (27.000000\du,14.000000\du)--(27.500000\du,14.000000\du)--(27.500000\du,12.000000\du)--(22.280200\du,12.000000\du)--(22.280200\du,14.104300\du);
}}
\pgfsetlinewidth{0.000000\du}
\pgfsetdash{}{0pt}
\pgfsetdash{}{0pt}
\pgfsetbuttcap
\pgfsetmiterjoin
\pgfsetlinewidth{0.000000\du}
\pgfsetbuttcap
\pgfsetmiterjoin
\pgfsetdash{}{0pt}
\definecolor{dialinecolor}{rgb}{0.000000, 0.000000, 0.000000}
\pgfsetstrokecolor{dialinecolor}
\draw (27.000000\du,19.000000\du)--(29.000000\du,19.000000\du)--(29.000000\du,20.000000\du)--(27.000000\du,20.000000\du)--cycle;
\pgfsetbuttcap
\pgfsetmiterjoin
\pgfsetdash{}{0pt}
\definecolor{dialinecolor}{rgb}{0.000000, 0.000000, 0.000000}
\pgfsetstrokecolor{dialinecolor}
\draw (27.000000\du,19.300000\du)--(27.400000\du,19.500000\du);
\pgfsetbuttcap
\pgfsetmiterjoin
\pgfsetdash{}{0pt}
\definecolor{dialinecolor}{rgb}{0.000000, 0.000000, 0.000000}
\pgfsetstrokecolor{dialinecolor}
\draw (27.400000\du,19.500000\du)--(27.000000\du,19.700000\du);
% setfont left to latex
\definecolor{dialinecolor}{rgb}{0.000000, 0.000000, 0.000000}
\pgfsetstrokecolor{dialinecolor}
\node at (28.200000\du,19.700000\du){};
\pgfsetlinewidth{0.000000\du}
\pgfsetdash{}{0pt}
\pgfsetdash{}{0pt}
\pgfsetbuttcap
{
\definecolor{dialinecolor}{rgb}{0.000000, 0.000000, 0.000000}
\pgfsetfillcolor{dialinecolor}
% was here!!!
\definecolor{dialinecolor}{rgb}{0.000000, 0.000000, 0.000000}
\pgfsetstrokecolor{dialinecolor}
\draw (28.000000\du,20.000000\du)--(28.000000\du,21.000000\du);
}
% setfont left to latex
\definecolor{dialinecolor}{rgb}{0.000000, 0.000000, 0.000000}
\pgfsetstrokecolor{dialinecolor}
\node[anchor=west] at (19.000000\du,9.000000\du){- Round Robin};
% setfont left to latex
\definecolor{dialinecolor}{rgb}{0.000000, 0.000000, 0.000000}
\pgfsetstrokecolor{dialinecolor}
\node[anchor=west] at (19.000000\du,9.800000\du){- Random (LFSR)};
% setfont left to latex
\definecolor{dialinecolor}{rgb}{0.000000, 0.000000, 0.000000}
\pgfsetstrokecolor{dialinecolor}
\node[anchor=west] at (17.000000\du,22.000000\du){- Fixed};
% setfont left to latex
\definecolor{dialinecolor}{rgb}{0.000000, 0.000000, 0.000000}
\pgfsetstrokecolor{dialinecolor}
\node[anchor=west] at (17.000000\du,22.800000\du){- Random (LFSR)};
% setfont left to latex
\definecolor{dialinecolor}{rgb}{0.000000, 0.000000, 0.000000}
\pgfsetstrokecolor{dialinecolor}
\node[anchor=west] at (18.000000\du,4.000000\du){- Read};
% setfont left to latex
\definecolor{dialinecolor}{rgb}{0.000000, 0.000000, 0.000000}
\pgfsetstrokecolor{dialinecolor}
\node[anchor=west] at (18.000000\du,4.800000\du){- Write};
% setfont left to latex
\definecolor{dialinecolor}{rgb}{0.000000, 0.000000, 0.000000}
\pgfsetstrokecolor{dialinecolor}
\node[anchor=west] at (18.000000\du,5.600000\du){- Interrupt};
\pgfsetlinewidth{0.000000\du}
\pgfsetdash{}{0pt}
\pgfsetdash{}{0pt}
\pgfsetmiterjoin
\pgfsetbuttcap
{
\definecolor{dialinecolor}{rgb}{0.000000, 0.000000, 0.000000}
\pgfsetfillcolor{dialinecolor}
% was here!!!
\pgfsetarrowsend{latex}
{\pgfsetcornersarced{\pgfpoint{0.000000\du}{0.000000\du}}\definecolor{dialinecolor}{rgb}{0.000000, 0.000000, 0.000000}
\pgfsetstrokecolor{dialinecolor}
\draw (40.000000\du,8.400000\du)--(40.000000\du,14.700000\du)--(41.000000\du,14.700000\du)--(41.000000\du,21.000000\du);
}}
\pgfsetlinewidth{0.000000\du}
\pgfsetdash{}{0pt}
\pgfsetdash{}{0pt}
\pgfsetmiterjoin
\pgfsetbuttcap
{
\definecolor{dialinecolor}{rgb}{0.000000, 0.000000, 0.000000}
\pgfsetfillcolor{dialinecolor}
% was here!!!
\pgfsetarrowsend{latex}
{\pgfsetcornersarced{\pgfpoint{0.000000\du}{0.000000\du}}\definecolor{dialinecolor}{rgb}{0.000000, 0.000000, 0.000000}
\pgfsetstrokecolor{dialinecolor}
\draw (40.000000\du,8.400000\du)--(40.000000\du,13.000000\du)--(41.000000\du,13.000000\du)--(41.000000\du,13.000000\du);
}}
% setfont left to latex
\definecolor{dialinecolor}{rgb}{0.000000, 0.000000, 0.000000}
\pgfsetstrokecolor{dialinecolor}
\node[anchor=west] at (41.000000\du,12.811906\du){ To Interrupt};
% setfont left to latex
\definecolor{dialinecolor}{rgb}{0.000000, 0.000000, 0.000000}
\pgfsetstrokecolor{dialinecolor}
\node[anchor=west] at (41.000000\du,13.611906\du){    Handling};
\end{tikzpicture}

  \end{center}
  \caption[Overview of fbist\_cgen.vhdl]{\label{fig:fbist_cgen}Overview of fbist\_cgen.vhdl
  }
\end{figure}

\subsection{Address Generator}
The Address Generator, shown in Figure~\ref{fig:fbist_agen} takes as
inputs the command and the generating engine, as well as a valid, and
generates an address and tag. The tag is used eventually as part of
the AXI tag and CAPPTag, as well as being used as a key in the LUT to
check results. Each engine in the command generator has a
corresponding engine in the address generator, each with a unique
configuration and state.

Each address generator is programmed with a single address type that
is static during test execution. This is related 1:1 with a command
generator. An address can either be pseudo-random, or incrementing by
a configurable constant. Additionally, the address can be ANDed or
ORed with a programmable mask. This allows, for example, different
commands to be performed in different areas of the memory, to prohibit
address collisions, or hit a fixed address. Although OCMB can handle
address collisions, the programming of these registers, especially in
relation to each other, affects how often this happens. The 8 engines
in a pool also use a common AND and OR mask, to allow configuration on
a pool basis.

The address generator also produces a tag, to be used as an
identifying tag in the AXI transaction as well as the OpenCAPI
transaction. This is arbitrary in both the FPGA and OCMB, and the
actual value is not significant, serving only as an accounting
method. Minimal configuration is provided for this. Each command that
passes through the address generator increments the tag by 1. Each
pool has a different mask that enables 1/4 of the total amount of tags
available to FBIST. Once the counter is maxed out, it overflows back
to 0. While it is possible that a tag can be reused while an operation
to that tag is still pending, it is very unlikely this would happen.

\begin{figure}[h]
  \begin{center}
    % Graphic for TeX using PGF
% Title: /home/rpking42/proj/chips/astra/doc/images/fbist_agen.dia
% Creator: Dia v0.97.3
% CreationDate: Mon Oct 15 15:46:29 2018
% For: rpking42
% \usepackage{tikz}
% The following commands are not supported in PSTricks at present
% We define them conditionally, so when they are implemented,
% this pgf file will use them.
\ifx\du\undefined
  \newlength{\du}
\fi
\setlength{\du}{15\unitlength}
\begin{tikzpicture}
\pgftransformxscale{1.000000}
\pgftransformyscale{-1.000000}
\definecolor{dialinecolor}{rgb}{0.000000, 0.000000, 0.000000}
\pgfsetstrokecolor{dialinecolor}
\definecolor{dialinecolor}{rgb}{1.000000, 1.000000, 1.000000}
\pgfsetfillcolor{dialinecolor}
\pgfsetlinewidth{0.000000\du}
\pgfsetdash{}{0pt}
\pgfsetdash{}{0pt}
\pgfsetbuttcap
\pgfsetmiterjoin
\pgfsetlinewidth{0.000000\du}
\pgfsetbuttcap
\pgfsetmiterjoin
\pgfsetdash{}{0pt}
\definecolor{dialinecolor}{rgb}{0.000000, 0.000000, 0.000000}
\pgfsetstrokecolor{dialinecolor}
\draw (17.000000\du,3.000000\du)--(24.000000\du,3.000000\du)--(24.000000\du,5.000000\du)--(17.000000\du,5.000000\du)--cycle;
\pgfsetbuttcap
\pgfsetmiterjoin
\pgfsetdash{}{0pt}
\definecolor{dialinecolor}{rgb}{0.000000, 0.000000, 0.000000}
\pgfsetstrokecolor{dialinecolor}
\draw (17.000000\du,3.600000\du)--(18.400000\du,4.000000\du);
\pgfsetbuttcap
\pgfsetmiterjoin
\pgfsetdash{}{0pt}
\definecolor{dialinecolor}{rgb}{0.000000, 0.000000, 0.000000}
\pgfsetstrokecolor{dialinecolor}
\draw (18.400000\du,4.000000\du)--(17.000000\du,4.400000\du);
% setfont left to latex
\definecolor{dialinecolor}{rgb}{0.000000, 0.000000, 0.000000}
\pgfsetstrokecolor{dialinecolor}
\node at (21.200000\du,4.200000\du){CFG\_ADDRESS};
\pgfsetlinewidth{0.000000\du}
\pgfsetdash{}{0pt}
\pgfsetdash{}{0pt}
\pgfsetbuttcap
\pgfsetmiterjoin
\pgfsetlinewidth{0.000000\du}
\pgfsetbuttcap
\pgfsetmiterjoin
\pgfsetdash{}{0pt}
\definecolor{dialinecolor}{rgb}{0.000000, 0.000000, 0.000000}
\pgfsetstrokecolor{dialinecolor}
\draw (17.000000\du,9.000000\du)--(24.000000\du,9.000000\du)--(24.000000\du,11.000000\du)--(17.000000\du,11.000000\du)--cycle;
\pgfsetbuttcap
\pgfsetmiterjoin
\pgfsetdash{}{0pt}
\definecolor{dialinecolor}{rgb}{0.000000, 0.000000, 0.000000}
\pgfsetstrokecolor{dialinecolor}
\draw (17.000000\du,9.600000\du)--(18.400000\du,10.000000\du);
\pgfsetbuttcap
\pgfsetmiterjoin
\pgfsetdash{}{0pt}
\definecolor{dialinecolor}{rgb}{0.000000, 0.000000, 0.000000}
\pgfsetstrokecolor{dialinecolor}
\draw (18.400000\du,10.000000\du)--(17.000000\du,10.400000\du);
% setfont left to latex
\definecolor{dialinecolor}{rgb}{0.000000, 0.000000, 0.000000}
\pgfsetstrokecolor{dialinecolor}
\node at (21.200000\du,10.200000\du){CFG\_ADDRESS};
% setfont left to latex
\definecolor{dialinecolor}{rgb}{0.000000, 0.000000, 0.000000}
\pgfsetstrokecolor{dialinecolor}
\node[anchor=east] at (17.000000\du,3.821250\du){Engine };
% setfont left to latex
\definecolor{dialinecolor}{rgb}{0.000000, 0.000000, 0.000000}
\pgfsetstrokecolor{dialinecolor}
\node[anchor=east] at (17.000000\du,4.621250\du){0 };
\pgfsetlinewidth{0.000000\du}
\pgfsetdash{}{0pt}
\pgfsetdash{}{0pt}
\pgfsetbuttcap
\pgfsetmiterjoin
\pgfsetlinewidth{0.000000\du}
\pgfsetbuttcap
\pgfsetmiterjoin
\pgfsetdash{}{0pt}
\definecolor{dialinecolor}{rgb}{0.000000, 0.000000, 0.000000}
\pgfsetstrokecolor{dialinecolor}
\draw (39.000000\du,0.000000\du)--(45.000000\du,0.000000\du)--(45.000000\du,2.000000\du)--(39.000000\du,2.000000\du)--cycle;
\pgfsetbuttcap
\pgfsetmiterjoin
\pgfsetdash{}{0pt}
\definecolor{dialinecolor}{rgb}{0.000000, 0.000000, 0.000000}
\pgfsetstrokecolor{dialinecolor}
\draw (39.000000\du,0.600000\du)--(40.200000\du,1.000000\du);
\pgfsetbuttcap
\pgfsetmiterjoin
\pgfsetdash{}{0pt}
\definecolor{dialinecolor}{rgb}{0.000000, 0.000000, 0.000000}
\pgfsetstrokecolor{dialinecolor}
\draw (40.200000\du,1.000000\du)--(39.000000\du,1.400000\du);
% setfont left to latex
\definecolor{dialinecolor}{rgb}{0.000000, 0.000000, 0.000000}
\pgfsetstrokecolor{dialinecolor}
\node at (42.600000\du,1.200000\du){command};
\pgfsetlinewidth{0.000000\du}
\pgfsetdash{}{0pt}
\pgfsetdash{}{0pt}
\pgfsetbuttcap
\pgfsetmiterjoin
\pgfsetlinewidth{0.000000\du}
\pgfsetbuttcap
\pgfsetmiterjoin
\pgfsetdash{}{0pt}
\definecolor{dialinecolor}{rgb}{0.000000, 0.000000, 0.000000}
\pgfsetstrokecolor{dialinecolor}
\draw (33.000000\du,0.000000\du)--(37.000000\du,0.000000\du)--(37.000000\du,2.000000\du)--(33.000000\du,2.000000\du)--cycle;
\pgfsetbuttcap
\pgfsetmiterjoin
\pgfsetdash{}{0pt}
\definecolor{dialinecolor}{rgb}{0.000000, 0.000000, 0.000000}
\pgfsetstrokecolor{dialinecolor}
\draw (33.000000\du,0.600000\du)--(33.800000\du,1.000000\du);
\pgfsetbuttcap
\pgfsetmiterjoin
\pgfsetdash{}{0pt}
\definecolor{dialinecolor}{rgb}{0.000000, 0.000000, 0.000000}
\pgfsetstrokecolor{dialinecolor}
\draw (33.800000\du,1.000000\du)--(33.000000\du,1.400000\du);
% setfont left to latex
\definecolor{dialinecolor}{rgb}{0.000000, 0.000000, 0.000000}
\pgfsetstrokecolor{dialinecolor}
\node at (35.400000\du,1.200000\du){engine};
\pgfsetlinewidth{0.000000\du}
\pgfsetdash{}{0pt}
\pgfsetdash{}{0pt}
\pgfsetbuttcap
\pgfsetmiterjoin
\pgfsetlinewidth{0.000000\du}
\pgfsetbuttcap
\pgfsetmiterjoin
\pgfsetdash{}{0pt}
\definecolor{dialinecolor}{rgb}{0.000000, 0.000000, 0.000000}
\pgfsetstrokecolor{dialinecolor}
\draw (27.000000\du,0.000000\du)--(31.000000\du,0.000000\du)--(31.000000\du,2.000000\du)--(27.000000\du,2.000000\du)--cycle;
\pgfsetbuttcap
\pgfsetmiterjoin
\pgfsetdash{}{0pt}
\definecolor{dialinecolor}{rgb}{0.000000, 0.000000, 0.000000}
\pgfsetstrokecolor{dialinecolor}
\draw (27.000000\du,0.600000\du)--(27.800000\du,1.000000\du);
\pgfsetbuttcap
\pgfsetmiterjoin
\pgfsetdash{}{0pt}
\definecolor{dialinecolor}{rgb}{0.000000, 0.000000, 0.000000}
\pgfsetstrokecolor{dialinecolor}
\draw (27.800000\du,1.000000\du)--(27.000000\du,1.400000\du);
% setfont left to latex
\definecolor{dialinecolor}{rgb}{0.000000, 0.000000, 0.000000}
\pgfsetstrokecolor{dialinecolor}
\node at (29.400000\du,1.200000\du){valid};
\pgfsetlinewidth{0.000000\du}
\pgfsetdash{}{0pt}
\pgfsetdash{}{0pt}
\pgfsetbuttcap
{
\definecolor{dialinecolor}{rgb}{0.000000, 0.000000, 0.000000}
\pgfsetfillcolor{dialinecolor}
% was here!!!
\pgfsetarrowsend{latex}
\definecolor{dialinecolor}{rgb}{0.000000, 0.000000, 0.000000}
\pgfsetstrokecolor{dialinecolor}
\draw (42.000000\du,-2.000000\du)--(42.000000\du,0.000000\du);
}
\pgfsetlinewidth{0.000000\du}
\pgfsetdash{}{0pt}
\pgfsetdash{}{0pt}
\pgfsetbuttcap
{
\definecolor{dialinecolor}{rgb}{0.000000, 0.000000, 0.000000}
\pgfsetfillcolor{dialinecolor}
% was here!!!
\pgfsetarrowsend{latex}
\definecolor{dialinecolor}{rgb}{0.000000, 0.000000, 0.000000}
\pgfsetstrokecolor{dialinecolor}
\draw (35.000000\du,-2.000000\du)--(35.000000\du,0.000000\du);
}
\pgfsetlinewidth{0.000000\du}
\pgfsetdash{}{0pt}
\pgfsetdash{}{0pt}
\pgfsetbuttcap
{
\definecolor{dialinecolor}{rgb}{0.000000, 0.000000, 0.000000}
\pgfsetfillcolor{dialinecolor}
% was here!!!
\pgfsetarrowsend{latex}
\definecolor{dialinecolor}{rgb}{0.000000, 0.000000, 0.000000}
\pgfsetstrokecolor{dialinecolor}
\draw (29.000000\du,-2.000000\du)--(29.000000\du,0.000000\du);
}
\pgfsetlinewidth{0.000000\du}
\pgfsetdash{}{0pt}
\pgfsetdash{}{0pt}
\pgfsetbuttcap
\pgfsetmiterjoin
\pgfsetlinewidth{0.000000\du}
\pgfsetbuttcap
\pgfsetmiterjoin
\pgfsetdash{}{0pt}
\definecolor{dialinecolor}{rgb}{0.000000, 0.000000, 0.000000}
\pgfsetstrokecolor{dialinecolor}
\draw (17.000000\du,6.000000\du)--(24.000000\du,6.000000\du)--(24.000000\du,8.000000\du)--(17.000000\du,8.000000\du)--cycle;
\pgfsetbuttcap
\pgfsetmiterjoin
\pgfsetdash{}{0pt}
\definecolor{dialinecolor}{rgb}{0.000000, 0.000000, 0.000000}
\pgfsetstrokecolor{dialinecolor}
\draw (17.000000\du,6.600000\du)--(18.400000\du,7.000000\du);
\pgfsetbuttcap
\pgfsetmiterjoin
\pgfsetdash{}{0pt}
\definecolor{dialinecolor}{rgb}{0.000000, 0.000000, 0.000000}
\pgfsetstrokecolor{dialinecolor}
\draw (18.400000\du,7.000000\du)--(17.000000\du,7.400000\du);
% setfont left to latex
\definecolor{dialinecolor}{rgb}{0.000000, 0.000000, 0.000000}
\pgfsetstrokecolor{dialinecolor}
\node at (21.200000\du,7.200000\du){CFG\_ADDRESS};
\pgfsetlinewidth{0.000000\du}
\pgfsetdash{}{0pt}
\pgfsetdash{}{0pt}
\pgfsetbuttcap
\pgfsetmiterjoin
\pgfsetlinewidth{0.000000\du}
\pgfsetbuttcap
\pgfsetmiterjoin
\pgfsetdash{}{0pt}
\definecolor{dialinecolor}{rgb}{0.000000, 0.000000, 0.000000}
\pgfsetstrokecolor{dialinecolor}
\draw (17.000000\du,12.000000\du)--(24.000000\du,12.000000\du)--(24.000000\du,14.000000\du)--(17.000000\du,14.000000\du)--cycle;
\pgfsetbuttcap
\pgfsetmiterjoin
\pgfsetdash{}{0pt}
\definecolor{dialinecolor}{rgb}{0.000000, 0.000000, 0.000000}
\pgfsetstrokecolor{dialinecolor}
\draw (17.000000\du,12.600000\du)--(18.400000\du,13.000000\du);
\pgfsetbuttcap
\pgfsetmiterjoin
\pgfsetdash{}{0pt}
\definecolor{dialinecolor}{rgb}{0.000000, 0.000000, 0.000000}
\pgfsetstrokecolor{dialinecolor}
\draw (18.400000\du,13.000000\du)--(17.000000\du,13.400000\du);
% setfont left to latex
\definecolor{dialinecolor}{rgb}{0.000000, 0.000000, 0.000000}
\pgfsetstrokecolor{dialinecolor}
\node at (21.200000\du,13.200000\du){CFG\_ADDRESS};
\pgfsetlinewidth{0.000000\du}
\pgfsetdash{}{0pt}
\pgfsetdash{}{0pt}
\pgfsetbuttcap
\pgfsetmiterjoin
\pgfsetlinewidth{0.000000\du}
\pgfsetbuttcap
\pgfsetmiterjoin
\pgfsetdash{}{0pt}
\definecolor{dialinecolor}{rgb}{0.000000, 0.000000, 0.000000}
\pgfsetstrokecolor{dialinecolor}
\draw (17.000000\du,15.000000\du)--(24.000000\du,15.000000\du)--(24.000000\du,17.000000\du)--(17.000000\du,17.000000\du)--cycle;
\pgfsetbuttcap
\pgfsetmiterjoin
\pgfsetdash{}{0pt}
\definecolor{dialinecolor}{rgb}{0.000000, 0.000000, 0.000000}
\pgfsetstrokecolor{dialinecolor}
\draw (17.000000\du,15.600000\du)--(18.400000\du,16.000000\du);
\pgfsetbuttcap
\pgfsetmiterjoin
\pgfsetdash{}{0pt}
\definecolor{dialinecolor}{rgb}{0.000000, 0.000000, 0.000000}
\pgfsetstrokecolor{dialinecolor}
\draw (18.400000\du,16.000000\du)--(17.000000\du,16.400000\du);
% setfont left to latex
\definecolor{dialinecolor}{rgb}{0.000000, 0.000000, 0.000000}
\pgfsetstrokecolor{dialinecolor}
\node at (21.200000\du,16.200000\du){CFG\_ADDRESS};
\pgfsetlinewidth{0.000000\du}
\pgfsetdash{}{0pt}
\pgfsetdash{}{0pt}
\pgfsetbuttcap
\pgfsetmiterjoin
\pgfsetlinewidth{0.000000\du}
\pgfsetbuttcap
\pgfsetmiterjoin
\pgfsetdash{}{0pt}
\definecolor{dialinecolor}{rgb}{0.000000, 0.000000, 0.000000}
\pgfsetstrokecolor{dialinecolor}
\draw (17.000000\du,18.000000\du)--(24.000000\du,18.000000\du)--(24.000000\du,20.000000\du)--(17.000000\du,20.000000\du)--cycle;
\pgfsetbuttcap
\pgfsetmiterjoin
\pgfsetdash{}{0pt}
\definecolor{dialinecolor}{rgb}{0.000000, 0.000000, 0.000000}
\pgfsetstrokecolor{dialinecolor}
\draw (17.000000\du,18.600000\du)--(18.400000\du,19.000000\du);
\pgfsetbuttcap
\pgfsetmiterjoin
\pgfsetdash{}{0pt}
\definecolor{dialinecolor}{rgb}{0.000000, 0.000000, 0.000000}
\pgfsetstrokecolor{dialinecolor}
\draw (18.400000\du,19.000000\du)--(17.000000\du,19.400000\du);
% setfont left to latex
\definecolor{dialinecolor}{rgb}{0.000000, 0.000000, 0.000000}
\pgfsetstrokecolor{dialinecolor}
\node at (21.200000\du,19.200000\du){CFG\_ADDRESS};
\pgfsetlinewidth{0.000000\du}
\pgfsetdash{}{0pt}
\pgfsetdash{}{0pt}
\pgfsetbuttcap
\pgfsetmiterjoin
\pgfsetlinewidth{0.000000\du}
\pgfsetbuttcap
\pgfsetmiterjoin
\pgfsetdash{}{0pt}
\definecolor{dialinecolor}{rgb}{0.000000, 0.000000, 0.000000}
\pgfsetstrokecolor{dialinecolor}
\draw (17.000000\du,21.000000\du)--(24.000000\du,21.000000\du)--(24.000000\du,23.000000\du)--(17.000000\du,23.000000\du)--cycle;
\pgfsetbuttcap
\pgfsetmiterjoin
\pgfsetdash{}{0pt}
\definecolor{dialinecolor}{rgb}{0.000000, 0.000000, 0.000000}
\pgfsetstrokecolor{dialinecolor}
\draw (17.000000\du,21.600000\du)--(18.400000\du,22.000000\du);
\pgfsetbuttcap
\pgfsetmiterjoin
\pgfsetdash{}{0pt}
\definecolor{dialinecolor}{rgb}{0.000000, 0.000000, 0.000000}
\pgfsetstrokecolor{dialinecolor}
\draw (18.400000\du,22.000000\du)--(17.000000\du,22.400000\du);
% setfont left to latex
\definecolor{dialinecolor}{rgb}{0.000000, 0.000000, 0.000000}
\pgfsetstrokecolor{dialinecolor}
\node at (21.200000\du,22.200000\du){CFG\_ADDRESS};
\pgfsetlinewidth{0.000000\du}
\pgfsetdash{}{0pt}
\pgfsetdash{}{0pt}
\pgfsetbuttcap
\pgfsetmiterjoin
\pgfsetlinewidth{0.000000\du}
\pgfsetbuttcap
\pgfsetmiterjoin
\pgfsetdash{}{0pt}
\definecolor{dialinecolor}{rgb}{0.000000, 0.000000, 0.000000}
\pgfsetstrokecolor{dialinecolor}
\draw (17.000000\du,24.000000\du)--(24.000000\du,24.000000\du)--(24.000000\du,26.000000\du)--(17.000000\du,26.000000\du)--cycle;
\pgfsetbuttcap
\pgfsetmiterjoin
\pgfsetdash{}{0pt}
\definecolor{dialinecolor}{rgb}{0.000000, 0.000000, 0.000000}
\pgfsetstrokecolor{dialinecolor}
\draw (17.000000\du,24.600000\du)--(18.400000\du,25.000000\du);
\pgfsetbuttcap
\pgfsetmiterjoin
\pgfsetdash{}{0pt}
\definecolor{dialinecolor}{rgb}{0.000000, 0.000000, 0.000000}
\pgfsetstrokecolor{dialinecolor}
\draw (18.400000\du,25.000000\du)--(17.000000\du,25.400000\du);
% setfont left to latex
\definecolor{dialinecolor}{rgb}{0.000000, 0.000000, 0.000000}
\pgfsetstrokecolor{dialinecolor}
\node at (21.200000\du,25.200000\du){CFG\_ADDRESS};
\pgfsetlinewidth{0.000000\du}
\pgfsetdash{}{0pt}
\pgfsetdash{}{0pt}
\pgfsetbuttcap
\pgfsetmiterjoin
\pgfsetlinewidth{0.000000\du}
\pgfsetbuttcap
\pgfsetmiterjoin
\pgfsetdash{}{0pt}
\definecolor{dialinecolor}{rgb}{0.000000, 0.000000, 0.000000}
\pgfsetstrokecolor{dialinecolor}
\draw (27.000000\du,12.000000\du)--(29.000000\du,13.250000\du)--(29.000000\du,15.750000\du)--(27.000000\du,17.000000\du)--cycle;
% setfont left to latex
\definecolor{dialinecolor}{rgb}{0.000000, 0.000000, 0.000000}
\pgfsetstrokecolor{dialinecolor}
\node at (28.000000\du,14.700000\du){};
\pgfsetlinewidth{0.000000\du}
\pgfsetdash{}{0pt}
\pgfsetdash{}{0pt}
\pgfsetbuttcap
{
\definecolor{dialinecolor}{rgb}{0.000000, 0.000000, 0.000000}
\pgfsetfillcolor{dialinecolor}
% was here!!!
\definecolor{dialinecolor}{rgb}{0.000000, 0.000000, 0.000000}
\pgfsetstrokecolor{dialinecolor}
\draw (27.000000\du,13.250000\du)--(27.000000\du,15.750000\du);
}
\pgfsetlinewidth{0.000000\du}
\pgfsetdash{}{0pt}
\pgfsetdash{}{0pt}
\pgfsetmiterjoin
\pgfsetbuttcap
{
\definecolor{dialinecolor}{rgb}{0.000000, 0.000000, 0.000000}
\pgfsetfillcolor{dialinecolor}
% was here!!!
{\pgfsetcornersarced{\pgfpoint{0.000000\du}{0.000000\du}}\definecolor{dialinecolor}{rgb}{0.000000, 0.000000, 0.000000}
\pgfsetstrokecolor{dialinecolor}
\draw (20.500000\du,5.000000\du)--(20.500000\du,5.500000\du)--(25.500000\du,5.500000\du)--(25.500000\du,13.250000\du)--(27.000000\du,13.250000\du);
}}
\pgfsetlinewidth{0.000000\du}
\pgfsetdash{}{0pt}
\pgfsetdash{}{0pt}
\pgfsetmiterjoin
\pgfsetbuttcap
{
\definecolor{dialinecolor}{rgb}{0.000000, 0.000000, 0.000000}
\pgfsetfillcolor{dialinecolor}
% was here!!!
{\pgfsetcornersarced{\pgfpoint{0.000000\du}{0.000000\du}}\definecolor{dialinecolor}{rgb}{0.000000, 0.000000, 0.000000}
\pgfsetstrokecolor{dialinecolor}
\draw (20.500000\du,8.000000\du)--(20.500000\du,8.500000\du)--(25.000000\du,8.500000\du)--(25.000000\du,13.607100\du)--(27.000000\du,13.607100\du);
}}
\pgfsetlinewidth{0.000000\du}
\pgfsetdash{}{0pt}
\pgfsetdash{}{0pt}
\pgfsetmiterjoin
\pgfsetbuttcap
{
\definecolor{dialinecolor}{rgb}{0.000000, 0.000000, 0.000000}
\pgfsetfillcolor{dialinecolor}
% was here!!!
{\pgfsetcornersarced{\pgfpoint{0.000000\du}{0.000000\du}}\definecolor{dialinecolor}{rgb}{0.000000, 0.000000, 0.000000}
\pgfsetstrokecolor{dialinecolor}
\draw (20.500000\du,11.000000\du)--(20.500000\du,11.500000\du)--(24.500000\du,11.500000\du)--(24.500000\du,13.964300\du)--(27.000000\du,13.964300\du);
}}
\pgfsetlinewidth{0.000000\du}
\pgfsetdash{}{0pt}
\pgfsetdash{}{0pt}
\pgfsetmiterjoin
\pgfsetbuttcap
{
\definecolor{dialinecolor}{rgb}{0.000000, 0.000000, 0.000000}
\pgfsetfillcolor{dialinecolor}
% was here!!!
{\pgfsetcornersarced{\pgfpoint{0.000000\du}{0.000000\du}}\definecolor{dialinecolor}{rgb}{0.000000, 0.000000, 0.000000}
\pgfsetstrokecolor{dialinecolor}
\draw (20.500000\du,14.000000\du)--(20.500000\du,14.500000\du)--(24.000000\du,14.500000\du)--(24.000000\du,14.321400\du)--(27.000000\du,14.321400\du);
}}
\pgfsetlinewidth{0.000000\du}
\pgfsetdash{}{0pt}
\pgfsetdash{}{0pt}
\pgfsetmiterjoin
\pgfsetbuttcap
{
\definecolor{dialinecolor}{rgb}{0.000000, 0.000000, 0.000000}
\pgfsetfillcolor{dialinecolor}
% was here!!!
{\pgfsetcornersarced{\pgfpoint{0.000000\du}{0.000000\du}}\definecolor{dialinecolor}{rgb}{0.000000, 0.000000, 0.000000}
\pgfsetstrokecolor{dialinecolor}
\draw (20.500000\du,17.000000\du)--(20.500000\du,17.500000\du)--(24.500000\du,17.500000\du)--(24.500000\du,14.678600\du)--(27.000000\du,14.678600\du);
}}
\pgfsetlinewidth{0.000000\du}
\pgfsetdash{}{0pt}
\pgfsetdash{}{0pt}
\pgfsetmiterjoin
\pgfsetbuttcap
{
\definecolor{dialinecolor}{rgb}{0.000000, 0.000000, 0.000000}
\pgfsetfillcolor{dialinecolor}
% was here!!!
{\pgfsetcornersarced{\pgfpoint{0.000000\du}{0.000000\du}}\definecolor{dialinecolor}{rgb}{0.000000, 0.000000, 0.000000}
\pgfsetstrokecolor{dialinecolor}
\draw (20.500000\du,20.000000\du)--(20.500000\du,20.500000\du)--(25.000000\du,20.500000\du)--(25.000000\du,15.035700\du)--(27.000000\du,15.035700\du);
}}
\pgfsetlinewidth{0.000000\du}
\pgfsetdash{}{0pt}
\pgfsetdash{}{0pt}
\pgfsetmiterjoin
\pgfsetbuttcap
{
\definecolor{dialinecolor}{rgb}{0.000000, 0.000000, 0.000000}
\pgfsetfillcolor{dialinecolor}
% was here!!!
{\pgfsetcornersarced{\pgfpoint{0.000000\du}{0.000000\du}}\definecolor{dialinecolor}{rgb}{0.000000, 0.000000, 0.000000}
\pgfsetstrokecolor{dialinecolor}
\draw (20.500000\du,23.000000\du)--(20.500000\du,23.500000\du)--(25.500000\du,23.500000\du)--(25.500000\du,15.392900\du)--(27.000000\du,15.392900\du);
}}
\pgfsetlinewidth{0.000000\du}
\pgfsetdash{}{0pt}
\pgfsetdash{}{0pt}
\pgfsetmiterjoin
\pgfsetbuttcap
{
\definecolor{dialinecolor}{rgb}{0.000000, 0.000000, 0.000000}
\pgfsetfillcolor{dialinecolor}
% was here!!!
{\pgfsetcornersarced{\pgfpoint{0.000000\du}{0.000000\du}}\definecolor{dialinecolor}{rgb}{0.000000, 0.000000, 0.000000}
\pgfsetstrokecolor{dialinecolor}
\draw (20.500000\du,26.000000\du)--(20.500000\du,26.500000\du)--(26.000000\du,26.500000\du)--(26.000000\du,15.750000\du)--(27.000000\du,15.750000\du);
}}
% setfont left to latex
\definecolor{dialinecolor}{rgb}{0.000000, 0.000000, 0.000000}
\pgfsetstrokecolor{dialinecolor}
\node[anchor=east] at (17.000000\du,6.821250\du){Engine };
% setfont left to latex
\definecolor{dialinecolor}{rgb}{0.000000, 0.000000, 0.000000}
\pgfsetstrokecolor{dialinecolor}
\node[anchor=east] at (17.000000\du,7.621250\du){1 };
% setfont left to latex
\definecolor{dialinecolor}{rgb}{0.000000, 0.000000, 0.000000}
\pgfsetstrokecolor{dialinecolor}
\node[anchor=east] at (17.000000\du,9.821250\du){Engine };
% setfont left to latex
\definecolor{dialinecolor}{rgb}{0.000000, 0.000000, 0.000000}
\pgfsetstrokecolor{dialinecolor}
\node[anchor=east] at (17.000000\du,10.621250\du){2 };
% setfont left to latex
\definecolor{dialinecolor}{rgb}{0.000000, 0.000000, 0.000000}
\pgfsetstrokecolor{dialinecolor}
\node[anchor=east] at (17.000000\du,12.821250\du){Engine };
% setfont left to latex
\definecolor{dialinecolor}{rgb}{0.000000, 0.000000, 0.000000}
\pgfsetstrokecolor{dialinecolor}
\node[anchor=east] at (17.000000\du,13.621250\du){3 };
% setfont left to latex
\definecolor{dialinecolor}{rgb}{0.000000, 0.000000, 0.000000}
\pgfsetstrokecolor{dialinecolor}
\node[anchor=east] at (17.000000\du,15.821250\du){Engine };
% setfont left to latex
\definecolor{dialinecolor}{rgb}{0.000000, 0.000000, 0.000000}
\pgfsetstrokecolor{dialinecolor}
\node[anchor=east] at (17.000000\du,16.621250\du){4 };
% setfont left to latex
\definecolor{dialinecolor}{rgb}{0.000000, 0.000000, 0.000000}
\pgfsetstrokecolor{dialinecolor}
\node[anchor=east] at (17.000000\du,18.821250\du){Engine };
% setfont left to latex
\definecolor{dialinecolor}{rgb}{0.000000, 0.000000, 0.000000}
\pgfsetstrokecolor{dialinecolor}
\node[anchor=east] at (17.000000\du,19.621250\du){5 };
% setfont left to latex
\definecolor{dialinecolor}{rgb}{0.000000, 0.000000, 0.000000}
\pgfsetstrokecolor{dialinecolor}
\node[anchor=east] at (17.000000\du,21.821250\du){Engine };
% setfont left to latex
\definecolor{dialinecolor}{rgb}{0.000000, 0.000000, 0.000000}
\pgfsetstrokecolor{dialinecolor}
\node[anchor=east] at (17.000000\du,22.621250\du){6 };
% setfont left to latex
\definecolor{dialinecolor}{rgb}{0.000000, 0.000000, 0.000000}
\pgfsetstrokecolor{dialinecolor}
\node[anchor=east] at (17.000000\du,24.821250\du){Engine };
% setfont left to latex
\definecolor{dialinecolor}{rgb}{0.000000, 0.000000, 0.000000}
\pgfsetstrokecolor{dialinecolor}
\node[anchor=east] at (17.000000\du,25.621250\du){7 };
\pgfsetlinewidth{0.000000\du}
\pgfsetdash{}{0pt}
\pgfsetdash{}{0pt}
\pgfsetmiterjoin
\pgfsetbuttcap
{
\definecolor{dialinecolor}{rgb}{0.000000, 0.000000, 0.000000}
\pgfsetfillcolor{dialinecolor}
% was here!!!
{\pgfsetcornersarced{\pgfpoint{0.000000\du}{0.000000\du}}\definecolor{dialinecolor}{rgb}{0.000000, 0.000000, 0.000000}
\pgfsetstrokecolor{dialinecolor}
\draw (35.000000\du,2.000000\du)--(35.000000\du,5.000000\du)--(28.000000\du,5.000000\du)--(28.000000\du,12.625000\du);
}}
\pgfsetlinewidth{0.000000\du}
\pgfsetdash{}{0pt}
\pgfsetdash{}{0pt}
\pgfsetbuttcap
\pgfsetmiterjoin
\pgfsetlinewidth{0.000000\du}
\pgfsetbuttcap
\pgfsetmiterjoin
\pgfsetdash{}{0pt}
\definecolor{dialinecolor}{rgb}{0.000000, 0.000000, 0.000000}
\pgfsetstrokecolor{dialinecolor}
\draw (39.000000\du,33.000000\du)--(45.000000\du,33.000000\du)--(45.000000\du,35.000000\du)--(39.000000\du,35.000000\du)--cycle;
\pgfsetbuttcap
\pgfsetmiterjoin
\pgfsetdash{}{0pt}
\definecolor{dialinecolor}{rgb}{0.000000, 0.000000, 0.000000}
\pgfsetstrokecolor{dialinecolor}
\draw (39.000000\du,33.600000\du)--(40.200000\du,34.000000\du);
\pgfsetbuttcap
\pgfsetmiterjoin
\pgfsetdash{}{0pt}
\definecolor{dialinecolor}{rgb}{0.000000, 0.000000, 0.000000}
\pgfsetstrokecolor{dialinecolor}
\draw (40.200000\du,34.000000\du)--(39.000000\du,34.400000\du);
% setfont left to latex
\definecolor{dialinecolor}{rgb}{0.000000, 0.000000, 0.000000}
\pgfsetstrokecolor{dialinecolor}
\node at (42.600000\du,34.200000\du){command};
\pgfsetlinewidth{0.000000\du}
\pgfsetdash{}{0pt}
\pgfsetdash{}{0pt}
\pgfsetbuttcap
{
\definecolor{dialinecolor}{rgb}{0.000000, 0.000000, 0.000000}
\pgfsetfillcolor{dialinecolor}
% was here!!!
\pgfsetarrowsend{latex}
\definecolor{dialinecolor}{rgb}{0.000000, 0.000000, 0.000000}
\pgfsetstrokecolor{dialinecolor}
\draw (42.000000\du,35.000000\du)--(42.000000\du,37.000000\du);
}
\pgfsetlinewidth{0.000000\du}
\pgfsetdash{}{0pt}
\pgfsetdash{}{0pt}
\pgfsetbuttcap
\pgfsetmiterjoin
\pgfsetlinewidth{0.000000\du}
\pgfsetbuttcap
\pgfsetmiterjoin
\pgfsetdash{}{0pt}
\definecolor{dialinecolor}{rgb}{0.000000, 0.000000, 0.000000}
\pgfsetstrokecolor{dialinecolor}
\draw (15.000000\du,33.000000\du)--(19.000000\du,33.000000\du)--(19.000000\du,35.000000\du)--(15.000000\du,35.000000\du)--cycle;
\pgfsetbuttcap
\pgfsetmiterjoin
\pgfsetdash{}{0pt}
\definecolor{dialinecolor}{rgb}{0.000000, 0.000000, 0.000000}
\pgfsetstrokecolor{dialinecolor}
\draw (15.000000\du,33.600000\du)--(15.800000\du,34.000000\du);
\pgfsetbuttcap
\pgfsetmiterjoin
\pgfsetdash{}{0pt}
\definecolor{dialinecolor}{rgb}{0.000000, 0.000000, 0.000000}
\pgfsetstrokecolor{dialinecolor}
\draw (15.800000\du,34.000000\du)--(15.000000\du,34.400000\du);
% setfont left to latex
\definecolor{dialinecolor}{rgb}{0.000000, 0.000000, 0.000000}
\pgfsetstrokecolor{dialinecolor}
\node at (17.400000\du,34.200000\du){address};
\pgfsetlinewidth{0.000000\du}
\pgfsetdash{}{0pt}
\pgfsetdash{}{0pt}
\pgfsetbuttcap
{
\definecolor{dialinecolor}{rgb}{0.000000, 0.000000, 0.000000}
\pgfsetfillcolor{dialinecolor}
% was here!!!
\pgfsetarrowsend{latex}
\definecolor{dialinecolor}{rgb}{0.000000, 0.000000, 0.000000}
\pgfsetstrokecolor{dialinecolor}
\draw (17.000000\du,35.000000\du)--(17.000000\du,37.000000\du);
}
\pgfsetlinewidth{0.000000\du}
\pgfsetdash{}{0pt}
\pgfsetdash{}{0pt}
\pgfsetbuttcap
\pgfsetmiterjoin
\pgfsetlinewidth{0.000000\du}
\pgfsetbuttcap
\pgfsetmiterjoin
\pgfsetdash{}{0pt}
\definecolor{dialinecolor}{rgb}{1.000000, 1.000000, 1.000000}
\pgfsetfillcolor{dialinecolor}
\pgfpathmoveto{\pgfpoint{32.366199\du}{11.871624\du}}
\pgfpathcurveto{\pgfpoint{31.901409\du}{11.861488\du}}{\pgfpoint{31.000000\du}{12.074326\du}}{\pgfpoint{31.126761\du}{12.530408\du}}
\pgfpathcurveto{\pgfpoint{31.253520\du}{12.986490\du}}{\pgfpoint{31.859155\du}{13.087837\du}}{\pgfpoint{32.112677\du}{12.956084\du}}
\pgfpathcurveto{\pgfpoint{32.366199\du}{12.824327\du}}{\pgfpoint{31.718310\du}{13.594594\du}}{\pgfpoint{32.957749\du}{13.797297\du}}
\pgfpathcurveto{\pgfpoint{34.197177\du}{14.000000\du}}{\pgfpoint{34.830981\du}{13.675675\du}}{\pgfpoint{34.647882\du}{13.442567\du}}
\pgfpathcurveto{\pgfpoint{34.464783\du}{13.209459\du}}{\pgfpoint{35.732392\du}{13.989865\du}}{\pgfpoint{36.323942\du}{13.543918\du}}
\pgfpathcurveto{\pgfpoint{36.915493\du}{13.097972\du}}{\pgfpoint{35.718307\du}{12.672300\du}}{\pgfpoint{35.971829\du}{12.733111\du}}
\pgfpathcurveto{\pgfpoint{36.225350\du}{12.793922\du}}{\pgfpoint{37.000000\du}{12.712841\du}}{\pgfpoint{36.746478\du}{11.952705\du}}
\pgfpathcurveto{\pgfpoint{36.492957\du}{11.192569\du}}{\pgfpoint{34.211262\du}{11.780407\du}}{\pgfpoint{34.464783\du}{11.668921\du}}
\pgfpathcurveto{\pgfpoint{34.718305\du}{11.557434\du}}{\pgfpoint{34.084501\du}{11.000000\du}}{\pgfpoint{33.295778\du}{11.111487\du}}
\pgfpathcurveto{\pgfpoint{32.507044\du}{11.222974\du}}{\pgfpoint{32.451100\du}{11.425282\du}}{\pgfpoint{32.366593\du}{11.871228\du}}
\pgfpathlineto{\pgfpoint{32.366199\du}{11.871624\du}}
\pgfusepath{fill}
\definecolor{dialinecolor}{rgb}{0.000000, 0.000000, 0.000000}
\pgfsetstrokecolor{dialinecolor}
\pgfpathmoveto{\pgfpoint{32.366199\du}{11.871624\du}}
\pgfpathcurveto{\pgfpoint{31.901409\du}{11.861488\du}}{\pgfpoint{31.000000\du}{12.074326\du}}{\pgfpoint{31.126761\du}{12.530408\du}}
\pgfpathcurveto{\pgfpoint{31.253520\du}{12.986490\du}}{\pgfpoint{31.859155\du}{13.087837\du}}{\pgfpoint{32.112677\du}{12.956084\du}}
\pgfpathcurveto{\pgfpoint{32.366199\du}{12.824327\du}}{\pgfpoint{31.718310\du}{13.594594\du}}{\pgfpoint{32.957749\du}{13.797297\du}}
\pgfpathcurveto{\pgfpoint{34.197177\du}{14.000000\du}}{\pgfpoint{34.830981\du}{13.675675\du}}{\pgfpoint{34.647882\du}{13.442567\du}}
\pgfpathcurveto{\pgfpoint{34.464783\du}{13.209459\du}}{\pgfpoint{35.732392\du}{13.989865\du}}{\pgfpoint{36.323942\du}{13.543918\du}}
\pgfpathcurveto{\pgfpoint{36.915493\du}{13.097972\du}}{\pgfpoint{35.718307\du}{12.672300\du}}{\pgfpoint{35.971829\du}{12.733111\du}}
\pgfpathcurveto{\pgfpoint{36.225350\du}{12.793922\du}}{\pgfpoint{37.000000\du}{12.712841\du}}{\pgfpoint{36.746478\du}{11.952705\du}}
\pgfpathcurveto{\pgfpoint{36.492957\du}{11.192569\du}}{\pgfpoint{34.211262\du}{11.780407\du}}{\pgfpoint{34.464783\du}{11.668921\du}}
\pgfpathcurveto{\pgfpoint{34.718305\du}{11.557434\du}}{\pgfpoint{34.084501\du}{11.000000\du}}{\pgfpoint{33.295778\du}{11.111487\du}}
\pgfpathcurveto{\pgfpoint{32.507044\du}{11.222974\du}}{\pgfpoint{32.451100\du}{11.425282\du}}{\pgfpoint{32.366593\du}{11.871228\du}}
\pgfpathlineto{\pgfpoint{32.366199\du}{11.871624\du}}
\pgfusepath{stroke}
% setfont left to latex
\definecolor{dialinecolor}{rgb}{0.000000, 0.000000, 0.000000}
\pgfsetstrokecolor{dialinecolor}
\node at (34.160839\du,12.752884\du){};
\pgfsetlinewidth{0.000000\du}
\pgfsetdash{}{0pt}
\pgfsetdash{}{0pt}
\pgfsetmiterjoin
\pgfsetbuttcap
{
\definecolor{dialinecolor}{rgb}{0.000000, 0.000000, 0.000000}
\pgfsetfillcolor{dialinecolor}
% was here!!!
\pgfsetarrowsend{latex}
{\pgfsetcornersarced{\pgfpoint{0.000000\du}{0.000000\du}}\definecolor{dialinecolor}{rgb}{0.000000, 0.000000, 0.000000}
\pgfsetstrokecolor{dialinecolor}
\draw (29.000000\du,2.000000\du)--(29.000000\du,8.000000\du)--(33.496300\du,8.000000\du)--(33.496300\du,11.097800\du);
}}
\pgfsetlinewidth{0.000000\du}
\pgfsetdash{}{0pt}
\pgfsetdash{}{0pt}
\pgfsetbuttcap
\pgfsetmiterjoin
\pgfsetlinewidth{0.000000\du}
\pgfsetbuttcap
\pgfsetmiterjoin
\pgfsetdash{}{0pt}
\definecolor{dialinecolor}{rgb}{0.000000, 0.000000, 0.000000}
\pgfsetstrokecolor{dialinecolor}
\draw (31.000000\du,21.000000\du)--(36.000000\du,21.000000\du)--(34.750000\du,23.000000\du)--(32.250000\du,23.000000\du)--cycle;
% setfont left to latex
\definecolor{dialinecolor}{rgb}{0.000000, 0.000000, 0.000000}
\pgfsetstrokecolor{dialinecolor}
\node at (33.500000\du,22.200000\du){};
\pgfsetlinewidth{0.000000\du}
\pgfsetdash{}{0pt}
\pgfsetdash{}{0pt}
\pgfsetmiterjoin
\pgfsetbuttcap
{
\definecolor{dialinecolor}{rgb}{0.000000, 0.000000, 0.000000}
\pgfsetfillcolor{dialinecolor}
% was here!!!
{\pgfsetcornersarced{\pgfpoint{0.000000\du}{0.000000\du}}\definecolor{dialinecolor}{rgb}{0.000000, 0.000000, 0.000000}
\pgfsetstrokecolor{dialinecolor}
\draw (29.000000\du,14.500000\du)--(30.000000\du,14.500000\du)--(30.000000\du,18.000000\du)--(30.000000\du,18.000000\du);
}}
\pgfsetlinewidth{0.000000\du}
\pgfsetdash{}{0pt}
\pgfsetdash{}{0pt}
\pgfsetmiterjoin
\pgfsetbuttcap
{
\definecolor{dialinecolor}{rgb}{0.000000, 0.000000, 0.000000}
\pgfsetfillcolor{dialinecolor}
% was here!!!
\pgfsetarrowsend{latex}
{\pgfsetcornersarced{\pgfpoint{0.000000\du}{0.000000\du}}\definecolor{dialinecolor}{rgb}{0.000000, 0.000000, 0.000000}
\pgfsetstrokecolor{dialinecolor}
\draw (33.500000\du,23.000000\du)--(33.500000\du,24.000000\du)--(27.000000\du,24.000000\du)--(27.000000\du,28.000000\du)--(17.000000\du,28.000000\du)--(17.000000\du,33.000000\du);
}}
\pgfsetlinewidth{0.000000\du}
\pgfsetdash{}{0pt}
\pgfsetdash{}{0pt}
\pgfsetbuttcap
\pgfsetmiterjoin
\pgfsetlinewidth{0.000000\du}
\pgfsetbuttcap
\pgfsetmiterjoin
\pgfsetdash{}{0pt}
\definecolor{dialinecolor}{rgb}{0.000000, 0.000000, 0.000000}
\pgfsetstrokecolor{dialinecolor}
\draw (29.000000\du,18.000000\du)--(31.000000\du,18.000000\du)--(31.000000\du,19.000000\du)--(29.000000\du,19.000000\du)--cycle;
\pgfsetbuttcap
\pgfsetmiterjoin
\pgfsetdash{}{0pt}
\definecolor{dialinecolor}{rgb}{0.000000, 0.000000, 0.000000}
\pgfsetstrokecolor{dialinecolor}
\draw (29.000000\du,18.300000\du)--(29.400000\du,18.500000\du);
\pgfsetbuttcap
\pgfsetmiterjoin
\pgfsetdash{}{0pt}
\definecolor{dialinecolor}{rgb}{0.000000, 0.000000, 0.000000}
\pgfsetstrokecolor{dialinecolor}
\draw (29.400000\du,18.500000\du)--(29.000000\du,18.700000\du);
% setfont left to latex
\definecolor{dialinecolor}{rgb}{0.000000, 0.000000, 0.000000}
\pgfsetstrokecolor{dialinecolor}
\node at (30.200000\du,18.700000\du){};
\pgfsetlinewidth{0.000000\du}
\pgfsetdash{}{0pt}
\pgfsetdash{}{0pt}
\pgfsetmiterjoin
\pgfsetbuttcap
{
\definecolor{dialinecolor}{rgb}{0.000000, 0.000000, 0.000000}
\pgfsetfillcolor{dialinecolor}
% was here!!!
\pgfsetarrowsend{latex}
{\pgfsetcornersarced{\pgfpoint{0.000000\du}{0.000000\du}}\definecolor{dialinecolor}{rgb}{0.000000, 0.000000, 0.000000}
\pgfsetstrokecolor{dialinecolor}
\draw (30.000000\du,19.000000\du)--(30.000000\du,19.000000\du)--(30.000000\du,22.000000\du)--(31.625000\du,22.000000\du);
}}
\pgfsetlinewidth{0.000000\du}
\pgfsetdash{}{0pt}
\pgfsetdash{}{0pt}
\pgfsetbuttcap
\pgfsetmiterjoin
\pgfsetlinewidth{0.000000\du}
\pgfsetbuttcap
\pgfsetmiterjoin
\pgfsetdash{}{0pt}
\definecolor{dialinecolor}{rgb}{0.000000, 0.000000, 0.000000}
\pgfsetstrokecolor{dialinecolor}
\draw (41.000000\du,31.000000\du)--(43.000000\du,31.000000\du)--(43.000000\du,32.000000\du)--(41.000000\du,32.000000\du)--cycle;
\pgfsetbuttcap
\pgfsetmiterjoin
\pgfsetdash{}{0pt}
\definecolor{dialinecolor}{rgb}{0.000000, 0.000000, 0.000000}
\pgfsetstrokecolor{dialinecolor}
\draw (41.000000\du,31.300000\du)--(41.400000\du,31.500000\du);
\pgfsetbuttcap
\pgfsetmiterjoin
\pgfsetdash{}{0pt}
\definecolor{dialinecolor}{rgb}{0.000000, 0.000000, 0.000000}
\pgfsetstrokecolor{dialinecolor}
\draw (41.400000\du,31.500000\du)--(41.000000\du,31.700000\du);
% setfont left to latex
\definecolor{dialinecolor}{rgb}{0.000000, 0.000000, 0.000000}
\pgfsetstrokecolor{dialinecolor}
\node at (42.200000\du,31.700000\du){};
\pgfsetlinewidth{0.000000\du}
\pgfsetdash{}{0pt}
\pgfsetdash{}{0pt}
\pgfsetmiterjoin
\pgfsetbuttcap
{
\definecolor{dialinecolor}{rgb}{0.000000, 0.000000, 0.000000}
\pgfsetfillcolor{dialinecolor}
% was here!!!
\pgfsetarrowsend{latex}
{\pgfsetcornersarced{\pgfpoint{0.000000\du}{0.000000\du}}\definecolor{dialinecolor}{rgb}{0.000000, 0.000000, 0.000000}
\pgfsetstrokecolor{dialinecolor}
\draw (42.000000\du,2.000000\du)--(42.000000\du,27.000000\du)--(42.000000\du,27.000000\du)--(42.000000\du,31.000000\du);
}}
\pgfsetlinewidth{0.000000\du}
\pgfsetdash{}{0pt}
\pgfsetdash{}{0pt}
\pgfsetbuttcap
\pgfsetmiterjoin
\pgfsetlinewidth{0.000000\du}
\pgfsetbuttcap
\pgfsetmiterjoin
\pgfsetdash{}{0pt}
\definecolor{dialinecolor}{rgb}{0.000000, 0.000000, 0.000000}
\pgfsetstrokecolor{dialinecolor}
\draw (27.000000\du,33.000000\du)--(31.000000\du,33.000000\du)--(31.000000\du,35.000000\du)--(27.000000\du,35.000000\du)--cycle;
\pgfsetbuttcap
\pgfsetmiterjoin
\pgfsetdash{}{0pt}
\definecolor{dialinecolor}{rgb}{0.000000, 0.000000, 0.000000}
\pgfsetstrokecolor{dialinecolor}
\draw (27.000000\du,33.600000\du)--(27.800000\du,34.000000\du);
\pgfsetbuttcap
\pgfsetmiterjoin
\pgfsetdash{}{0pt}
\definecolor{dialinecolor}{rgb}{0.000000, 0.000000, 0.000000}
\pgfsetstrokecolor{dialinecolor}
\draw (27.800000\du,34.000000\du)--(27.000000\du,34.400000\du);
% setfont left to latex
\definecolor{dialinecolor}{rgb}{0.000000, 0.000000, 0.000000}
\pgfsetstrokecolor{dialinecolor}
\node at (29.400000\du,34.200000\du){engine};
\pgfsetlinewidth{0.000000\du}
\pgfsetdash{}{0pt}
\pgfsetdash{}{0pt}
\pgfsetbuttcap
\pgfsetmiterjoin
\pgfsetlinewidth{0.000000\du}
\pgfsetbuttcap
\pgfsetmiterjoin
\pgfsetdash{}{0pt}
\definecolor{dialinecolor}{rgb}{0.000000, 0.000000, 0.000000}
\pgfsetstrokecolor{dialinecolor}
\draw (21.000000\du,33.000000\du)--(25.000000\du,33.000000\du)--(25.000000\du,35.000000\du)--(21.000000\du,35.000000\du)--cycle;
\pgfsetbuttcap
\pgfsetmiterjoin
\pgfsetdash{}{0pt}
\definecolor{dialinecolor}{rgb}{0.000000, 0.000000, 0.000000}
\pgfsetstrokecolor{dialinecolor}
\draw (21.000000\du,33.600000\du)--(21.800000\du,34.000000\du);
\pgfsetbuttcap
\pgfsetmiterjoin
\pgfsetdash{}{0pt}
\definecolor{dialinecolor}{rgb}{0.000000, 0.000000, 0.000000}
\pgfsetstrokecolor{dialinecolor}
\draw (21.800000\du,34.000000\du)--(21.000000\du,34.400000\du);
% setfont left to latex
\definecolor{dialinecolor}{rgb}{0.000000, 0.000000, 0.000000}
\pgfsetstrokecolor{dialinecolor}
\node at (23.400000\du,34.200000\du){valid};
\pgfsetlinewidth{0.000000\du}
\pgfsetdash{}{0pt}
\pgfsetdash{}{0pt}
\pgfsetbuttcap
{
\definecolor{dialinecolor}{rgb}{0.000000, 0.000000, 0.000000}
\pgfsetfillcolor{dialinecolor}
% was here!!!
\pgfsetarrowsend{latex}
\definecolor{dialinecolor}{rgb}{0.000000, 0.000000, 0.000000}
\pgfsetstrokecolor{dialinecolor}
\draw (23.000000\du,32.000000\du)--(23.000000\du,33.000000\du);
}
\pgfsetlinewidth{0.000000\du}
\pgfsetdash{}{0pt}
\pgfsetdash{}{0pt}
\pgfsetbuttcap
\pgfsetmiterjoin
\pgfsetlinewidth{0.000000\du}
\pgfsetbuttcap
\pgfsetmiterjoin
\pgfsetdash{}{0pt}
\definecolor{dialinecolor}{rgb}{0.000000, 0.000000, 0.000000}
\pgfsetstrokecolor{dialinecolor}
\draw (22.000000\du,31.000000\du)--(24.000000\du,31.000000\du)--(24.000000\du,32.000000\du)--(22.000000\du,32.000000\du)--cycle;
\pgfsetbuttcap
\pgfsetmiterjoin
\pgfsetdash{}{0pt}
\definecolor{dialinecolor}{rgb}{0.000000, 0.000000, 0.000000}
\pgfsetstrokecolor{dialinecolor}
\draw (22.000000\du,31.300000\du)--(22.400000\du,31.500000\du);
\pgfsetbuttcap
\pgfsetmiterjoin
\pgfsetdash{}{0pt}
\definecolor{dialinecolor}{rgb}{0.000000, 0.000000, 0.000000}
\pgfsetstrokecolor{dialinecolor}
\draw (22.400000\du,31.500000\du)--(22.000000\du,31.700000\du);
% setfont left to latex
\definecolor{dialinecolor}{rgb}{0.000000, 0.000000, 0.000000}
\pgfsetstrokecolor{dialinecolor}
\node at (23.200000\du,31.700000\du){};
\pgfsetlinewidth{0.000000\du}
\pgfsetdash{}{0pt}
\pgfsetdash{}{0pt}
\pgfsetbuttcap
\pgfsetmiterjoin
\pgfsetlinewidth{0.000000\du}
\pgfsetbuttcap
\pgfsetmiterjoin
\pgfsetdash{}{0pt}
\definecolor{dialinecolor}{rgb}{0.000000, 0.000000, 0.000000}
\pgfsetstrokecolor{dialinecolor}
\draw (28.000000\du,31.000000\du)--(30.000000\du,31.000000\du)--(30.000000\du,32.000000\du)--(28.000000\du,32.000000\du)--cycle;
\pgfsetbuttcap
\pgfsetmiterjoin
\pgfsetdash{}{0pt}
\definecolor{dialinecolor}{rgb}{0.000000, 0.000000, 0.000000}
\pgfsetstrokecolor{dialinecolor}
\draw (28.000000\du,31.300000\du)--(28.400000\du,31.500000\du);
\pgfsetbuttcap
\pgfsetmiterjoin
\pgfsetdash{}{0pt}
\definecolor{dialinecolor}{rgb}{0.000000, 0.000000, 0.000000}
\pgfsetstrokecolor{dialinecolor}
\draw (28.400000\du,31.500000\du)--(28.000000\du,31.700000\du);
% setfont left to latex
\definecolor{dialinecolor}{rgb}{0.000000, 0.000000, 0.000000}
\pgfsetstrokecolor{dialinecolor}
\node at (29.200000\du,31.700000\du){};
\pgfsetlinewidth{0.000000\du}
\pgfsetdash{}{0pt}
\pgfsetdash{}{0pt}
\pgfsetmiterjoin
\pgfsetbuttcap
{
\definecolor{dialinecolor}{rgb}{0.000000, 0.000000, 0.000000}
\pgfsetfillcolor{dialinecolor}
% was here!!!
\pgfsetarrowsend{latex}
{\pgfsetcornersarced{\pgfpoint{0.000000\du}{0.000000\du}}\definecolor{dialinecolor}{rgb}{0.000000, 0.000000, 0.000000}
\pgfsetstrokecolor{dialinecolor}
\draw (35.000000\du,2.000000\du)--(35.000000\du,2.000000\du)--(35.000000\du,5.000000\du)--(40.000000\du,5.000000\du)--(40.000000\du,26.000000\du)--(29.000000\du,26.000000\du)--(29.000000\du,31.000000\du);
}}
\pgfsetlinewidth{0.000000\du}
\pgfsetdash{}{0pt}
\pgfsetdash{}{0pt}
\pgfsetmiterjoin
\pgfsetbuttcap
{
\definecolor{dialinecolor}{rgb}{0.000000, 0.000000, 0.000000}
\pgfsetfillcolor{dialinecolor}
% was here!!!
\pgfsetarrowsend{latex}
{\pgfsetcornersarced{\pgfpoint{0.000000\du}{0.000000\du}}\definecolor{dialinecolor}{rgb}{0.000000, 0.000000, 0.000000}
\pgfsetstrokecolor{dialinecolor}
\draw (29.000000\du,2.000000\du)--(29.000000\du,2.000000\du)--(29.000000\du,8.000000\du)--(39.000000\du,8.000000\du)--(39.000000\du,25.000000\du)--(28.000000\du,25.000000\du)--(28.000000\du,29.000000\du)--(23.000000\du,29.000000\du)--(23.000000\du,31.000000\du);
}}
\pgfsetlinewidth{0.000000\du}
\pgfsetdash{}{0pt}
\pgfsetdash{}{0pt}
\pgfsetbuttcap
\pgfsetmiterjoin
\pgfsetlinewidth{0.000000\du}
\pgfsetbuttcap
\pgfsetmiterjoin
\pgfsetdash{}{0pt}
\pgfsetbuttcap
\pgfsetmiterjoin
\pgfsetdash{}{0pt}
\definecolor{dialinecolor}{rgb}{0.000000, 0.000000, 0.000000}
\pgfsetfillcolor{dialinecolor}
\pgfpathellipse{\pgfpoint{33.500000\du}{8.000000\du}}{\pgfpoint{0.125000\du}{0\du}}{\pgfpoint{0\du}{0.125000\du}}
\pgfusepath{fill}
\definecolor{dialinecolor}{rgb}{0.000000, 0.000000, 0.000000}
\pgfsetstrokecolor{dialinecolor}
\pgfpathellipse{\pgfpoint{33.500000\du}{8.000000\du}}{\pgfpoint{0.125000\du}{0\du}}{\pgfpoint{0\du}{0.125000\du}}
\pgfusepath{stroke}
\pgfsetlinewidth{0.000000\du}
\pgfsetdash{}{0pt}
\pgfsetdash{}{0pt}
\pgfsetbuttcap
{
\definecolor{dialinecolor}{rgb}{0.000000, 0.000000, 0.000000}
\pgfsetfillcolor{dialinecolor}
% was here!!!
\definecolor{dialinecolor}{rgb}{0.000000, 0.000000, 0.000000}
\pgfsetstrokecolor{dialinecolor}
\draw (32.250000\du,21.000000\du)--(34.750000\du,21.000000\du);
}
\pgfsetlinewidth{0.000000\du}
\pgfsetdash{}{0pt}
\pgfsetdash{}{0pt}
\pgfsetmiterjoin
\pgfsetbuttcap
{
\definecolor{dialinecolor}{rgb}{0.000000, 0.000000, 0.000000}
\pgfsetfillcolor{dialinecolor}
% was here!!!
\pgfsetarrowsend{latex}
{\pgfsetcornersarced{\pgfpoint{0.000000\du}{0.000000\du}}\definecolor{dialinecolor}{rgb}{0.000000, 0.000000, 0.000000}
\pgfsetstrokecolor{dialinecolor}
\draw (31.112000\du,12.455600\du)--(30.000000\du,12.455600\du)--(30.000000\du,14.000000\du)--(32.250000\du,14.000000\du)--(32.250000\du,21.000000\du);
}}
\pgfsetlinewidth{0.000000\du}
\pgfsetdash{}{0pt}
\pgfsetdash{}{0pt}
\pgfsetmiterjoin
\pgfsetbuttcap
{
\definecolor{dialinecolor}{rgb}{0.000000, 0.000000, 0.000000}
\pgfsetfillcolor{dialinecolor}
% was here!!!
\pgfsetarrowsend{latex}
{\pgfsetcornersarced{\pgfpoint{0.000000\du}{0.000000\du}}\definecolor{dialinecolor}{rgb}{0.000000, 0.000000, 0.000000}
\pgfsetstrokecolor{dialinecolor}
\draw (36.794100\du,12.237700\du)--(37.500000\du,12.237700\du)--(37.500000\du,17.000000\du)--(34.750000\du,17.000000\du)--(34.750000\du,21.000000\du);
}}
\pgfsetlinewidth{0.000000\du}
\pgfsetdash{}{0pt}
\pgfsetdash{}{0pt}
\pgfsetmiterjoin
\pgfsetbuttcap
{
\definecolor{dialinecolor}{rgb}{0.000000, 0.000000, 0.000000}
\pgfsetfillcolor{dialinecolor}
% was here!!!
\pgfsetarrowsend{latex}
{\pgfsetcornersarced{\pgfpoint{0.000000\du}{0.000000\du}}\definecolor{dialinecolor}{rgb}{0.000000, 0.000000, 0.000000}
\pgfsetstrokecolor{dialinecolor}
\draw (33.692500\du,13.862500\du)--(33.692500\du,16.000000\du)--(33.500000\du,16.000000\du)--(33.500000\du,21.000000\du);
}}
% setfont left to latex
\definecolor{dialinecolor}{rgb}{0.000000, 0.000000, 0.000000}
\pgfsetstrokecolor{dialinecolor}
\node[anchor=west] at (14.800000\du,26.800000\du){- Random (LFSR)};
% setfont left to latex
\definecolor{dialinecolor}{rgb}{0.000000, 0.000000, 0.000000}
\pgfsetstrokecolor{dialinecolor}
\node[anchor=west] at (14.800000\du,27.600000\du){- Increment};
\pgfsetlinewidth{0.000000\du}
\pgfsetdash{}{0pt}
\pgfsetdash{}{0pt}
\pgfsetbuttcap
{
\definecolor{dialinecolor}{rgb}{0.000000, 0.000000, 0.000000}
\pgfsetfillcolor{dialinecolor}
% was here!!!
\pgfsetarrowsend{latex}
\definecolor{dialinecolor}{rgb}{0.000000, 0.000000, 0.000000}
\pgfsetstrokecolor{dialinecolor}
\draw (23.000000\du,35.000000\du)--(23.000000\du,37.000000\du);
}
\pgfsetlinewidth{0.000000\du}
\pgfsetdash{}{0pt}
\pgfsetdash{}{0pt}
\pgfsetbuttcap
{
\definecolor{dialinecolor}{rgb}{0.000000, 0.000000, 0.000000}
\pgfsetfillcolor{dialinecolor}
% was here!!!
\pgfsetarrowsend{latex}
\definecolor{dialinecolor}{rgb}{0.000000, 0.000000, 0.000000}
\pgfsetstrokecolor{dialinecolor}
\draw (29.000000\du,35.000000\du)--(29.000000\du,37.000000\du);
}
\pgfsetlinewidth{0.000000\du}
\pgfsetdash{}{0pt}
\pgfsetdash{}{0pt}
\pgfsetbuttcap
\pgfsetmiterjoin
\pgfsetlinewidth{0.000000\du}
\pgfsetbuttcap
\pgfsetmiterjoin
\pgfsetdash{}{0pt}
\pgfsetbuttcap
\pgfsetmiterjoin
\pgfsetdash{}{0pt}
\definecolor{dialinecolor}{rgb}{0.000000, 0.000000, 0.000000}
\pgfsetfillcolor{dialinecolor}
\pgfpathellipse{\pgfpoint{35.000000\du}{5.000000\du}}{\pgfpoint{0.125000\du}{0\du}}{\pgfpoint{0\du}{0.125000\du}}
\pgfusepath{fill}
\definecolor{dialinecolor}{rgb}{0.000000, 0.000000, 0.000000}
\pgfsetstrokecolor{dialinecolor}
\pgfpathellipse{\pgfpoint{35.000000\du}{5.000000\du}}{\pgfpoint{0.125000\du}{0\du}}{\pgfpoint{0\du}{0.125000\du}}
\pgfusepath{stroke}
\pgfsetlinewidth{0.000000\du}
\pgfsetdash{}{0pt}
\pgfsetdash{}{0pt}
\pgfsetbuttcap
{
\definecolor{dialinecolor}{rgb}{0.000000, 0.000000, 0.000000}
\pgfsetfillcolor{dialinecolor}
% was here!!!
\pgfsetarrowsend{latex}
\definecolor{dialinecolor}{rgb}{0.000000, 0.000000, 0.000000}
\pgfsetstrokecolor{dialinecolor}
\draw (29.000000\du,32.000000\du)--(29.000000\du,33.000000\du);
}
\pgfsetlinewidth{0.000000\du}
\pgfsetdash{}{0pt}
\pgfsetdash{}{0pt}
\pgfsetbuttcap
{
\definecolor{dialinecolor}{rgb}{0.000000, 0.000000, 0.000000}
\pgfsetfillcolor{dialinecolor}
% was here!!!
\pgfsetarrowsend{latex}
\definecolor{dialinecolor}{rgb}{0.000000, 0.000000, 0.000000}
\pgfsetstrokecolor{dialinecolor}
\draw (42.000000\du,32.000000\du)--(42.000000\du,33.000000\du);
}
\pgfsetlinewidth{0.000000\du}
\pgfsetdash{}{0pt}
\pgfsetdash{}{0pt}
\pgfsetbuttcap
\pgfsetmiterjoin
\pgfsetlinewidth{0.000000\du}
\pgfsetbuttcap
\pgfsetmiterjoin
\pgfsetdash{}{0pt}
\definecolor{dialinecolor}{rgb}{0.000000, 0.000000, 0.000000}
\pgfsetstrokecolor{dialinecolor}
\draw (33.000000\du,33.000000\du)--(37.000000\du,33.000000\du)--(37.000000\du,35.000000\du)--(33.000000\du,35.000000\du)--cycle;
\pgfsetbuttcap
\pgfsetmiterjoin
\pgfsetdash{}{0pt}
\definecolor{dialinecolor}{rgb}{0.000000, 0.000000, 0.000000}
\pgfsetstrokecolor{dialinecolor}
\draw (33.000000\du,33.600000\du)--(33.800000\du,34.000000\du);
\pgfsetbuttcap
\pgfsetmiterjoin
\pgfsetdash{}{0pt}
\definecolor{dialinecolor}{rgb}{0.000000, 0.000000, 0.000000}
\pgfsetstrokecolor{dialinecolor}
\draw (33.800000\du,34.000000\du)--(33.000000\du,34.400000\du);
% setfont left to latex
\definecolor{dialinecolor}{rgb}{0.000000, 0.000000, 0.000000}
\pgfsetstrokecolor{dialinecolor}
\node at (35.400000\du,34.200000\du){tag};
\pgfsetlinewidth{0.000000\du}
\pgfsetdash{}{0pt}
\pgfsetdash{}{0pt}
\pgfsetbuttcap
{
\definecolor{dialinecolor}{rgb}{0.000000, 0.000000, 0.000000}
\pgfsetfillcolor{dialinecolor}
% was here!!!
\pgfsetarrowsend{latex}
\definecolor{dialinecolor}{rgb}{0.000000, 0.000000, 0.000000}
\pgfsetstrokecolor{dialinecolor}
\draw (35.000000\du,35.000000\du)--(35.000000\du,37.000000\du);
}
\pgfsetlinewidth{0.000000\du}
\pgfsetdash{}{0pt}
\pgfsetdash{}{0pt}
\pgfsetbuttcap
\pgfsetmiterjoin
\pgfsetlinewidth{0.000000\du}
\pgfsetbuttcap
\pgfsetmiterjoin
\pgfsetdash{}{0pt}
\definecolor{dialinecolor}{rgb}{0.000000, 0.000000, 0.000000}
\pgfsetstrokecolor{dialinecolor}
\draw (34.000000\du,30.000000\du)--(36.000000\du,30.000000\du)--(36.000000\du,31.000000\du)--(34.000000\du,31.000000\du)--cycle;
\pgfsetbuttcap
\pgfsetmiterjoin
\pgfsetdash{}{0pt}
\definecolor{dialinecolor}{rgb}{0.000000, 0.000000, 0.000000}
\pgfsetstrokecolor{dialinecolor}
\draw (34.000000\du,30.300000\du)--(34.400000\du,30.500000\du);
\pgfsetbuttcap
\pgfsetmiterjoin
\pgfsetdash{}{0pt}
\definecolor{dialinecolor}{rgb}{0.000000, 0.000000, 0.000000}
\pgfsetstrokecolor{dialinecolor}
\draw (34.400000\du,30.500000\du)--(34.000000\du,30.700000\du);
% setfont left to latex
\definecolor{dialinecolor}{rgb}{0.000000, 0.000000, 0.000000}
\pgfsetstrokecolor{dialinecolor}
\node at (35.200000\du,30.700000\du){};
\pgfsetlinewidth{0.000000\du}
\pgfsetdash{}{0pt}
\pgfsetdash{}{0pt}
\pgfsetbuttcap
{
\definecolor{dialinecolor}{rgb}{0.000000, 0.000000, 0.000000}
\pgfsetfillcolor{dialinecolor}
% was here!!!
\pgfsetarrowsend{latex}
\definecolor{dialinecolor}{rgb}{0.000000, 0.000000, 0.000000}
\pgfsetstrokecolor{dialinecolor}
\draw (35.000000\du,31.000000\du)--(35.000000\du,33.000000\du);
}
\pgfsetlinewidth{0.000000\du}
\pgfsetdash{}{0pt}
\pgfsetdash{}{0pt}
\pgfsetbuttcap
\pgfsetmiterjoin
\pgfsetlinewidth{0.000000\du}
\pgfsetbuttcap
\pgfsetmiterjoin
\pgfsetdash{}{0pt}
\definecolor{dialinecolor}{rgb}{1.000000, 1.000000, 1.000000}
\pgfsetfillcolor{dialinecolor}
\pgfpathmoveto{\pgfpoint{36.683099\du}{27.586631\du}}
\pgfpathcurveto{\pgfpoint{36.450704\du}{27.579810\du}}{\pgfpoint{36.000000\du}{27.723057\du}}{\pgfpoint{36.063380\du}{28.030014\du}}
\pgfpathcurveto{\pgfpoint{36.126760\du}{28.336972\du}}{\pgfpoint{36.429578\du}{28.405182\du}}{\pgfpoint{36.556338\du}{28.316508\du}}
\pgfpathcurveto{\pgfpoint{36.683099\du}{28.227832\du}}{\pgfpoint{36.359155\du}{28.746246\du}}{\pgfpoint{36.978875\du}{28.882672\du}}
\pgfpathcurveto{\pgfpoint{37.598589\du}{29.019098\du}}{\pgfpoint{37.915491\du}{28.800817\du}}{\pgfpoint{37.823941\du}{28.643927\du}}
\pgfpathcurveto{\pgfpoint{37.732392\du}{28.487038\du}}{\pgfpoint{38.366196\du}{29.012276\du}}{\pgfpoint{38.661971\du}{28.712140\du}}
\pgfpathcurveto{\pgfpoint{38.957746\du}{28.412003\du}}{\pgfpoint{38.359154\du}{28.125512\du}}{\pgfpoint{38.485914\du}{28.166440\du}}
\pgfpathcurveto{\pgfpoint{38.612675\du}{28.207368\du}}{\pgfpoint{39.000000\du}{28.152797\du}}{\pgfpoint{38.873239\du}{27.641201\du}}
\pgfpathcurveto{\pgfpoint{38.746478\du}{27.129605\du}}{\pgfpoint{37.605631\du}{27.525239\du}}{\pgfpoint{37.732392\du}{27.450205\du}}
\pgfpathcurveto{\pgfpoint{37.859153\du}{27.375171\du}}{\pgfpoint{37.542250\du}{27.000000\du}}{\pgfpoint{37.147889\du}{27.075034\du}}
\pgfpathcurveto{\pgfpoint{36.753522\du}{27.150069\du}}{\pgfpoint{36.725550\du}{27.286229\du}}{\pgfpoint{36.683297\du}{27.586365\du}}
\pgfpathlineto{\pgfpoint{36.683099\du}{27.586631\du}}
\pgfusepath{fill}
\definecolor{dialinecolor}{rgb}{0.000000, 0.000000, 0.000000}
\pgfsetstrokecolor{dialinecolor}
\pgfpathmoveto{\pgfpoint{36.683099\du}{27.586631\du}}
\pgfpathcurveto{\pgfpoint{36.450704\du}{27.579810\du}}{\pgfpoint{36.000000\du}{27.723057\du}}{\pgfpoint{36.063380\du}{28.030014\du}}
\pgfpathcurveto{\pgfpoint{36.126760\du}{28.336972\du}}{\pgfpoint{36.429578\du}{28.405182\du}}{\pgfpoint{36.556338\du}{28.316508\du}}
\pgfpathcurveto{\pgfpoint{36.683099\du}{28.227832\du}}{\pgfpoint{36.359155\du}{28.746246\du}}{\pgfpoint{36.978875\du}{28.882672\du}}
\pgfpathcurveto{\pgfpoint{37.598589\du}{29.019098\du}}{\pgfpoint{37.915491\du}{28.800817\du}}{\pgfpoint{37.823941\du}{28.643927\du}}
\pgfpathcurveto{\pgfpoint{37.732392\du}{28.487038\du}}{\pgfpoint{38.366196\du}{29.012276\du}}{\pgfpoint{38.661971\du}{28.712140\du}}
\pgfpathcurveto{\pgfpoint{38.957746\du}{28.412003\du}}{\pgfpoint{38.359154\du}{28.125512\du}}{\pgfpoint{38.485914\du}{28.166440\du}}
\pgfpathcurveto{\pgfpoint{38.612675\du}{28.207368\du}}{\pgfpoint{39.000000\du}{28.152797\du}}{\pgfpoint{38.873239\du}{27.641201\du}}
\pgfpathcurveto{\pgfpoint{38.746478\du}{27.129605\du}}{\pgfpoint{37.605631\du}{27.525239\du}}{\pgfpoint{37.732392\du}{27.450205\du}}
\pgfpathcurveto{\pgfpoint{37.859153\du}{27.375171\du}}{\pgfpoint{37.542250\du}{27.000000\du}}{\pgfpoint{37.147889\du}{27.075034\du}}
\pgfpathcurveto{\pgfpoint{36.753522\du}{27.150069\du}}{\pgfpoint{36.725550\du}{27.286229\du}}{\pgfpoint{36.683297\du}{27.586365\du}}
\pgfpathlineto{\pgfpoint{36.683099\du}{27.586631\du}}
\pgfusepath{stroke}
% setfont left to latex
\definecolor{dialinecolor}{rgb}{0.000000, 0.000000, 0.000000}
\pgfsetstrokecolor{dialinecolor}
\node at (37.580420\du,28.245141\du){};
\pgfsetlinewidth{0.000000\du}
\pgfsetdash{}{0pt}
\pgfsetdash{}{0pt}
\pgfsetmiterjoin
\pgfsetbuttcap
{
\definecolor{dialinecolor}{rgb}{0.000000, 0.000000, 0.000000}
\pgfsetfillcolor{dialinecolor}
% was here!!!
\pgfsetarrowsend{latex}
{\pgfsetcornersarced{\pgfpoint{0.000000\du}{0.000000\du}}\definecolor{dialinecolor}{rgb}{0.000000, 0.000000, 0.000000}
\pgfsetstrokecolor{dialinecolor}
\draw (36.056000\du,27.979700\du)--(36.056000\du,28.000000\du)--(35.000000\du,28.000000\du)--(35.000000\du,30.000000\du);
}}
\pgfsetlinewidth{0.000000\du}
\pgfsetdash{}{0pt}
\pgfsetdash{}{0pt}
\pgfsetmiterjoin
\pgfsetbuttcap
{
\definecolor{dialinecolor}{rgb}{0.000000, 0.000000, 0.000000}
\pgfsetfillcolor{dialinecolor}
% was here!!!
\pgfsetarrowsend{latex}
{\pgfsetcornersarced{\pgfpoint{0.000000\du}{0.000000\du}}\definecolor{dialinecolor}{rgb}{0.000000, 0.000000, 0.000000}
\pgfsetstrokecolor{dialinecolor}
\draw (35.000000\du,31.000000\du)--(35.000000\du,32.000000\du)--(37.346300\du,32.000000\du)--(37.346300\du,28.926600\du);
}}
\pgfsetlinewidth{0.000000\du}
\pgfsetdash{}{0pt}
\pgfsetdash{}{0pt}
\pgfsetmiterjoin
\pgfsetbuttcap
{
\definecolor{dialinecolor}{rgb}{0.000000, 0.000000, 0.000000}
\pgfsetfillcolor{dialinecolor}
% was here!!!
\pgfsetarrowsend{latex}
{\pgfsetcornersarced{\pgfpoint{0.000000\du}{0.000000\du}}\definecolor{dialinecolor}{rgb}{0.000000, 0.000000, 0.000000}
\pgfsetstrokecolor{dialinecolor}
\draw (29.000000\du,2.000000\du)--(29.000000\du,8.000000\du)--(39.000000\du,8.000000\du)--(39.000000\du,25.000000\du)--(40.800000\du,25.000000\du)--(40.800000\du,27.833000\du)--(38.897100\du,27.833000\du);
}}
\pgfsetlinewidth{0.000000\du}
\pgfsetdash{}{0pt}
\pgfsetdash{}{0pt}
\pgfsetbuttcap
\pgfsetmiterjoin
\pgfsetlinewidth{0.000000\du}
\pgfsetbuttcap
\pgfsetmiterjoin
\pgfsetdash{}{0pt}
\pgfsetbuttcap
\pgfsetmiterjoin
\pgfsetdash{}{0pt}
\definecolor{dialinecolor}{rgb}{0.000000, 0.000000, 0.000000}
\pgfsetfillcolor{dialinecolor}
\pgfpathellipse{\pgfpoint{39.000000\du}{25.000000\du}}{\pgfpoint{0.125000\du}{0\du}}{\pgfpoint{0\du}{0.125000\du}}
\pgfusepath{fill}
\definecolor{dialinecolor}{rgb}{0.000000, 0.000000, 0.000000}
\pgfsetstrokecolor{dialinecolor}
\pgfpathellipse{\pgfpoint{39.000000\du}{25.000000\du}}{\pgfpoint{0.125000\du}{0\du}}{\pgfpoint{0\du}{0.125000\du}}
\pgfusepath{stroke}
\end{tikzpicture}

  \end{center}
  \caption[Overview of fbist\_agen.vhdl]{\label{fig:fbist_agen}Overview of fbist\_agen.vhdl
  }
\end{figure}

\subsection{Data Generator}
The Data Generator, shown in Figure~\ref{fig:fbist_dgen}, uses an
address to generate data for write commands. Read commands pass
through unchanged. The tag, engine, and other values are not used when
generating data, as these aren't necessarily the same when reading the
generated data in the future. However, there are some mutations that
can be programmed to be added to commands that modify the data. This
ensures that the the data in memory is modified over time, rather than
the same data being written again and again. The mutations are also
used when mixing commands of different sizes to the same location.

In general, what data is used doesn't fundamentally alter the behavior
of a test. Data is data, and in practice not much behavior depends on
it. For this reason, the data generator is less configurable than the
command and address generators. Each pool is configured with a certain
data mode, with each command in that pool using that method of
generating data. Data generation modes are:
\begin{itemize}
  \item Random
  \item Data Equals Address
  \item All 0
  \item All 1
  \item Repeating 0xA, 0x5
  \item Repeating 0xC, 0x3
  \item Repeating 0xF, 0x0
  \item User-Supplied Pattern
\end{itemize}

\subsection{FIFO and Flow Control}
Everything described previously is generated and propagated downstream
at a steady state. If FBIST is isolated, a valid command, with address
and data, would be presented at the output of the data generator every
cycle. However, when interfacing with the OpenCAPI link, this is not
necessarily desired. The OpenCAPI layer adds some behavior that FBIST
isn't aware of, such as handling of credits mixed in with commands and
possible replays. Additionally, since FBIST broadcasts to all OpenCAPI
ports on Fire, the behavior of one link affects the traffic to
others. Thus, there is a FIFO between the output of the data generator
and input to the OpenCAPI link. This absorbs any possible bubbles
needed in the FBIST traffic, and also provides a clock gating signal
to shutdown the FBIST pipeline when needed.

The output of the FIFO is multiple AXI masters in parallel to talk to
the TLT. In order to achieve the maximum bandwidth on the OpenCAPI
link, FBIST needs to present multiple commands in a single cycle. This
requires the AXI masters and slaves to work together, which is a
requirement above the AXI spec. Additionally, the AXI spec is relaxed
a bit in order to achieve back to back transactions. While this
behavior is not part of the AXI spec, other AXI masters that do in
fact abide by the spec can talk to the TLT, albeit with a slight
decrease in latency.

\subsection{Upstream Checker}
The Upstream Checker, shown in Figure~\ref{fig:fbist_chk}, consists
primarily of a LUT and a copy of the DGEN. The LUT stores the tags and
addresses of outstanding OpenCAPI commands. When a response is
received, these are removed from the LUT, checked, and either logged
(for failure) or discarded (for success). For write commands, there is
no way of checking the data directly; only the response itself is
checked. For reads, we can check the response, as well as the returned
data. Because data is deterministically generated from the address,
after the command is removed from the LUT, we regenerate the expected
data, and compare with the actual data. If there is a miscompare, it
is not generally known if this is an issue with the read or the
write. Further tests, or tracing on OCMB, would be needed to determine
which operation failed.

\begin{figure}[h]
  \begin{center}
    % Graphic for TeX using PGF
% Title: /home/rpking42/proj/chips/astra/doc/images/fbist_chk.dia
% Creator: Dia v0.97.3
% CreationDate: Wed Nov  7 17:59:41 2018
% For: rpking42
% \usepackage{tikz}
% The following commands are not supported in PSTricks at present
% We define them conditionally, so when they are implemented,
% this pgf file will use them.
\ifx\du\undefined
  \newlength{\du}
\fi
\setlength{\du}{15\unitlength}
\begin{tikzpicture}
\pgftransformxscale{1.000000}
\pgftransformyscale{-1.000000}
\definecolor{dialinecolor}{rgb}{0.000000, 0.000000, 0.000000}
\pgfsetstrokecolor{dialinecolor}
\definecolor{dialinecolor}{rgb}{1.000000, 1.000000, 1.000000}
\pgfsetfillcolor{dialinecolor}
\pgfsetlinewidth{0.000000\du}
\pgfsetdash{}{0pt}
\pgfsetdash{}{0pt}
\pgfsetbuttcap
\pgfsetmiterjoin
\pgfsetlinewidth{0.000000\du}
\pgfsetbuttcap
\pgfsetmiterjoin
\pgfsetdash{}{0pt}
\definecolor{dialinecolor}{rgb}{0.000000, 0.000000, 0.000000}
\pgfsetstrokecolor{dialinecolor}
\draw (16.000000\du,47.000000\du)--(22.000000\du,47.000000\du)--(22.000000\du,49.000000\du)--(16.000000\du,49.000000\du)--cycle;
\pgfsetbuttcap
\pgfsetmiterjoin
\pgfsetdash{}{0pt}
\definecolor{dialinecolor}{rgb}{0.000000, 0.000000, 0.000000}
\pgfsetstrokecolor{dialinecolor}
\draw (16.000000\du,47.600000\du)--(17.200000\du,48.000000\du);
\pgfsetbuttcap
\pgfsetmiterjoin
\pgfsetdash{}{0pt}
\definecolor{dialinecolor}{rgb}{0.000000, 0.000000, 0.000000}
\pgfsetstrokecolor{dialinecolor}
\draw (17.200000\du,48.000000\du)--(16.000000\du,48.400000\du);
% setfont left to latex
\definecolor{dialinecolor}{rgb}{0.000000, 0.000000, 0.000000}
\pgfsetstrokecolor{dialinecolor}
\node at (19.600000\du,48.200000\du){command};
\pgfsetlinewidth{0.000000\du}
\pgfsetdash{}{0pt}
\pgfsetdash{}{0pt}
\pgfsetbuttcap
\pgfsetmiterjoin
\pgfsetlinewidth{0.000000\du}
\pgfsetbuttcap
\pgfsetmiterjoin
\pgfsetdash{}{0pt}
\definecolor{dialinecolor}{rgb}{0.000000, 0.000000, 0.000000}
\pgfsetstrokecolor{dialinecolor}
\draw (15.000000\du,33.000000\du)--(19.000000\du,33.000000\du)--(19.000000\du,35.000000\du)--(15.000000\du,35.000000\du)--cycle;
\pgfsetbuttcap
\pgfsetmiterjoin
\pgfsetdash{}{0pt}
\definecolor{dialinecolor}{rgb}{0.000000, 0.000000, 0.000000}
\pgfsetstrokecolor{dialinecolor}
\draw (15.000000\du,33.600000\du)--(15.800000\du,34.000000\du);
\pgfsetbuttcap
\pgfsetmiterjoin
\pgfsetdash{}{0pt}
\definecolor{dialinecolor}{rgb}{0.000000, 0.000000, 0.000000}
\pgfsetstrokecolor{dialinecolor}
\draw (15.800000\du,34.000000\du)--(15.000000\du,34.400000\du);
% setfont left to latex
\definecolor{dialinecolor}{rgb}{0.000000, 0.000000, 0.000000}
\pgfsetstrokecolor{dialinecolor}
\node at (17.400000\du,34.200000\du){address};
\pgfsetlinewidth{0.000000\du}
\pgfsetdash{}{0pt}
\pgfsetdash{}{0pt}
\pgfsetbuttcap
\pgfsetmiterjoin
\pgfsetlinewidth{0.000000\du}
\pgfsetbuttcap
\pgfsetmiterjoin
\pgfsetdash{}{0pt}
\definecolor{dialinecolor}{rgb}{0.000000, 0.000000, 0.000000}
\pgfsetstrokecolor{dialinecolor}
\draw (16.000000\du,40.000000\du)--(20.000000\du,40.000000\du)--(20.000000\du,42.000000\du)--(16.000000\du,42.000000\du)--cycle;
\pgfsetbuttcap
\pgfsetmiterjoin
\pgfsetdash{}{0pt}
\definecolor{dialinecolor}{rgb}{0.000000, 0.000000, 0.000000}
\pgfsetstrokecolor{dialinecolor}
\draw (16.000000\du,40.600000\du)--(16.800000\du,41.000000\du);
\pgfsetbuttcap
\pgfsetmiterjoin
\pgfsetdash{}{0pt}
\definecolor{dialinecolor}{rgb}{0.000000, 0.000000, 0.000000}
\pgfsetstrokecolor{dialinecolor}
\draw (16.800000\du,41.000000\du)--(16.000000\du,41.400000\du);
% setfont left to latex
\definecolor{dialinecolor}{rgb}{0.000000, 0.000000, 0.000000}
\pgfsetstrokecolor{dialinecolor}
\node at (18.400000\du,41.200000\du){engine};
\pgfsetlinewidth{0.000000\du}
\pgfsetdash{}{0pt}
\pgfsetdash{}{0pt}
\pgfsetbuttcap
\pgfsetmiterjoin
\pgfsetlinewidth{0.000000\du}
\pgfsetbuttcap
\pgfsetmiterjoin
\pgfsetdash{}{0pt}
\definecolor{dialinecolor}{rgb}{0.000000, 0.000000, 0.000000}
\pgfsetstrokecolor{dialinecolor}
\draw (15.500000\du,36.500000\du)--(19.500000\du,36.500000\du)--(19.500000\du,38.500000\du)--(15.500000\du,38.500000\du)--cycle;
\pgfsetbuttcap
\pgfsetmiterjoin
\pgfsetdash{}{0pt}
\definecolor{dialinecolor}{rgb}{0.000000, 0.000000, 0.000000}
\pgfsetstrokecolor{dialinecolor}
\draw (15.500000\du,37.100000\du)--(16.300000\du,37.500000\du);
\pgfsetbuttcap
\pgfsetmiterjoin
\pgfsetdash{}{0pt}
\definecolor{dialinecolor}{rgb}{0.000000, 0.000000, 0.000000}
\pgfsetstrokecolor{dialinecolor}
\draw (16.300000\du,37.500000\du)--(15.500000\du,37.900000\du);
% setfont left to latex
\definecolor{dialinecolor}{rgb}{0.000000, 0.000000, 0.000000}
\pgfsetstrokecolor{dialinecolor}
\node at (17.900000\du,37.700000\du){valid};
\pgfsetlinewidth{0.000000\du}
\pgfsetdash{}{0pt}
\pgfsetdash{}{0pt}
\pgfsetbuttcap
\pgfsetmiterjoin
\pgfsetlinewidth{0.000000\du}
\pgfsetbuttcap
\pgfsetmiterjoin
\pgfsetdash{}{0pt}
\definecolor{dialinecolor}{rgb}{0.000000, 0.000000, 0.000000}
\pgfsetstrokecolor{dialinecolor}
\draw (16.500000\du,43.500000\du)--(20.500000\du,43.500000\du)--(20.500000\du,45.500000\du)--(16.500000\du,45.500000\du)--cycle;
\pgfsetbuttcap
\pgfsetmiterjoin
\pgfsetdash{}{0pt}
\definecolor{dialinecolor}{rgb}{0.000000, 0.000000, 0.000000}
\pgfsetstrokecolor{dialinecolor}
\draw (16.500000\du,44.100000\du)--(17.300000\du,44.500000\du);
\pgfsetbuttcap
\pgfsetmiterjoin
\pgfsetdash{}{0pt}
\definecolor{dialinecolor}{rgb}{0.000000, 0.000000, 0.000000}
\pgfsetstrokecolor{dialinecolor}
\draw (17.300000\du,44.500000\du)--(16.500000\du,44.900000\du);
% setfont left to latex
\definecolor{dialinecolor}{rgb}{0.000000, 0.000000, 0.000000}
\pgfsetstrokecolor{dialinecolor}
\node at (18.900000\du,44.700000\du){tag};
\pgfsetlinewidth{0.000000\du}
\pgfsetdash{}{0pt}
\pgfsetdash{}{0pt}
\pgfsetmiterjoin
\definecolor{dialinecolor}{rgb}{0.000000, 0.000000, 0.000000}
\pgfsetstrokecolor{dialinecolor}
\draw (18.000000\du,25.000000\du)--(18.000000\du,31.500000\du)--(26.000000\du,31.500000\du)--(26.000000\du,25.000000\du)--cycle;
% setfont left to latex
\definecolor{dialinecolor}{rgb}{0.000000, 0.000000, 0.000000}
\pgfsetstrokecolor{dialinecolor}
\node at (22.000000\du,27.671150\du){BRAM};
% setfont left to latex
\definecolor{dialinecolor}{rgb}{0.000000, 0.000000, 0.000000}
\pgfsetstrokecolor{dialinecolor}
\node at (22.000000\du,28.471250\du){};
% setfont left to latex
\definecolor{dialinecolor}{rgb}{0.000000, 0.000000, 0.000000}
\pgfsetstrokecolor{dialinecolor}
\node at (22.000000\du,29.271350\du){WIDTH x DEPTH};
\pgfsetlinewidth{0.000000\du}
\pgfsetdash{}{0pt}
\pgfsetdash{}{0pt}
\pgfsetmiterjoin
\pgfsetbuttcap
{
\definecolor{dialinecolor}{rgb}{0.000000, 0.000000, 0.000000}
\pgfsetfillcolor{dialinecolor}
% was here!!!
\pgfsetarrowsend{latex}
{\pgfsetcornersarced{\pgfpoint{0.000000\du}{0.000000\du}}\definecolor{dialinecolor}{rgb}{0.000000, 0.000000, 0.000000}
\pgfsetstrokecolor{dialinecolor}
\draw (17.000000\du,33.000000\du)--(17.000000\du,32.500000\du)--(22.000000\du,32.500000\du)--(22.000000\du,31.500000\du);
}}
\pgfsetlinewidth{0.000000\du}
\pgfsetdash{}{0pt}
\pgfsetdash{}{0pt}
\pgfsetmiterjoin
\pgfsetbuttcap
{
\definecolor{dialinecolor}{rgb}{0.000000, 0.000000, 0.000000}
\pgfsetfillcolor{dialinecolor}
% was here!!!
\pgfsetarrowsend{latex}
{\pgfsetcornersarced{\pgfpoint{0.000000\du}{0.000000\du}}\definecolor{dialinecolor}{rgb}{0.000000, 0.000000, 0.000000}
\pgfsetstrokecolor{dialinecolor}
\draw (18.000000\du,40.000000\du)--(18.000000\du,39.500000\du)--(22.000000\du,39.500000\du)--(22.000000\du,31.500000\du);
}}
\pgfsetlinewidth{0.000000\du}
\pgfsetdash{}{0pt}
\pgfsetdash{}{0pt}
\pgfsetmiterjoin
\pgfsetbuttcap
{
\definecolor{dialinecolor}{rgb}{0.000000, 0.000000, 0.000000}
\pgfsetfillcolor{dialinecolor}
% was here!!!
\pgfsetarrowsend{latex}
{\pgfsetcornersarced{\pgfpoint{0.000000\du}{0.000000\du}}\definecolor{dialinecolor}{rgb}{0.000000, 0.000000, 0.000000}
\pgfsetstrokecolor{dialinecolor}
\draw (18.500000\du,43.500000\du)--(18.500000\du,43.000000\du)--(24.000000\du,43.000000\du)--(24.000000\du,31.500000\du);
}}
\pgfsetlinewidth{0.000000\du}
\pgfsetdash{}{0pt}
\pgfsetdash{}{0pt}
\pgfsetmiterjoin
\pgfsetbuttcap
{
\definecolor{dialinecolor}{rgb}{0.000000, 0.000000, 0.000000}
\pgfsetfillcolor{dialinecolor}
% was here!!!
\pgfsetarrowsend{latex}
{\pgfsetcornersarced{\pgfpoint{0.000000\du}{0.000000\du}}\definecolor{dialinecolor}{rgb}{0.000000, 0.000000, 0.000000}
\pgfsetstrokecolor{dialinecolor}
\draw (19.000000\du,47.000000\du)--(19.000000\du,46.500000\du)--(22.000000\du,46.500000\du)--(22.000000\du,31.500000\du);
}}
\pgfsetlinewidth{0.000000\du}
\pgfsetdash{}{0pt}
\pgfsetdash{}{0pt}
\pgfsetbuttcap
\pgfsetmiterjoin
\pgfsetlinewidth{0.000000\du}
\pgfsetbuttcap
\pgfsetmiterjoin
\pgfsetdash{}{0pt}
\pgfsetbuttcap
\pgfsetmiterjoin
\pgfsetdash{}{0pt}
\definecolor{dialinecolor}{rgb}{0.000000, 0.000000, 0.000000}
\pgfsetfillcolor{dialinecolor}
\pgfpathellipse{\pgfpoint{22.000000\du}{32.500000\du}}{\pgfpoint{0.125000\du}{0\du}}{\pgfpoint{0\du}{0.125000\du}}
\pgfusepath{fill}
\definecolor{dialinecolor}{rgb}{0.000000, 0.000000, 0.000000}
\pgfsetstrokecolor{dialinecolor}
\pgfpathellipse{\pgfpoint{22.000000\du}{32.500000\du}}{\pgfpoint{0.125000\du}{0\du}}{\pgfpoint{0\du}{0.125000\du}}
\pgfusepath{stroke}
\pgfsetlinewidth{0.000000\du}
\pgfsetdash{}{0pt}
\pgfsetdash{}{0pt}
\pgfsetbuttcap
\pgfsetmiterjoin
\pgfsetlinewidth{0.000000\du}
\pgfsetbuttcap
\pgfsetmiterjoin
\pgfsetdash{}{0pt}
\pgfsetbuttcap
\pgfsetmiterjoin
\pgfsetdash{}{0pt}
\definecolor{dialinecolor}{rgb}{0.000000, 0.000000, 0.000000}
\pgfsetfillcolor{dialinecolor}
\pgfpathellipse{\pgfpoint{22.000000\du}{39.500000\du}}{\pgfpoint{0.125000\du}{0\du}}{\pgfpoint{0\du}{0.125000\du}}
\pgfusepath{fill}
\definecolor{dialinecolor}{rgb}{0.000000, 0.000000, 0.000000}
\pgfsetstrokecolor{dialinecolor}
\pgfpathellipse{\pgfpoint{22.000000\du}{39.500000\du}}{\pgfpoint{0.125000\du}{0\du}}{\pgfpoint{0\du}{0.125000\du}}
\pgfusepath{stroke}
\pgfsetlinewidth{0.000000\du}
\pgfsetdash{}{0pt}
\pgfsetdash{}{0pt}
\pgfsetbuttcap
{
\definecolor{dialinecolor}{rgb}{0.000000, 0.000000, 0.000000}
\pgfsetfillcolor{dialinecolor}
% was here!!!
\definecolor{dialinecolor}{rgb}{0.000000, 0.000000, 0.000000}
\pgfsetstrokecolor{dialinecolor}
\draw (18.000000\du,31.500000\du)--(26.000000\du,31.500000\du);
}
\pgfsetlinewidth{0.000000\du}
\pgfsetdash{}{0pt}
\pgfsetdash{}{0pt}
\pgfsetmiterjoin
\pgfsetbuttcap
{
\definecolor{dialinecolor}{rgb}{0.000000, 0.000000, 0.000000}
\pgfsetfillcolor{dialinecolor}
% was here!!!
\pgfsetarrowsend{latex}
{\pgfsetcornersarced{\pgfpoint{0.000000\du}{0.000000\du}}\definecolor{dialinecolor}{rgb}{0.000000, 0.000000, 0.000000}
\pgfsetstrokecolor{dialinecolor}
\draw (17.500000\du,36.500000\du)--(17.500000\du,36.000000\du)--(20.000000\du,36.000000\du)--(20.000000\du,31.500000\du);
}}
% setfont left to latex
\definecolor{dialinecolor}{rgb}{0.000000, 0.000000, 0.000000}
\pgfsetstrokecolor{dialinecolor}
\node[anchor=west] at (24.500000\du,32.595000\du){W Port A};
% setfont left to latex
\definecolor{dialinecolor}{rgb}{0.000000, 0.000000, 0.000000}
\pgfsetstrokecolor{dialinecolor}
\node at (24.000000\du,31.105000\du){waddr};
% setfont left to latex
\definecolor{dialinecolor}{rgb}{0.000000, 0.000000, 0.000000}
\pgfsetstrokecolor{dialinecolor}
\node at (22.000000\du,31.105000\du){wdata};
\pgfsetlinewidth{0.000000\du}
\pgfsetdash{}{0pt}
\pgfsetdash{}{0pt}
\pgfsetmiterjoin
\pgfsetbuttcap
{
\definecolor{dialinecolor}{rgb}{0.000000, 0.000000, 0.000000}
\pgfsetfillcolor{dialinecolor}
% was here!!!
\pgfsetarrowsend{latex}
{\pgfsetcornersarced{\pgfpoint{0.000000\du}{0.000000\du}}\definecolor{dialinecolor}{rgb}{0.000000, 0.000000, 0.000000}
\pgfsetstrokecolor{dialinecolor}
\draw (14.500000\du,36.000000\du)--(14.500000\du,36.000000\du)--(17.000000\du,36.000000\du)--(17.000000\du,35.000000\du);
}}
\pgfsetlinewidth{0.000000\du}
\pgfsetdash{}{0pt}
\pgfsetdash{}{0pt}
\pgfsetmiterjoin
\pgfsetbuttcap
{
\definecolor{dialinecolor}{rgb}{0.000000, 0.000000, 0.000000}
\pgfsetfillcolor{dialinecolor}
% was here!!!
\pgfsetarrowsend{latex}
{\pgfsetcornersarced{\pgfpoint{0.000000\du}{0.000000\du}}\definecolor{dialinecolor}{rgb}{0.000000, 0.000000, 0.000000}
\pgfsetstrokecolor{dialinecolor}
\draw (14.500000\du,39.500000\du)--(14.500000\du,39.500000\du)--(17.500000\du,39.500000\du)--(17.500000\du,38.500000\du);
}}
\pgfsetlinewidth{0.000000\du}
\pgfsetdash{}{0pt}
\pgfsetdash{}{0pt}
\pgfsetmiterjoin
\pgfsetbuttcap
{
\definecolor{dialinecolor}{rgb}{0.000000, 0.000000, 0.000000}
\pgfsetfillcolor{dialinecolor}
% was here!!!
\pgfsetarrowsend{latex}
{\pgfsetcornersarced{\pgfpoint{0.000000\du}{0.000000\du}}\definecolor{dialinecolor}{rgb}{0.000000, 0.000000, 0.000000}
\pgfsetstrokecolor{dialinecolor}
\draw (14.500000\du,43.000000\du)--(14.500000\du,43.000000\du)--(18.000000\du,43.000000\du)--(18.000000\du,42.000000\du);
}}
\pgfsetlinewidth{0.000000\du}
\pgfsetdash{}{0pt}
\pgfsetdash{}{0pt}
\pgfsetmiterjoin
\pgfsetbuttcap
{
\definecolor{dialinecolor}{rgb}{0.000000, 0.000000, 0.000000}
\pgfsetfillcolor{dialinecolor}
% was here!!!
\pgfsetarrowsend{latex}
{\pgfsetcornersarced{\pgfpoint{0.000000\du}{0.000000\du}}\definecolor{dialinecolor}{rgb}{0.000000, 0.000000, 0.000000}
\pgfsetstrokecolor{dialinecolor}
\draw (14.500000\du,46.500000\du)--(14.500000\du,46.500000\du)--(18.500000\du,46.500000\du)--(18.500000\du,45.500000\du);
}}
\pgfsetlinewidth{0.000000\du}
\pgfsetdash{}{0pt}
\pgfsetdash{}{0pt}
\pgfsetmiterjoin
\pgfsetbuttcap
{
\definecolor{dialinecolor}{rgb}{0.000000, 0.000000, 0.000000}
\pgfsetfillcolor{dialinecolor}
% was here!!!
\pgfsetarrowsend{latex}
{\pgfsetcornersarced{\pgfpoint{0.000000\du}{0.000000\du}}\definecolor{dialinecolor}{rgb}{0.000000, 0.000000, 0.000000}
\pgfsetstrokecolor{dialinecolor}
\draw (14.500000\du,50.000000\du)--(14.500000\du,50.000000\du)--(19.000000\du,50.000000\du)--(19.000000\du,49.000000\du);
}}
% setfont left to latex
\definecolor{dialinecolor}{rgb}{0.000000, 0.000000, 0.000000}
\pgfsetstrokecolor{dialinecolor}
\node[anchor=east] at (14.500000\du,44.347500\du){From agen};
\pgfsetlinewidth{0.000000\du}
\pgfsetdash{}{0pt}
\pgfsetdash{}{0pt}
\pgfsetbuttcap
\pgfsetmiterjoin
\pgfsetlinewidth{0.000000\du}
\pgfsetbuttcap
\pgfsetmiterjoin
\pgfsetdash{}{0pt}
\definecolor{dialinecolor}{rgb}{0.000000, 0.000000, 0.000000}
\pgfsetstrokecolor{dialinecolor}
\draw (34.000000\du,47.000000\du)--(40.000000\du,47.000000\du)--(40.000000\du,49.000000\du)--(34.000000\du,49.000000\du)--cycle;
\pgfsetbuttcap
\pgfsetmiterjoin
\pgfsetdash{}{0pt}
\definecolor{dialinecolor}{rgb}{0.000000, 0.000000, 0.000000}
\pgfsetstrokecolor{dialinecolor}
\draw (34.000000\du,47.600000\du)--(35.200000\du,48.000000\du);
\pgfsetbuttcap
\pgfsetmiterjoin
\pgfsetdash{}{0pt}
\definecolor{dialinecolor}{rgb}{0.000000, 0.000000, 0.000000}
\pgfsetstrokecolor{dialinecolor}
\draw (35.200000\du,48.000000\du)--(34.000000\du,48.400000\du);
% setfont left to latex
\definecolor{dialinecolor}{rgb}{0.000000, 0.000000, 0.000000}
\pgfsetstrokecolor{dialinecolor}
\node at (37.600000\du,48.200000\du){response};
\pgfsetlinewidth{0.000000\du}
\pgfsetdash{}{0pt}
\pgfsetdash{}{0pt}
\pgfsetbuttcap
\pgfsetmiterjoin
\pgfsetlinewidth{0.000000\du}
\pgfsetbuttcap
\pgfsetmiterjoin
\pgfsetdash{}{0pt}
\definecolor{dialinecolor}{rgb}{0.000000, 0.000000, 0.000000}
\pgfsetstrokecolor{dialinecolor}
\draw (25.000000\du,38.000000\du)--(29.000000\du,38.000000\du)--(29.000000\du,40.000000\du)--(25.000000\du,40.000000\du)--cycle;
\pgfsetbuttcap
\pgfsetmiterjoin
\pgfsetdash{}{0pt}
\definecolor{dialinecolor}{rgb}{0.000000, 0.000000, 0.000000}
\pgfsetstrokecolor{dialinecolor}
\draw (25.000000\du,38.600000\du)--(25.800000\du,39.000000\du);
\pgfsetbuttcap
\pgfsetmiterjoin
\pgfsetdash{}{0pt}
\definecolor{dialinecolor}{rgb}{0.000000, 0.000000, 0.000000}
\pgfsetstrokecolor{dialinecolor}
\draw (25.800000\du,39.000000\du)--(25.000000\du,39.400000\du);
% setfont left to latex
\definecolor{dialinecolor}{rgb}{0.000000, 0.000000, 0.000000}
\pgfsetstrokecolor{dialinecolor}
\node at (27.400000\du,39.200000\du){data};
\pgfsetlinewidth{0.000000\du}
\pgfsetdash{}{0pt}
\pgfsetdash{}{0pt}
\pgfsetbuttcap
\pgfsetmiterjoin
\pgfsetlinewidth{0.000000\du}
\pgfsetbuttcap
\pgfsetmiterjoin
\pgfsetdash{}{0pt}
\definecolor{dialinecolor}{rgb}{0.000000, 0.000000, 0.000000}
\pgfsetstrokecolor{dialinecolor}
\draw (28.000000\du,41.000000\du)--(32.000000\du,41.000000\du)--(32.000000\du,43.000000\du)--(28.000000\du,43.000000\du)--cycle;
\pgfsetbuttcap
\pgfsetmiterjoin
\pgfsetdash{}{0pt}
\definecolor{dialinecolor}{rgb}{0.000000, 0.000000, 0.000000}
\pgfsetstrokecolor{dialinecolor}
\draw (28.000000\du,41.600000\du)--(28.800000\du,42.000000\du);
\pgfsetbuttcap
\pgfsetmiterjoin
\pgfsetdash{}{0pt}
\definecolor{dialinecolor}{rgb}{0.000000, 0.000000, 0.000000}
\pgfsetstrokecolor{dialinecolor}
\draw (28.800000\du,42.000000\du)--(28.000000\du,42.400000\du);
% setfont left to latex
\definecolor{dialinecolor}{rgb}{0.000000, 0.000000, 0.000000}
\pgfsetstrokecolor{dialinecolor}
\node at (30.400000\du,42.200000\du){valid};
\pgfsetlinewidth{0.000000\du}
\pgfsetdash{}{0pt}
\pgfsetdash{}{0pt}
\pgfsetbuttcap
\pgfsetmiterjoin
\pgfsetlinewidth{0.000000\du}
\pgfsetbuttcap
\pgfsetmiterjoin
\pgfsetdash{}{0pt}
\definecolor{dialinecolor}{rgb}{0.000000, 0.000000, 0.000000}
\pgfsetstrokecolor{dialinecolor}
\draw (31.000000\du,44.000000\du)--(35.000000\du,44.000000\du)--(35.000000\du,46.000000\du)--(31.000000\du,46.000000\du)--cycle;
\pgfsetbuttcap
\pgfsetmiterjoin
\pgfsetdash{}{0pt}
\definecolor{dialinecolor}{rgb}{0.000000, 0.000000, 0.000000}
\pgfsetstrokecolor{dialinecolor}
\draw (31.000000\du,44.600000\du)--(31.800000\du,45.000000\du);
\pgfsetbuttcap
\pgfsetmiterjoin
\pgfsetdash{}{0pt}
\definecolor{dialinecolor}{rgb}{0.000000, 0.000000, 0.000000}
\pgfsetstrokecolor{dialinecolor}
\draw (31.800000\du,45.000000\du)--(31.000000\du,45.400000\du);
% setfont left to latex
\definecolor{dialinecolor}{rgb}{0.000000, 0.000000, 0.000000}
\pgfsetstrokecolor{dialinecolor}
\node at (33.400000\du,45.200000\du){tag};
\pgfsetlinewidth{0.000000\du}
\pgfsetdash{}{0pt}
\pgfsetdash{}{0pt}
\pgfsetmiterjoin
\pgfsetbuttcap
{
\definecolor{dialinecolor}{rgb}{0.000000, 0.000000, 0.000000}
\pgfsetfillcolor{dialinecolor}
% was here!!!
\pgfsetarrowsend{latex}
{\pgfsetcornersarced{\pgfpoint{0.000000\du}{0.000000\du}}\definecolor{dialinecolor}{rgb}{0.000000, 0.000000, 0.000000}
\pgfsetstrokecolor{dialinecolor}
\draw (27.000000\du,50.000000\du)--(27.000000\du,48.000000\du)--(27.000000\du,48.000000\du)--(27.000000\du,40.000000\du);
}}
\pgfsetlinewidth{0.000000\du}
\pgfsetdash{}{0pt}
\pgfsetdash{}{0pt}
\pgfsetmiterjoin
\pgfsetbuttcap
{
\definecolor{dialinecolor}{rgb}{0.000000, 0.000000, 0.000000}
\pgfsetfillcolor{dialinecolor}
% was here!!!
\pgfsetarrowsend{latex}
{\pgfsetcornersarced{\pgfpoint{0.000000\du}{0.000000\du}}\definecolor{dialinecolor}{rgb}{0.000000, 0.000000, 0.000000}
\pgfsetstrokecolor{dialinecolor}
\draw (30.000000\du,50.000000\du)--(30.000000\du,48.000000\du)--(30.000000\du,48.000000\du)--(30.000000\du,43.000000\du);
}}
\pgfsetlinewidth{0.000000\du}
\pgfsetdash{}{0pt}
\pgfsetdash{}{0pt}
\pgfsetmiterjoin
\pgfsetbuttcap
{
\definecolor{dialinecolor}{rgb}{0.000000, 0.000000, 0.000000}
\pgfsetfillcolor{dialinecolor}
% was here!!!
\pgfsetarrowsend{latex}
{\pgfsetcornersarced{\pgfpoint{0.000000\du}{0.000000\du}}\definecolor{dialinecolor}{rgb}{0.000000, 0.000000, 0.000000}
\pgfsetstrokecolor{dialinecolor}
\draw (33.000000\du,50.000000\du)--(33.000000\du,49.000000\du)--(33.000000\du,49.000000\du)--(33.000000\du,46.000000\du);
}}
\pgfsetlinewidth{0.000000\du}
\pgfsetdash{}{0pt}
\pgfsetdash{}{0pt}
\pgfsetmiterjoin
\pgfsetbuttcap
{
\definecolor{dialinecolor}{rgb}{0.000000, 0.000000, 0.000000}
\pgfsetfillcolor{dialinecolor}
% was here!!!
\pgfsetarrowsend{latex}
{\pgfsetcornersarced{\pgfpoint{0.000000\du}{0.000000\du}}\definecolor{dialinecolor}{rgb}{0.000000, 0.000000, 0.000000}
\pgfsetstrokecolor{dialinecolor}
\draw (37.000000\du,50.000000\du)--(37.000000\du,50.000000\du)--(37.000000\du,50.000000\du)--(37.000000\du,49.000000\du);
}}
% setfont left to latex
\definecolor{dialinecolor}{rgb}{0.000000, 0.000000, 0.000000}
\pgfsetstrokecolor{dialinecolor}
\node at (31.500000\du,51.095000\du){From axi};
% setfont left to latex
\definecolor{dialinecolor}{rgb}{0.000000, 0.000000, 0.000000}
\pgfsetstrokecolor{dialinecolor}
\node at (20.000000\du,31.081875\du){wen};
% setfont left to latex
\definecolor{dialinecolor}{rgb}{0.000000, 0.000000, 0.000000}
\pgfsetstrokecolor{dialinecolor}
\node at (23.333333\du,25.367500\du){raddr};
% setfont left to latex
\definecolor{dialinecolor}{rgb}{0.000000, 0.000000, 0.000000}
\pgfsetstrokecolor{dialinecolor}
\node at (20.666667\du,25.367500\du){rdata};
\pgfsetlinewidth{0.000000\du}
\pgfsetdash{}{0pt}
\pgfsetdash{}{0pt}
\pgfsetbuttcap
{
\definecolor{dialinecolor}{rgb}{0.000000, 0.000000, 0.000000}
\pgfsetfillcolor{dialinecolor}
% was here!!!
\definecolor{dialinecolor}{rgb}{0.000000, 0.000000, 0.000000}
\pgfsetstrokecolor{dialinecolor}
\draw (18.000000\du,25.000000\du)--(26.000000\du,25.000000\du);
}
% setfont left to latex
\definecolor{dialinecolor}{rgb}{0.000000, 0.000000, 0.000000}
\pgfsetstrokecolor{dialinecolor}
\node[anchor=east] at (20.000000\du,24.347500\du){R Port B};
\pgfsetlinewidth{0.000000\du}
\pgfsetdash{}{0pt}
\pgfsetdash{}{0pt}
\pgfsetmiterjoin
\pgfsetbuttcap
{
\definecolor{dialinecolor}{rgb}{0.000000, 0.000000, 0.000000}
\pgfsetfillcolor{dialinecolor}
% was here!!!
\pgfsetarrowsend{latex}
{\pgfsetcornersarced{\pgfpoint{0.000000\du}{0.000000\du}}\definecolor{dialinecolor}{rgb}{0.000000, 0.000000, 0.000000}
\pgfsetstrokecolor{dialinecolor}
\draw (33.000000\du,44.000000\du)--(33.000000\du,24.000000\du)--(23.333333\du,24.000000\du)--(23.333333\du,25.000000\du);
}}
\pgfsetlinewidth{0.000000\du}
\pgfsetdash{}{0pt}
\pgfsetdash{}{0pt}
\pgfsetmiterjoin
\definecolor{dialinecolor}{rgb}{0.000000, 0.000000, 0.000000}
\pgfsetstrokecolor{dialinecolor}
\draw (19.000000\du,20.000000\du)--(19.000000\du,23.000000\du)--(25.000000\du,23.000000\du)--(25.000000\du,20.000000\du)--cycle;
% setfont left to latex
\definecolor{dialinecolor}{rgb}{0.000000, 0.000000, 0.000000}
\pgfsetstrokecolor{dialinecolor}
\node at (22.000000\du,21.721250\du){fbist\_dgen};
\pgfsetlinewidth{0.000000\du}
\pgfsetdash{}{0pt}
\pgfsetdash{}{0pt}
\pgfsetmiterjoin
\pgfsetbuttcap
{
\definecolor{dialinecolor}{rgb}{0.000000, 0.000000, 0.000000}
\pgfsetfillcolor{dialinecolor}
% was here!!!
\pgfsetarrowsend{latex}
{\pgfsetcornersarced{\pgfpoint{0.000000\du}{0.000000\du}}\definecolor{dialinecolor}{rgb}{0.000000, 0.000000, 0.000000}
\pgfsetstrokecolor{dialinecolor}
\draw (20.666667\du,25.000000\du)--(20.666667\du,24.000000\du)--(22.000000\du,24.000000\du)--(22.000000\du,23.000000\du);
}}
\pgfsetlinewidth{0.000000\du}
\pgfsetdash{}{0pt}
\pgfsetdash{}{0pt}
\pgfsetbuttcap
\pgfsetmiterjoin
\pgfsetlinewidth{0.000000\du}
\pgfsetbuttcap
\pgfsetmiterjoin
\pgfsetdash{}{0pt}
\definecolor{dialinecolor}{rgb}{0.000000, 0.000000, 0.000000}
\pgfsetstrokecolor{dialinecolor}
\draw (35.000000\du,34.000000\du)--(38.000000\du,34.000000\du)--(38.000000\du,35.000000\du)--(35.000000\du,35.000000\du)--cycle;
\pgfsetbuttcap
\pgfsetmiterjoin
\pgfsetdash{}{0pt}
\definecolor{dialinecolor}{rgb}{0.000000, 0.000000, 0.000000}
\pgfsetstrokecolor{dialinecolor}
\draw (35.000000\du,34.300000\du)--(35.600000\du,34.500000\du);
\pgfsetbuttcap
\pgfsetmiterjoin
\pgfsetdash{}{0pt}
\definecolor{dialinecolor}{rgb}{0.000000, 0.000000, 0.000000}
\pgfsetstrokecolor{dialinecolor}
\draw (35.600000\du,34.500000\du)--(35.000000\du,34.700000\du);
% setfont left to latex
\definecolor{dialinecolor}{rgb}{0.000000, 0.000000, 0.000000}
\pgfsetstrokecolor{dialinecolor}
\node at (36.800000\du,34.700000\du){};
\pgfsetlinewidth{0.000000\du}
\pgfsetdash{}{0pt}
\pgfsetdash{}{0pt}
\pgfsetmiterjoin
\pgfsetbuttcap
{
\definecolor{dialinecolor}{rgb}{0.000000, 0.000000, 0.000000}
\pgfsetfillcolor{dialinecolor}
% was here!!!
\pgfsetarrowsend{latex}
{\pgfsetcornersarced{\pgfpoint{0.000000\du}{0.000000\du}}\definecolor{dialinecolor}{rgb}{0.000000, 0.000000, 0.000000}
\pgfsetstrokecolor{dialinecolor}
\draw (27.000000\du,38.000000\du)--(27.000000\du,37.000000\du)--(36.500000\du,37.000000\du)--(36.500000\du,35.000000\du);
}}
\pgfsetlinewidth{0.000000\du}
\pgfsetdash{}{0pt}
\pgfsetdash{}{0pt}
\pgfsetmiterjoin
\pgfsetbuttcap
{
\definecolor{dialinecolor}{rgb}{0.000000, 0.000000, 0.000000}
\pgfsetfillcolor{dialinecolor}
% was here!!!
\pgfsetarrowsend{latex}
{\pgfsetcornersarced{\pgfpoint{0.000000\du}{0.000000\du}}\definecolor{dialinecolor}{rgb}{0.000000, 0.000000, 0.000000}
\pgfsetstrokecolor{dialinecolor}
\draw (30.000000\du,41.000000\du)--(30.000000\du,40.000000\du)--(36.500000\du,40.000000\du)--(36.500000\du,35.000000\du);
}}
\pgfsetlinewidth{0.000000\du}
\pgfsetdash{}{0pt}
\pgfsetdash{}{0pt}
\pgfsetmiterjoin
\pgfsetbuttcap
{
\definecolor{dialinecolor}{rgb}{0.000000, 0.000000, 0.000000}
\pgfsetfillcolor{dialinecolor}
% was here!!!
\pgfsetarrowsend{latex}
{\pgfsetcornersarced{\pgfpoint{0.000000\du}{0.000000\du}}\definecolor{dialinecolor}{rgb}{0.000000, 0.000000, 0.000000}
\pgfsetstrokecolor{dialinecolor}
\draw (33.000000\du,44.000000\du)--(33.000000\du,43.000000\du)--(36.500000\du,43.000000\du)--(36.500000\du,35.000000\du);
}}
\pgfsetlinewidth{0.000000\du}
\pgfsetdash{}{0pt}
\pgfsetdash{}{0pt}
\pgfsetmiterjoin
\pgfsetbuttcap
{
\definecolor{dialinecolor}{rgb}{0.000000, 0.000000, 0.000000}
\pgfsetfillcolor{dialinecolor}
% was here!!!
\pgfsetarrowsend{latex}
{\pgfsetcornersarced{\pgfpoint{0.000000\du}{0.000000\du}}\definecolor{dialinecolor}{rgb}{0.000000, 0.000000, 0.000000}
\pgfsetstrokecolor{dialinecolor}
\draw (37.000000\du,47.000000\du)--(37.000000\du,46.000000\du)--(36.500000\du,46.000000\du)--(36.500000\du,35.000000\du);
}}
\pgfsetlinewidth{0.000000\du}
\pgfsetdash{}{0pt}
\pgfsetdash{}{0pt}
\pgfsetbuttcap
\pgfsetmiterjoin
\pgfsetlinewidth{0.000000\du}
\pgfsetbuttcap
\pgfsetmiterjoin
\pgfsetdash{}{0pt}
\pgfsetbuttcap
\pgfsetmiterjoin
\pgfsetdash{}{0pt}
\definecolor{dialinecolor}{rgb}{0.000000, 0.000000, 0.000000}
\pgfsetfillcolor{dialinecolor}
\pgfpathellipse{\pgfpoint{36.500000\du}{37.000000\du}}{\pgfpoint{0.125000\du}{0\du}}{\pgfpoint{0\du}{0.125000\du}}
\pgfusepath{fill}
\definecolor{dialinecolor}{rgb}{0.000000, 0.000000, 0.000000}
\pgfsetstrokecolor{dialinecolor}
\pgfpathellipse{\pgfpoint{36.500000\du}{37.000000\du}}{\pgfpoint{0.125000\du}{0\du}}{\pgfpoint{0\du}{0.125000\du}}
\pgfusepath{stroke}
\pgfsetlinewidth{0.000000\du}
\pgfsetdash{}{0pt}
\pgfsetdash{}{0pt}
\pgfsetbuttcap
\pgfsetmiterjoin
\pgfsetlinewidth{0.000000\du}
\pgfsetbuttcap
\pgfsetmiterjoin
\pgfsetdash{}{0pt}
\pgfsetbuttcap
\pgfsetmiterjoin
\pgfsetdash{}{0pt}
\definecolor{dialinecolor}{rgb}{0.000000, 0.000000, 0.000000}
\pgfsetfillcolor{dialinecolor}
\pgfpathellipse{\pgfpoint{36.500000\du}{40.000000\du}}{\pgfpoint{0.125000\du}{0\du}}{\pgfpoint{0\du}{0.125000\du}}
\pgfusepath{fill}
\definecolor{dialinecolor}{rgb}{0.000000, 0.000000, 0.000000}
\pgfsetstrokecolor{dialinecolor}
\pgfpathellipse{\pgfpoint{36.500000\du}{40.000000\du}}{\pgfpoint{0.125000\du}{0\du}}{\pgfpoint{0\du}{0.125000\du}}
\pgfusepath{stroke}
\pgfsetlinewidth{0.000000\du}
\pgfsetdash{}{0pt}
\pgfsetdash{}{0pt}
\pgfsetbuttcap
\pgfsetmiterjoin
\pgfsetlinewidth{0.000000\du}
\pgfsetbuttcap
\pgfsetmiterjoin
\pgfsetdash{}{0pt}
\pgfsetbuttcap
\pgfsetmiterjoin
\pgfsetdash{}{0pt}
\definecolor{dialinecolor}{rgb}{0.000000, 0.000000, 0.000000}
\pgfsetfillcolor{dialinecolor}
\pgfpathellipse{\pgfpoint{36.500000\du}{43.000000\du}}{\pgfpoint{0.125000\du}{0\du}}{\pgfpoint{0\du}{0.125000\du}}
\pgfusepath{fill}
\definecolor{dialinecolor}{rgb}{0.000000, 0.000000, 0.000000}
\pgfsetstrokecolor{dialinecolor}
\pgfpathellipse{\pgfpoint{36.500000\du}{43.000000\du}}{\pgfpoint{0.125000\du}{0\du}}{\pgfpoint{0\du}{0.125000\du}}
\pgfusepath{stroke}
\pgfsetlinewidth{0.000000\du}
\pgfsetdash{}{0pt}
\pgfsetdash{}{0pt}
\pgfsetbuttcap
\pgfsetmiterjoin
\pgfsetlinewidth{0.000000\du}
\pgfsetbuttcap
\pgfsetmiterjoin
\pgfsetdash{}{0pt}
\definecolor{dialinecolor}{rgb}{0.000000, 0.000000, 0.000000}
\pgfsetstrokecolor{dialinecolor}
\draw (35.000000\du,31.000000\du)--(38.000000\du,31.000000\du)--(38.000000\du,32.000000\du)--(35.000000\du,32.000000\du)--cycle;
\pgfsetbuttcap
\pgfsetmiterjoin
\pgfsetdash{}{0pt}
\definecolor{dialinecolor}{rgb}{0.000000, 0.000000, 0.000000}
\pgfsetstrokecolor{dialinecolor}
\draw (35.000000\du,31.300000\du)--(35.600000\du,31.500000\du);
\pgfsetbuttcap
\pgfsetmiterjoin
\pgfsetdash{}{0pt}
\definecolor{dialinecolor}{rgb}{0.000000, 0.000000, 0.000000}
\pgfsetstrokecolor{dialinecolor}
\draw (35.600000\du,31.500000\du)--(35.000000\du,31.700000\du);
% setfont left to latex
\definecolor{dialinecolor}{rgb}{0.000000, 0.000000, 0.000000}
\pgfsetstrokecolor{dialinecolor}
\node at (36.800000\du,31.700000\du){};
\pgfsetlinewidth{0.000000\du}
\pgfsetdash{}{0pt}
\pgfsetdash{}{0pt}
\pgfsetmiterjoin
\pgfsetbuttcap
{
\definecolor{dialinecolor}{rgb}{0.000000, 0.000000, 0.000000}
\pgfsetfillcolor{dialinecolor}
% was here!!!
\pgfsetarrowsend{latex}
{\pgfsetcornersarced{\pgfpoint{0.000000\du}{0.000000\du}}\definecolor{dialinecolor}{rgb}{0.000000, 0.000000, 0.000000}
\pgfsetstrokecolor{dialinecolor}
\draw (36.500000\du,34.000000\du)--(36.500000\du,34.000000\du)--(36.500000\du,32.000000\du)--(36.500000\du,32.000000\du);
}}
\pgfsetlinewidth{0.000000\du}
\pgfsetdash{}{0pt}
\pgfsetdash{}{0pt}
\pgfsetbuttcap
\pgfsetmiterjoin
\pgfsetlinewidth{0.000000\du}
\pgfsetbuttcap
\pgfsetmiterjoin
\pgfsetdash{}{0pt}
\definecolor{dialinecolor}{rgb}{0.000000, 0.000000, 0.000000}
\pgfsetstrokecolor{dialinecolor}
\draw (35.000000\du,28.000000\du)--(38.000000\du,28.000000\du)--(38.000000\du,29.000000\du)--(35.000000\du,29.000000\du)--cycle;
\pgfsetbuttcap
\pgfsetmiterjoin
\pgfsetdash{}{0pt}
\definecolor{dialinecolor}{rgb}{0.000000, 0.000000, 0.000000}
\pgfsetstrokecolor{dialinecolor}
\draw (35.000000\du,28.300000\du)--(35.600000\du,28.500000\du);
\pgfsetbuttcap
\pgfsetmiterjoin
\pgfsetdash{}{0pt}
\definecolor{dialinecolor}{rgb}{0.000000, 0.000000, 0.000000}
\pgfsetstrokecolor{dialinecolor}
\draw (35.600000\du,28.500000\du)--(35.000000\du,28.700000\du);
% setfont left to latex
\definecolor{dialinecolor}{rgb}{0.000000, 0.000000, 0.000000}
\pgfsetstrokecolor{dialinecolor}
\node at (36.800000\du,28.700000\du){};
\pgfsetlinewidth{0.000000\du}
\pgfsetdash{}{0pt}
\pgfsetdash{}{0pt}
\pgfsetmiterjoin
\pgfsetbuttcap
{
\definecolor{dialinecolor}{rgb}{0.000000, 0.000000, 0.000000}
\pgfsetfillcolor{dialinecolor}
% was here!!!
\pgfsetarrowsend{latex}
{\pgfsetcornersarced{\pgfpoint{0.000000\du}{0.000000\du}}\definecolor{dialinecolor}{rgb}{0.000000, 0.000000, 0.000000}
\pgfsetstrokecolor{dialinecolor}
\draw (36.500000\du,31.000000\du)--(36.500000\du,31.000000\du)--(36.500000\du,29.000000\du)--(36.500000\du,29.000000\du);
}}
\end{tikzpicture}

  \end{center}
  \caption[Overview of fbist\_chk.vhdl]{\label{fig:fbist_chk}Overview of fbist\_chk.vhdl
  }
\end{figure}

\subsection{Error Logging}
Each checking failure generates an error object, which are stored in
an array for later processing. If more error storage is needed, these
error objects can be programmed to be stored out in the dedicated DDR4
ISDIMM as well, although if the error rate is high enough it's
possible that the ISDIMM memory controller will not keep up.

Each error object consists of the following items (as space allows):
\begin{itemize}
  \item Command
  \item Address
  \item Tag
  \item Generating Engine
  \item Expected Data can be worked out from the tag ? TBD
  \item Actual Data
  \item Timestamp
\end{itemize}

The error array stores error information when FBIST detects a bad or
invalid response to a read or write operation. The array can then be
accessed by clean-up code 4 bytes at a time using only one address.
Data is returned one quadbyte at a time
sequentially by reads to the the FBIST\_ERROR\_ARRAY\_DATA register. Each error requires
17 reads to assemble 68 bytes of information. The eighteenth read
will return the LS 4 bytes of the second oldest error and so on.

A second register supplies current read pointer information and the
number of valid error entries in the array (max 63).

Firmware can reset the array read pointer at any time by writing to
a third error aray register.

These regsters are described in the following subsection ("Registers").


In addition to error objects, overall error and success counts can be
observed.( - TBD ?)

\subsection{Registers}

\textbf{Note}: FBIST consists of multiple pools each with 3 engines.
Each engine gets unique configuration registers. In the tables
below, \$\{N\} enumerates through the configuration regsiters that are
identical for each engine in a pool, with $0 \leq \$\{N\} < 8$.

The following configurations available are a subset of what will be in
the final design. A currently missing configuration option does not
mean it won't be supported in the future.

The register addresses aren't final, and will most likely be
reorganized in the future. Any code written targeting these registers
should be capable of remaining flexible.

% FBIST Pool 0 Engine N
\subsubsection{FBIST Pool 0 Engine \$\{N\}}
\begin{tabular}{ r | p{350px} }
  Mnemonic       & FBIST\_POOL\_0\_ENGINE\_\$\{N\}             \\
  Address Offset & $4 * \$\{N\}$                               \\
  VHDL Name      & fbist\_cfg.fbist\_cfg\_regs.reg\_\$\{N\}\_q \\ \hline

  Description &
  Address and Command configuration for Engine \$\{N\} in Pool 0. \\
\end{tabular}
\\
\begin{tabularx}{\textwidth}{r|l|l|l|X}
  \hline
  Bit   & Field Mnemonic & Type & Reset & Description \\ \hline

  31:16 & Reserved       & RW   & 0     &

  Reserved 0. \\

  15:8  & ADDRESS\_MODE  & RW   & 0     &

  Addressing scheme to use when arbiter selects this engine.

  \setlength\parindent{24pt}
  \indent 0x00 : Constant Address                        \newline
  \indent 0x01 : Increment by one address alignment unit \newline
  \indent 0x02 : Decrement by one address alignment unit \newline
  \indent 0x03 : Use state of a 42-bit LFSR              \newline
  \indent All others Reserved                            \\

  7:0   & COMMAND\_MODE  & RW   & 0     &

  Command to send when arbiter selects this engine.

  \setlength\parindent{24pt}
  \indent 0x00 : No Operation   \newline
  \indent 0x01 : 128 Byte Write \newline
  \indent 0x02 : 64 Byte Write  \newline
  \indent 0x03 : 32 Byte Write  \newline
  \indent 0x04 : 128 Byte Read  \newline
  \indent 0x05 : 64 Byte Read   \newline
  \indent 0x06 : 32 Byte Read   \newline
  \indent All others Reserved   \\
\end{tabularx}

% FBIST Pool 0 Spacing
\subsubsection{FBIST Pool 0 Spacing}
\begin{tabular}{ r | p{350px} }
  Mnemonic       & FBIST\_POOL\_0\_SPACING                 \\
  Address Offset & \texttt{0x00020}                        \\
  VHDL Name      &  fbist\_cfg.fbist\_cfg\_regs.reg\_08\_q \\ \hline

  Description &
  Spacing configuration for commands in a pool. \\
\end{tabular}
\\
\begin{tabularx}{\textwidth}{r|l|l|l|X}
  \hline
  Bit   & Field Mnemonic  & Type & Reset & Description \\ \hline

  31:20 & Reserved        & RW   & 0     &

  Reserved 0. \\

  19:16 & SPACING\_SCHEME & RW   & 0     &

  Algorithm to use when spacing commands.

  \setlength\parindent{24pt}
  \indent 0x0 : Fixed Spacing \newline
  \indent 0x1 : 18-bit LFSR   \newline
  \indent All others Reserved \\

  15:0  & SPACING\_COUNT  & RW   & 0     &

  When in fixed spacing mode, this specifies the number of cycles to
  wait between valid commands. When in 18-bit LFSR mode, bits 2:0
  specify the number of sequential 1s in the least significant bits of
  the LFSR that must be seen before sending a valid command, with
  ``000'' meaning only the least significant bit must be '1', while
  ``111'' means the 8 least significant bits must be '1'. A higher
  value means commands are sent less frequently. In this mode, bits
  15:3 are unused.\\
\end{tabularx}

% FBIST Pool 0 Arbiter
\subsubsection{FBIST Pool 0 Arbiter}
\begin{tabular}{ r | p{350px} }
  Mnemonic       & FBIST\_POOL\_0\_ARB                     \\
  Address Offset & \texttt{0x00024}                        \\
  VHDL Name      &  fbist\_cfg.fbist\_cfg\_regs.reg\_09\_q \\ \hline

  Description &
  Abitration configuration for commands in a pool. \\
\end{tabular}
\\
\begin{tabularx}{\textwidth}{r|l|l|l|X}
  \hline
  Bit   & Field Mnemonic & Type & Reset & Description \\ \hline

  31:4  & Reserved       & RW   & 0     &

  Reserved 0. \\

  3:0   & ARB\_SCHEME    & RW   & 0     &

  Arbitration algorithm used for commands in a pool.

  \setlength\parindent{24pt}
  \indent 0x0 : Round Robin                     \newline
  \indent 0x1 : Lowest 3 bits of a   6-bit LFSR \newline
  \indent 0x2 : Lowest 3 bits of an 18-bit LFSR \newline
  \indent All others Reserved                   \\
\end{tabularx}

% FBIST Pool 0 Status
\subsubsection{FBIST Pool 0 Status}
\begin{tabular}{ r | p{350px} }
  Mnemonic       & FBIST\_POOL\_0\_STATUS                  \\
  Address Offset & \texttt{0x00028}                        \\
  VHDL Name      &  fbist\_cfg.fbist\_cfg\_regs.reg\_0A\_q \\ \hline

  Description &
  FBIST execution status. \\
\end{tabular}
\\
\begin{tabularx}{\textwidth}{r|l|l|l|X}
  \hline
  Bit   & Field Mnemonic      & Type & Reset & Description \\ \hline

  31    & STOP\_ON\_ERROR     & RW   & 0     &

  FBIST will stop if there is an error. Any outstanding commands will
  be completed, but no new commands will be generated. \\

  30    & NO\_ERROR\_IF\_HANG & RW   & 0     &

  If 1, no error is flagged if FBIST hangs. \\

  29:4  & Reserved            & RW   & 0     &

  Reserved 0. \\

  3     & DONE                & RWX  & 0     &

  Automatically set to 1 when hardware stops FBIST execution, such as
  if there was an error with stop on error enabled. Write 0 to
  clear. Not cleared automatically when a new run starts. \\

  2     & IP                  & ROX  & 0     &

  In progress. 1 when FBIST is running, 0 otherwise. \\

  1     & STOP                & RWX  & 0     &

  Write to 1 to stop execution. Self clears to 0. \\

  0     & START               & RWX  & 0     &

  Write to 1 to start execution. Self clears to 0. \\
\end{tabularx}

% FBIST Pool 0 Engine N Address Start Low
\subsubsection{FBIST Pool 0 Engine \$\{N\} Address Start Low}
\begin{tabular}{ r | p{350px} }
  Mnemonic       & FBIST\_POOL\_0\_ENGINE\_\$\{N\}\_ADDRESS\_START\_LOW     \\
  Address Offset & $4 * 2 * \$\{N\} + \texttt{0x2C}$                        \\
  VHDL Name      &  fbist\_cfg.fbist\_cfg\_regs.reg\_(0xB + 2 * \$\{N\})\_q \\ \hline

  Description &
  Least significant 32 bits of the start address for Pool 0 Engine \$\{N\}. \\
\end{tabular}
\\
\begin{tabularx}{\textwidth}{r|l|l|l|X}
  \hline
  Bit   & Field Mnemonic      & Type & Reset & Description \\ \hline

  31:0  & ADDRESS\_START\_LOW & RW   & 0     &

  Bits 31:0 of the start address used for the address generation for
  Engine \$\{N\} in Pool 0 in all address schemes. \\
\end{tabularx}

% FBIST Pool 0 Engine N Address Start High
\subsubsection{FBIST Pool 0 Engine \$\{N\} Address Start High}
\begin{tabular}{ r | p{350px} }
  Mnemonic       & FBIST\_POOL\_0\_ENGINE\_\$\{N\}\_ADDRESS\_START\_HIGH        \\
  Address Offset & $4 * 2 * \$\{N\} + \texttt{0x30}$                            \\
  VHDL Name      &  fbist\_cfg.fbist\_cfg\_regs.reg\_(0xB + 2 * \$\{N\} + 1)\_q \\ \hline

  Description &
  Most significant 32 bits of the start address for Pool 0 Engine \$\{N\}. \\
\end{tabular}
\\
\begin{tabularx}{\textwidth}{r|l|l|l|X}
  \hline
  Bit   & Field Mnemonic       & Type & Reset & Description \\ \hline

  31:0  & ADDRESS\_START\_HIGH & RW   & 0     &

  Bits 63:32 of the start address used for the address generation for
  Engine \$\{N\} in Pool 0 in all address schemes. \\
\end{tabularx}

% FBIST Pool 0 Address AND Mask Low
\subsubsection{FBIST Pool 0 Address AND Mask Low}
\begin{tabular}{ r | p{350px} }
  Mnemonic       & FBIST\_POOL\_0\_ADDRESS\_AND\_MASK\_LOW \\
  Address Offset & \texttt{0x0006C}                        \\
  VHDL Name      &  fbist\_cfg.fbist\_cfg\_regs.reg\_1B\_q \\ \hline

  Description &
  Least significant 32 bits of the AND mask address for Pool 0. \\
\end{tabular}
\\
\begin{tabularx}{\textwidth}{r|l|l|l|X}
  \hline
  Bit   & Field Mnemonic          & Type & Reset      & Description \\ \hline

  31:0  & ADDRESS\_AND\_MASK\_LOW & RW   & 0xFFFFFFFF &

  Bits 31:0 of the AND mask address used for the address generation
  for all engines in Pool 0. \\
\end{tabularx}

% FBIST Pool 0 Address AND Mask High
\subsubsection{FBIST Pool 0 Address AND Mask High}
\begin{tabular}{ r | p{350px} }
  Mnemonic       & FBIST\_POOL\_0\_ADDRESS\_AND\_MASK\_HIGH \\
  Address Offset & \texttt{0x00070}                         \\
  VHDL Name      &  fbist\_cfg.fbist\_cfg\_regs.reg\_1C\_q  \\ \hline

  Description &
  Least significant 32 bits of the AND mask address for Pool 0. \\
\end{tabular}
\\
\begin{tabularx}{\textwidth}{r|l|l|l|X}
  \hline
  Bit   & Field Mnemonic           & Type & Reset      & Description \\ \hline

  31:0  & ADDRESS\_AND\_MASK\_HIGH & RW   & 0xFFFFFFFF &

  Bits 63:32 of the AND mask address used for the address generation
  for all engines in Pool 0. \\
\end{tabularx}

% FBIST Pool 0 Address OR Mask Low
\subsubsection{FBIST Pool 0 Address OR Mask Low}
\begin{tabular}{ r | p{350px} }
  Mnemonic       & FBIST\_POOL\_0\_ADDRESS\_OR\_MASK\_LOW  \\
  Address Offset & \texttt{0x00074}                        \\
  VHDL Name      &  fbist\_cfg.fbist\_cfg\_regs.reg\_1C\_q \\ \hline

  Description &
  Least significant 32 bits of the OR mask address for Pool 0. \\
\end{tabular}
\\
\begin{tabularx}{\textwidth}{r|l|l|l|X}
  \hline
  Bit   & Field Mnemonic         & Type & Reset & Description \\ \hline

  31:0  & ADDRESS\_OR\_MASK\_LOW & RW   & 0     &

  Bits 31:0 of the OR mask address used for the address generation
  for all engines in Pool 0. \\
\end{tabularx}

% FBIST Pool 0 Address OR Mask High
\subsubsection{FBIST Pool 0 Address OR Mask High}
\begin{tabular}{ r | p{350px} }
  Mnemonic       & FBIST\_POOL\_0\_ADDRESS\_OR\_MASK\_HIGH \\
  Address Offset & \texttt{0x00078}                        \\
  VHDL Name      &  fbist\_cfg.fbist\_cfg\_regs.reg\_1D\_q \\ \hline

  Description &
  Least significant 32 bits of the OR mask address for Pool 0. \\
\end{tabular}
\\
\begin{tabularx}{\textwidth}{r|l|l|l|X}
  \hline
  Bit   & Field Mnemonic          & Type & Reset & Description \\ \hline

  31:0  & ADDRESS\_OR\_MASK\_HIGH & RW   & 0     &

  Bits 63:32 of the OR mask address used for the address generation
  for all engines in Pool 0. \\
\end{tabularx}

% FBIST Pool 0 Stats Number of Reads
\subsubsection{FBIST Pool 0 Stats Number of Reads}
\begin{tabular}{ r | p{350px} }
  Mnemonic       & FBIST\_POOL\_0\_STATS\_NUMBER\_OF\_READS \\
  Address Offset & \texttt{0x0007C}                         \\
  VHDL Name      &  fbist\_cfg.fbist\_cfg\_regs.reg\_1F\_q  \\ \hline

  Description &
  Command statistics for Pool 0. \\
\end{tabular}
\\
\begin{tabularx}{\textwidth}{r|l|l|l|X}
  \hline
  Bit   & Field Mnemonic & Type & Reset & Description \\ \hline

  31:0  & NUM\_READS     & ROX  & 0     &

  Number of reads done by Pool 0 during an FBIST run. This register is
  automatically reset when a new FBIST run starts. This field contains
  bits 31:0 of this value. Bits 47:32 are in
  FBIST\_POOL\_0\_STATS\_NUMBER\_OF\_READS\_WRITES\_EXTENDED. \\
\end{tabularx}

% FBIST Pool 0 Stats Number of Writes
\subsubsection{FBIST Pool 0 Stats Number of Writes}
\begin{tabular}{ r | p{350px} }
  Mnemonic       & FBIST\_POOL\_0\_STATS\_NUMBER\_OF\_WRITES \\
  Address Offset & \texttt{0x00080}                          \\
  VHDL Name      &  fbist\_cfg.fbist\_cfg\_regs.reg\_20\_q   \\ \hline

  Description &
  Command statistics for Pool 0. \\
\end{tabular}
\\
\begin{tabularx}{\textwidth}{r|l|l|l|X}
  \hline
  Bit   & Field Mnemonic & Type & Reset & Description \\ \hline

  31:0  & NUM\_WRITES     & ROX  & 0     &

  Number of writes done by Pool 0 during an FBIST run. This register
  is automatically reset when a new FBIST run starts. This field
  contains bits 31:0 of this value. Bits 47:32 are in
  FBIST\_POOL\_0\_STATS\_NUMBER\_OF\_READS\_WRITES\_EXTENDED. \\
\end{tabularx}

% FBIST Pool 0 Stats Number of Bytes Read
\subsubsection{FBIST Pool 0 Stats Number of Bytes Read}
\begin{tabular}{ r | p{350px} }
  Mnemonic       & FBIST\_POOL\_0\_STATS\_NUMBER\_OF\_BYTES\_READ \\
  Address Offset & \texttt{0x00084}                               \\
  VHDL Name      &  fbist\_cfg.fbist\_cfg\_regs.reg\_21\_q        \\ \hline

  Description &
  Command statistics for Pool 0. \\
\end{tabular}
\\
\begin{tabularx}{\textwidth}{r|l|l|l|X}
  \hline
  Bit   & Field Mnemonic   & Type & Reset & Description \\ \hline

  31:0  & NUM\_BYTES\_READ & ROX  & 0     &

  Number of 32-byte words read by Pool 0 during an FBIST run. This
  register is automatically reset when a new FBIST run starts. This
  field contains bits 31:0 of this value. Bits 47:32 are in
  FBIST\_POOL\_0\_STATS\_NUMBER\_OF\_BYTES\_READ\_WRITTEN\_EXTENDED. \\
\end{tabularx}

% FBIST Pool 0 Stats Number of Bytes Written
\subsubsection{FBIST Pool 0 Stats Number of Bytes Written}
\begin{tabular}{ r | p{350px} }
  Mnemonic       & FBIST\_POOL\_0\_STATS\_NUMBER\_OF\_BYTES\_WRITTEN \\
  Address Offset & \texttt{0x00088}                                  \\
  VHDL Name      &  fbist\_cfg.fbist\_cfg\_regs.reg\_22\_q           \\ \hline

  Description &
  Command statistics for Pool 0. \\
\end{tabular}
\\
\begin{tabularx}{\textwidth}{r|l|l|l|X}
  \hline
  Bit   & Field Mnemonic      & Type & Reset & Description \\ \hline

  31:0  & NUM\_BYTES\_WRITTEN & ROX  & 0     &

  Number of 32-byte words written by Pool 0 during an FBIST run. This
  register is automatically reset when a new FBIST run starts. This
  field contains bits 31:0 of this value. Bits 47:32 are in
  FBIST\_POOL\_0\_STATS\_NUMBER\_OF\_BYTES\_READ\_WRITTEN\_EXTENDED. \\
\end{tabularx}

% FBIST Pool 0 Stats Test Time
\subsubsection{FBIST Pool 0 Test Time}
\begin{tabular}{ r | p{350px} }
  Mnemonic       & FBIST\_POOL\_0\_STATS\_TEST\_TIME       \\
  Address Offset & \texttt{0x0008C}                        \\
  VHDL Name      &  fbist\_cfg.fbist\_cfg\_regs.reg\_23\_q \\ \hline

  Description &
  Command statistics for Pool 0. \\
\end{tabular}
\\
\begin{tabularx}{\textwidth}{r|l|l|l|X}
  \hline
  Bit   & Field Mnemonic & Type & Reset & Description \\ \hline

  31:0  & TEST\_TIME     & ROX  & 0     &

  Number of cycles an FBIST test runs for. This register is
  automatically reset when a new FBIST run starts. This field contains
  bits 31:0 of this value. Bits 47:32 are in
  FBIST\_POOL\_0\_STATS\_TEST\_TIME\_EXTENDED. Among all the stats,
  this counter increments the fastest and rolls over in about 8 days
  at a 25.6 GHz link speed. \\
\end{tabularx}

% FBIST Error
\subsubsection{FBIST Error}
\begin{tabular}{ r | p{350px} }
  Mnemonic       & FBIST\_ERROR                            \\
  Address Offset & \texttt{0x00090}                        \\
  VHDL Name      &  fbist\_cfg.fbist\_cfg\_regs.reg\_24\_q \\ \hline

  Description &
  FBIST test error information. \\
\end{tabular}
\\
\begin{tabularx}{\textwidth}{r|l|l|l|X}
  \hline
  Bit   & Field Mnemonic  & Type & Reset & Description \\ \hline

  31:4  & Reserved        & RO   & 0     &

  Reserved 0. \\

  3     & HANG\_ERROR     & RW   & 0     &

  If 1, FBIST has not seen a response in 2\textsuperscript{31} cycles and is
  considered hung, which is an error condition. Not reset
  automatically at start of test. \\

  2     & Reserved        & RW  & 0     &

  Reserved 0. \\

  1     & MISMATCH\_ERROR & RW   & 0     &

  If 1, FBIST has seen commands or data that don't match expected
  values. Write to 0 to reset. Not reset automatically at start of
  test. \\

  0     & FBIST\_ERROR    & RW   & 0     &

  If 1, FBIST has seen an error. The type of error can be determined
  by the other bits in this register, at least one of which should be
  set. Not reset automatically at start of test. \\
\end{tabularx}

% FBIST Pool 0 Data
\subsubsection{FBIST Pool 0 Data}
\begin{tabular}{ r | p{350px} }
  Mnemonic       & FBIST\_POOL\_0\_DATA                    \\
  Address Offset & \texttt{0x00094}                        \\
  VHDL Name      &  fbist\_cfg.fbist\_cfg\_regs.reg\_25\_q \\ \hline

  Description &
  FBIST data generation configuration. \\
\end{tabular}
\\
\begin{tabularx}{\textwidth}{r|l|l|l|X}
  \hline
  Bit   & Field Mnemonic       & Type & Reset & Description \\ \hline

  31:28 & ADDRMOD7\_DATA\_MODE & RW   & 0     &

  Data generation mode for reads and write to addresses where OpenCAPI
  PA(8:6) = 0x7. See ADDRMOD7\_DATA\_MODE for decode.

  \setlength\parindent{24pt}
  \indent 0x0 : Data Equals Address \newline
  \indent 0x1 : All 0               \newline
  \indent 0x2 : All 1               \newline
  \indent 0x3 : Repeating 0xA       \newline
  \indent 0x4 : Repeating 0xC       \newline
  \indent 0x5 : Repeating 0x6       \newline
  \indent 0x6 : Repeating 0xF0      \newline
  \indent 0x7 : Reserved            \newline
  \indent 0x8 : User Data 0         \newline
  \indent 0x9 : User Data 1         \newline
  \indent All others Reserved       \\

  27:24 & ADDRMOD6\_DATA\_MODE & RW   & 0     &

  Data generation mode for reads and write to addresses where OpenCAPI
  PA(8:6) = 0x6. See ADDRMOD7\_DATA\_MODE for decode. \\

  23:20 & ADDRMOD5\_DATA\_MODE & RW   & 0     &

  Data generation mode for reads and write to addresses where OpenCAPI
  PA(8:6) = 0x5. See ADDRMOD7\_DATA\_MODE for decode. \\

  19:16 & ADDRMOD4\_DATA\_MODE & RW   & 0     &

  Data generation mode for reads and write to addresses where OpenCAPI
  PA(8:6) = 0x4. See ADDRMOD7\_DATA\_MODE for decode. \\

  15:12 & ADDRMOD3\_DATA\_MODE & RW   & 0     &

  Data generation mode for reads and write to addresses where OpenCAPI
  PA(8:6) = 0x3. See ADDRMOD7\_DATA\_MODE for decode. \\

  11:8  & ADDRMOD2\_DATA\_MODE & RW   & 0     &

  Data generation mode for reads and write to addresses where OpenCAPI
  PA(8:6) = 0x2. See ADDRMOD7\_DATA\_MODE for decode. \\

  7:4   & ADDRMOD1\_DATA\_MODE & RW   & 0     &

  Data generation mode for reads and write to addresses where OpenCAPI
  PA(8:6) = 0x1. See ADDRMOD7\_DATA\_MODE for decode. \\

  3:0   & ADDRMOD0\_DATA\_MODE & RW   & 0     &

  Data generation mode for reads and write to addresses where OpenCAPI
  PA(8:6) = 0x0. See ADDRMOD7\_DATA\_MODE for decode. \\
\end{tabularx}

% FBIST User Data 0 High
\subsubsection{FBIST User Data 0 High}
\begin{tabular}{ r | p{350px} }
  Mnemonic       & FBIST\_USER\_DATA\_0\_HIGH              \\
  Address Offset & \texttt{0x00098}                        \\
  VHDL Name      &  fbist\_cfg.fbist\_cfg\_regs.reg\_26\_q \\ \hline

  Description &
  Most significant 32 bits of User Data 0 used in data generation. \\
\end{tabular}
\\
\begin{tabularx}{\textwidth}{r|l|l|l|X}
  \hline
  Bit   & Field Mnemonic      & Type & Reset & Description \\ \hline

  31:0  & USER\_DATA\_0\_HIGH & RW   & 0     &

  Bits 63:32 of User Data 0 used in data generation when selecting the
  User Data 0 data generation mode. \\
\end{tabularx}

% FBIST User Data 0 Low
\subsubsection{FBIST User Data 0 Low}
\begin{tabular}{ r | p{350px} }
  Mnemonic       & FBIST\_USER\_DATA\_0\_LOW              \\
  Address Offset & \texttt{0x0009C}                        \\
  VHDL Name      &  fbist\_cfg.fbist\_cfg\_regs.reg\_27\_q \\ \hline

  Description &
  Least significant 32 bits of User Data 0 used in data generation. \\
\end{tabular}
\\
\begin{tabularx}{\textwidth}{r|l|l|l|X}
  \hline
  Bit   & Field Mnemonic     & Type & Reset & Description \\ \hline

  31:0  & USER\_DATA\_0\_LOW & RW   & 0     &

  Bits 31:0 of User Data 0 used in data generation when selecting the
  User Data 0 data generation mode. \\
\end{tabularx}

% FBIST User Data 1 High
\subsubsection{FBIST User Data 1 High}
\begin{tabular}{ r | p{350px} }
  Mnemonic       & FBIST\_USER\_DATA\_1\_HIGH              \\
  Address Offset & \texttt{0x000A0}                        \\
  VHDL Name      &  fbist\_cfg.fbist\_cfg\_regs.reg\_28\_q \\ \hline

  Description &
  Most significant 32 bits of User Data 1 used in data generation. \\
\end{tabular}
\\
\begin{tabularx}{\textwidth}{r|l|l|l|X}
  \hline
  Bit   & Field Mnemonic      & Type & Reset & Description \\ \hline

  31:0  & USER\_DATA\_1\_HIGH & RW   & 0     &

  Bits 63:32 of User Data 1 used in data generation when selecting the
  User Data 1 data generation mode. \\
\end{tabularx}

% FBIST User Data 1 Low
\subsubsection{FBIST User Data 1 Low}
\begin{tabular}{ r | p{350px} }
  Mnemonic       & FBIST\_USER\_DATA\_1\_LOW              \\
  Address Offset & \texttt{0x000A4}                        \\
  VHDL Name      &  fbist\_cfg.fbist\_cfg\_regs.reg\_29\_q \\ \hline

  Description &
  Least significant 32 bits of User Data 1 used in data generation. \\
\end{tabular}
\\
\begin{tabularx}{\textwidth}{r|l|l|l|X}
  \hline
  Bit   & Field Mnemonic     & Type & Reset & Description \\ \hline

  31:0  & USER\_DATA\_1\_LOW & RW   & 0     &

  Bits 31:0 of User Data 1 used in data generation when selecting the
  User Data 1 data generation mode. \\
\end{tabularx}

% FBIST Pool 0 Stats Number of Reads Writes Extended
\subsubsection{FBIST Pool 0 Stats Number of Reads Writes Extended}
\begin{tabular}{ r | p{350px} }
  Mnemonic       & FBIST\_POOL\_0\_STATS\_NUMBER\_OF\_READ\_WRITES\_EXTENDED \\
  Address Offset & \texttt{0x000A8}                                          \\
  VHDL Name      &  fbist\_cfg.fbist\_cfg\_regs.reg\_2A\_q                   \\ \hline

  Description &
  Command statistics for Pool 0. \\
\end{tabular}
\\
\begin{tabularx}{\textwidth}{r|l|l|l|X}
  \hline
  Bit   & Field Mnemonic   & Type & Reset & Description \\ \hline

  31:16 & NUM\_WRITES\_EXT & ROX  & 0     &

  Number of writes done by Pool 0 during an FBIST run. This register is
  automatically reset when a new FBIST run starts. This field contains
  bits 47:32 of this value. Bits 31:0 are in
  FBIST\_POOL\_0\_STATS\_NUMBER\_OF\_WRITES. \\

  15:0  & NUM\_READS\_EXT  & ROX  & 0     &

  Number of reads done by Pool 0 during an FBIST run. This register is
  automatically reset when a new FBIST run starts. This field contains
  bits 47:32 of this value. Bits 31:0 are in
  FBIST\_POOL\_0\_STATS\_NUMBER\_OF\_READS. \\
\end{tabularx}

% FBIST Pool 0 Stats Number of Bytes Read Written Extended
\subsubsection{FBIST Pool 0 Stats Number of Bytes Read Written Extended}
\begin{tabular}{ r | p{350px} }
  Mnemonic       & FBIST\_POOL\_0\_STATS\_NUMBER\_OF\_BYTES\_READ\_WRITTEN\_EXTENDED \\
  Address Offset & \texttt{0x000AC}                                                  \\
  VHDL Name      &  fbist\_cfg.fbist\_cfg\_regs.reg\_2B\_q                           \\ \hline

  Description &
  Command statistics for Pool 0. \\
\end{tabular}
\\
\begin{tabularx}{\textwidth}{r|l|l|l|X}
  \hline
  Bit   & Field Mnemonic           & Type & Reset & Description \\ \hline

  31:16 & NUM\_BYTES\_WRITTEN\_EXT & ROX  & 0     &

  Number of 32-byte words read by Pool 0 during an FBIST run. This
  register is automatically reset when a new FBIST run starts. This
  field contains bits 47:32 of this value. Bits 31:0 are in
  FBIST\_POOL\_0\_STATS\_NUMBER\_OF\_BYTES\_WRITTEN. \\

  15:0  & NUM\_BYTES\_READ\_EXT    & ROX  & 0     &

  Number of 32-byte words read by Pool 0 during an FBIST run. This
  register is automatically reset when a new FBIST run starts. This
  field contains bits 47:32 of this value. Bits 31:0 are in
  FBIST\_POOL\_0\_STATS\_NUMBER\_OF\_BYTES\_READ. \\
\end{tabularx}

% FBIST Pool 0 Stats Test Time Extended
\subsubsection{FBIST Pool 0 Stats Test Time Extended}
\begin{tabular}{ r | p{350px} }
  Mnemonic       & FBIST\_POOL\_0\_STATS\_TEST\_TIME\_EXTENDED \\
  Address Offset & \texttt{0x000B0}                            \\
  VHDL Name      &  fbist\_cfg.fbist\_cfg\_regs.reg\_2C\_q     \\ \hline

  Description &
  Command statistics for Pool 0. \\
\end{tabular}
\\
\begin{tabularx}{\textwidth}{r|l|l|l|X}
  \hline
  Bit   & Field Mnemonic  & Type & Reset & Description \\ \hline

  31:16 & Reserved        & RO   & 0     &

  Reserved 0. \\

  15:0  & TEST\_TIME\_EXT & ROX  & 0     &

  Number of cycles an FBIST test runs for. This register is
  automatically reset when a new FBIST run starts. This field contains
  bits 47:32 of this value. Bits 31:0 are in
  FBIST\_POOL\_0\_STATS\_TEST\_TIME. Among all the stats, this counter
  increments the fastest and rolls over in about 8 days at a 25.6 GHz
  link speed. \\
\end{tabularx}

% FBIST Pool 0 Engine 0 Latency Low
\subsubsection{FBIST Pool 0 Engine 0 Latency Low}
\begin{tabular}{ r | p{350px} }
  Mnemonic       & FBIST\_POOL\_0\_ENGINE\_0\_LATENCY\_LOW \\
  Address Offset & \texttt{0x000B4}                        \\
  VHDL Name      &  fbist\_cfg.fbist\_cfg\_regs.reg\_2D\_q \\ \hline

  Description &
  Latency calculations for Pool 0. \\
\end{tabular}
\\
\begin{tabularx}{\textwidth}{r|l|l|l|X}
  \hline
  Bit   & Field Mnemonic      & Type & Reset & Description \\ \hline

  31:0  & TOTAL\_LATENCY\_LOW & ROX  & 0     &

  Bits 31:0 of the total latency for all commands for this engine. The
  latency for a single command is measure from when the command was
  inserted into the LUT while being dispatched to when it was read
  from the LUT to be checked. The average latency is the total latency
  divided by the number of responses. \\
\end{tabularx}

% FBIST Pool 0 Engine 0 Latency High
\subsubsection{FBIST Pool 0 Engine 0 Latency High}
\begin{tabular}{ r | p{350px} }
  Mnemonic       & FBIST\_POOL\_0\_ENGINE\_0\_LATENCY\_HIGH \\
  Address Offset & \texttt{0x000B8}                         \\
  VHDL Name      &  fbist\_cfg.fbist\_cfg\_regs.reg\_2E\_q  \\ \hline

  Description &
  Latency calculations for Pool 0. \\
\end{tabular}
\\
\begin{tabularx}{\textwidth}{r|l|l|l|X}
  \hline
  Bit   & Field Mnemonic       & Type & Reset & Description \\ \hline

  31:0  & TOTAL\_LATENCY\_HIGH & ROX  & 0     &

  Bits 63:32 of the total latency for all commands for this engine,
  continue from the corresponding TOTAL\_LATENCY\_LOW. \\
\end{tabularx}

% FBIST Pool 0 Engine 1 Latency Low
\subsubsection{FBIST Pool 0 Engine 1 Latency Low}
\begin{tabular}{ r | p{350px} }
  Mnemonic       & FBIST\_POOL\_0\_ENGINE\_1\_LATENCY\_LOW \\
  Address Offset & \texttt{0x000BC}                        \\
  VHDL Name      &  fbist\_cfg.fbist\_cfg\_regs.reg\_2F\_q \\ \hline

  Description &
  Latency calculations for Pool 0. \\
\end{tabular}
\\
\begin{tabularx}{\textwidth}{r|l|l|l|X}
  \hline
  Bit   & Field Mnemonic      & Type & Reset & Description \\ \hline

  31:0  & TOTAL\_LATENCY\_LOW & ROX  & 0     &

  Bits 31:0 of the total latency for all commands for this engine. The
  latency for a single command is measure from when the command was
  inserted into the LUT while being dispatched to when it was read
  from the LUT to be checked. The average latency is the total latency
  divided by the number of responses. \\
\end{tabularx}

% FBIST ERROR ARRAY ENTRIES          }
\subsubsection{FBIST Error Array Data  }

\begin{tabular}{ r | p{350px} }
  Mnemonic       & FBIST\_ERROR\_ARRAY\_DATA      \\
  Address Offset & \texttt{0x000C0}               \\
  VHDL Name      & fbist\_fes\_buffer.exp\_rd\_data \\ \hline

  Description &
  This register is read repeatedly, 17 times per error, each read supplies
  four bytes of an error entry. \\
\end{tabular}
\\
\begin{tabularx}{\textwidth}{r|l|l|l|X}
  \hline
  Bit   & Field Mnemonic & Type & Reset & Description \\ \hline

  31:0  & error\_snapshot\_data       & RO   & X     &  first read gives LS 4 bytes \newline
  of received data ..... \\
  31:0  & error\_snapshot\_data    & RO   & X     & ....  subsequent 15 reads give the remaining
  bytes of the 64B data returned if a read.\\
  31:24  & error\_snapshot       & RO   & X     &  (mod seventeen reads)  \newline
  \setlength\parindent{24pt}
  \indent 0x01 : good write response \newline
  \indent 0x02 : bad  write response \newline
  \indent 0x03 : good read response \newline
  \indent 0x04 : bad  read response \\
  23:20  & error\_snapshot\_engine & RO   & X  &  (mod seventeen reads)  \newline
  fbist engine \\
  19:4   & error\_snapshot\_tag & RO   & X     &  (mod seventeen reads)  \newline
  actual and expected capptag. \\
  3:2   & error\_snapshot\_dp & RO   & X     &  (mod seventeen reads)  \newline
  OpenCAPI dpart. \\
  1    & error\_snapshot\_ow & RO   & X     &  (mod seventeen reads)  \newline
  OpenCAPI OW. \\
  0   & error\_snapshot\_dmisc & RO   & X     &  (mod seventeen reads) \newline
  '1' for data compare error, else '0'. \\
\end{tabularx}

% FBIST ERROR ARRAY INFO
\subsubsection{FBIST Error Array Info }
\begin{tabular}{ r | p{350px} }
  Mnemonic       & FBIST\_ERROR\_ARRAY\_INFO        \\
  Address Offset & \texttt{0x000C4}                 \\
  VHDL Name      & fbist\_fes\_buffer.exp\_rd\_info \\ \hline

  Description &
  This register supplies information about the Error Array
  pointers. These pointers are reset by hardware when fbist starts. \\
\end{tabular}
\\
\begin{tabularx}{\textwidth}{r|l|l|l|X}
  \hline
  Bit   & Field Mnemonic               & Type & Reset & Description \\ \hline

  31:24 & error\_array\_valid\_samples & RO   & 0     &

  The number of valid samples in the array \\

  23:16 & Reserved                     & RO   & 0x00  &

  Reserved 0. \\

  15:8  & error\_current\_sample       & RO   & 0     &

  The current sample returned when FBIST\_ERROR\_ARRAY\_DATA register
  is read. \\

  7:0   & error\_current\_qw           & RO   & 0     &

  The quadbyte of the current sample returned when
  FBIST\_ERROR\_ARRAY\_DATA register is read. \\
\end{tabularx}

% FBIST ERROR ARRAY RESET
\subsubsection{FBIST Error Array Reset}
\begin{tabular}{ r | p{350px} }
  Mnemonic       & FBIST\_ERROR\_ARRAY\_INFO      \\
  Address Offset & \texttt{0x000C8}               \\
  VHDL Name      & fbist\_fes\_buffer.reset\_rptr \\ \hline

  Description &
  Any write to this register will reset the read pointers used in the
  error array logic. Immediately following a write to this register a
  read of the FBIST\_ERROR\_ARRAY\_DATA register will return the first
  4 bytes of the oldest sample in the array. \\
\end{tabular}
\\
\begin{tabularx}{\textwidth}{r|l|l|l|X}
  \hline
  Bit   & Field Mnemonic            & Type & Reset & Description \\ \hline

  31:0  & error\_array\_reset\_rptr & WO   & -     &

  The data is not used for anything and is don't care. \\
\end{tabularx}

% FBIST Pool 0 Engine 1 Latency High
\subsubsection{FBIST Pool 0 Engine 1 Latency High}
\begin{tabular}{ r | p{350px} }
  Mnemonic       & FBIST\_POOL\_0\_ENGINE\_1\_LATENCY\_HIGH \\
  Address Offset & \texttt{0x000CC}                         \\
  VHDL Name      &  fbist\_cfg.fbist\_cfg\_regs.reg\_33\_q  \\ \hline

  Description &
  Latency calculations for Pool 0. \\
\end{tabular}
\\
\begin{tabularx}{\textwidth}{r|l|l|l|X}
  \hline
  Bit   & Field Mnemonic       & Type & Reset & Description \\ \hline

  31:0  & TOTAL\_LATENCY\_HIGH & ROX  & 0     &

  Bits 63:32 of the total latency for all commands for this engine,
  continue from the corresponding TOTAL\_LATENCY\_LOW. \\
\end{tabularx}

% FBIST Pool 0 Engine 2 Latency Low
\subsubsection{FBIST Pool 0 Engine 2 Latency Low}
\begin{tabular}{ r | p{350px} }
  Mnemonic       & FBIST\_POOL\_0\_ENGINE\_2\_LATENCY\_LOW \\
  Address Offset & \texttt{0x000D0}                        \\
  VHDL Name      &  fbist\_cfg.fbist\_cfg\_regs.reg\_34\_q \\ \hline

  Description &
  Latency calculations for Pool 0. \\
\end{tabular}
\\
\begin{tabularx}{\textwidth}{r|l|l|l|X}
  \hline
  Bit   & Field Mnemonic      & Type & Reset & Description \\ \hline

  31:0  & TOTAL\_LATENCY\_LOW & ROX  & 0     &

  Bits 31:0 of the total latency for all commands for this engine. The
  latency for a single command is measure from when the command was
  inserted into the LUT while being dispatched to when it was read
  from the LUT to be checked. The average latency is the total latency
  divided by the number of responses. \\
\end{tabularx}

% FBIST Pool 0 Engine 2 Latency High
\subsubsection{FBIST Pool 0 Engine 2 Latency High}
\begin{tabular}{ r | p{350px} }
  Mnemonic       & FBIST\_POOL\_0\_ENGINE\_2\_LATENCY\_HIGH \\
  Address Offset & \texttt{0x000D4}                         \\
  VHDL Name      &  fbist\_cfg.fbist\_cfg\_regs.reg\_35\_q  \\ \hline

  Description &
  Latency calculations for Pool 0. \\
\end{tabular}
\\
\begin{tabularx}{\textwidth}{r|l|l|l|X}
  \hline
  Bit   & Field Mnemonic       & Type & Reset & Description \\ \hline

  31:0  & TOTAL\_LATENCY\_HIGH & ROX  & 0     &

  Bits 63:32 of the total latency for all commands for this engine,
  continue from the corresponding TOTAL\_LATENCY\_LOW. \\
\end{tabularx}

% FBIST Pool 0 Engine 3 Latency Low
\subsubsection{FBIST Pool 0 Engine 3 Latency Low}
\begin{tabular}{ r | p{350px} }
  Mnemonic       & FBIST\_POOL\_0\_ENGINE\_3\_LATENCY\_LOW \\
  Address Offset & \texttt{0x000D8}                        \\
  VHDL Name      &  fbist\_cfg.fbist\_cfg\_regs.reg\_36\_q \\ \hline

  Description &
  Latency calculations for Pool 0. \\
\end{tabular}
\\
\begin{tabularx}{\textwidth}{r|l|l|l|X}
  \hline
  Bit   & Field Mnemonic      & Type & Reset & Description \\ \hline

  31:0  & TOTAL\_LATENCY\_LOW & ROX  & 0     &

  Bits 31:0 of the total latency for all commands for this engine. The
  latency for a single command is measure from when the command was
  inserted into the LUT while being dispatched to when it was read
  from the LUT to be checked. The average latency is the total latency
  divided by the number of responses. \\
\end{tabularx}

% FBIST Pool 0 Engine 3 Latency High
\subsubsection{FBIST Pool 0 Engine 3 Latency High}
\begin{tabular}{ r | p{350px} }
  Mnemonic       & FBIST\_POOL\_0\_ENGINE\_3\_LATENCY\_HIGH \\
  Address Offset & \texttt{0x000DC}                         \\
  VHDL Name      &  fbist\_cfg.fbist\_cfg\_regs.reg\_37\_q  \\ \hline

  Description &
  Latency calculations for Pool 0. \\
\end{tabular}
\\
\begin{tabularx}{\textwidth}{r|l|l|l|X}
  \hline
  Bit   & Field Mnemonic       & Type & Reset & Description \\ \hline

  31:0  & TOTAL\_LATENCY\_HIGH & ROX  & 0     &

  Bits 63:32 of the total latency for all commands for this engine,
  continue from the corresponding TOTAL\_LATENCY\_LOW. \\
\end{tabularx}

% FBIST Pool 0 Engine 4 Latency Low
\subsubsection{FBIST Pool 0 Engine 4 Latency Low}
\begin{tabular}{ r | p{350px} }
  Mnemonic       & FBIST\_POOL\_0\_ENGINE\_4\_LATENCY\_LOW \\
  Address Offset & \texttt{0x000E0}                        \\
  VHDL Name      &  fbist\_cfg.fbist\_cfg\_regs.reg\_38\_q \\ \hline

  Description &
  Latency calculations for Pool 0. \\
\end{tabular}
\\
\begin{tabularx}{\textwidth}{r|l|l|l|X}
  \hline
  Bit   & Field Mnemonic      & Type & Reset & Description \\ \hline

  31:0  & TOTAL\_LATENCY\_LOW & ROX  & 0     &

  Bits 31:0 of the total latency for all commands for this engine. The
  latency for a single command is measure from when the command was
  inserted into the LUT while being dispatched to when it was read
  from the LUT to be checked. The average latency is the total latency
  divided by the number of responses. \\
\end{tabularx}

% FBIST Pool 0 Engine 4 Latency High
\subsubsection{FBIST Pool 0 Engine 4 Latency High}
\begin{tabular}{ r | p{350px} }
  Mnemonic       & FBIST\_POOL\_0\_ENGINE\_4\_LATENCY\_HIGH \\
  Address Offset & \texttt{0x000E4}                         \\
  VHDL Name      &  fbist\_cfg.fbist\_cfg\_regs.reg\_39\_q  \\ \hline

  Description &
  Latency calculations for Pool 0. \\
\end{tabular}
\\
\begin{tabularx}{\textwidth}{r|l|l|l|X}
  \hline
  Bit   & Field Mnemonic       & Type & Reset & Description \\ \hline

  31:0  & TOTAL\_LATENCY\_HIGH & ROX  & 0     &

  Bits 63:32 of the total latency for all commands for this engine,
  continue from the corresponding TOTAL\_LATENCY\_LOW. \\
\end{tabularx}

% FBIST Pool 0 Engine 5 Latency Low
\subsubsection{FBIST Pool 0 Engine 5 Latency Low}
\begin{tabular}{ r | p{350px} }
  Mnemonic       & FBIST\_POOL\_0\_ENGINE\_5\_LATENCY\_LOW \\
  Address Offset & \texttt{0x000E8}                        \\
  VHDL Name      &  fbist\_cfg.fbist\_cfg\_regs.reg\_3A\_q \\ \hline

  Description &
  Latency calculations for Pool 0. \\
\end{tabular}
\\
\begin{tabularx}{\textwidth}{r|l|l|l|X}
  \hline
  Bit   & Field Mnemonic      & Type & Reset & Description \\ \hline

  31:0  & TOTAL\_LATENCY\_LOW & ROX  & 0     &

  Bits 31:0 of the total latency for all commands for this engine. The
  latency for a single command is measure from when the command was
  inserted into the LUT while being dispatched to when it was read
  from the LUT to be checked. The average latency is the total latency
  divided by the number of responses. \\
\end{tabularx}

% FBIST Pool 0 Engine 5 Latency High
\subsubsection{FBIST Pool 0 Engine 5 Latency High}
\begin{tabular}{ r | p{350px} }
  Mnemonic       & FBIST\_POOL\_0\_ENGINE\_5\_LATENCY\_HIGH \\
  Address Offset & \texttt{0x000EC}                         \\
  VHDL Name      &  fbist\_cfg.fbist\_cfg\_regs.reg\_3B\_q  \\ \hline

  Description &
  Latency calculations for Pool 0. \\
\end{tabular}
\\
\begin{tabularx}{\textwidth}{r|l|l|l|X}
  \hline
  Bit   & Field Mnemonic       & Type & Reset & Description \\ \hline

  31:0  & TOTAL\_LATENCY\_HIGH & ROX  & 0     &

  Bits 63:32 of the total latency for all commands for this engine,
  continue from the corresponding TOTAL\_LATENCY\_LOW. \\
\end{tabularx}

% FBIST Pool 0 Engine 6 Latency Low
\subsubsection{FBIST Pool 0 Engine 6 Latency Low}
\begin{tabular}{ r | p{350px} }
  Mnemonic       & FBIST\_POOL\_0\_ENGINE\_6\_LATENCY\_LOW \\
  Address Offset & \texttt{0x000F0}                        \\
  VHDL Name      &  fbist\_cfg.fbist\_cfg\_regs.reg\_3C\_q \\ \hline

  Description &
  Latency calculations for Pool 0. \\
\end{tabular}
\\
\begin{tabularx}{\textwidth}{r|l|l|l|X}
  \hline
  Bit   & Field Mnemonic      & Type & Reset & Description \\ \hline

  31:0  & TOTAL\_LATENCY\_LOW & ROX  & 0     &

  Bits 31:0 of the total latency for all commands for this engine. The
  latency for a single command is measure from when the command was
  inserted into the LUT while being dispatched to when it was read
  from the LUT to be checked. The average latency is the total latency
  divided by the number of responses. \\
\end{tabularx}

% FBIST Pool 0 Engine 6 Latency High
\subsubsection{FBIST Pool 0 Engine 6 Latency High}
\begin{tabular}{ r | p{350px} }
  Mnemonic       & FBIST\_POOL\_0\_ENGINE\_6\_LATENCY\_HIGH \\
  Address Offset & \texttt{0x000F4}                         \\
  VHDL Name      &  fbist\_cfg.fbist\_cfg\_regs.reg\_3D\_q  \\ \hline

  Description &
  Latency calculations for Pool 0. \\
\end{tabular}
\\
\begin{tabularx}{\textwidth}{r|l|l|l|X}
  \hline
  Bit   & Field Mnemonic       & Type & Reset & Description \\ \hline

  31:0  & TOTAL\_LATENCY\_HIGH & ROX  & 0     &

  Bits 63:32 of the total latency for all commands for this engine,
  continue from the corresponding TOTAL\_LATENCY\_LOW. \\
\end{tabularx}

% FBIST Pool 0 Engine 7 Latency Low
\subsubsection{FBIST Pool 0 Engine 7 Latency Low}
\begin{tabular}{ r | p{350px} }
  Mnemonic       & FBIST\_POOL\_0\_ENGINE\_7\_LATENCY\_LOW \\
  Address Offset & \texttt{0x000F8}                        \\
  VHDL Name      &  fbist\_cfg.fbist\_cfg\_regs.reg\_3E\_q \\ \hline

  Description &
  Latency calculations for Pool 0. \\
\end{tabular}
\\
\begin{tabularx}{\textwidth}{r|l|l|l|X}
  \hline
  Bit   & Field Mnemonic      & Type & Reset & Description \\ \hline

  31:0  & TOTAL\_LATENCY\_LOW & ROX  & 0     &

  Bits 31:0 of the total latency for all commands for this engine. The
  latency for a single command is measure from when the command was
  inserted into the LUT while being dispatched to when it was read
  from the LUT to be checked. The average latency is the total latency
  divided by the number of responses. \\
\end{tabularx}

% FBIST Pool 0 Engine 7 Latency High
\subsubsection{FBIST Pool 0 Engine 7 Latency High}
\begin{tabular}{ r | p{350px} }
  Mnemonic       & FBIST\_POOL\_0\_ENGINE\_7\_LATENCY\_HIGH \\
  Address Offset & \texttt{0x000FC}                         \\
  VHDL Name      &  fbist\_cfg.fbist\_cfg\_regs.reg\_3F\_q  \\ \hline

  Description &
  Latency calculations for Pool 0. \\
\end{tabular}
\\
\begin{tabularx}{\textwidth}{r|l|l|l|X}
  \hline
  Bit   & Field Mnemonic       & Type & Reset & Description \\ \hline

  31:0  & TOTAL\_LATENCY\_HIGH & ROX  & 0     &

  Bits 63:32 of the total latency for all commands for this engine,
  continue from the corresponding TOTAL\_LATENCY\_LOW. \\
\end{tabularx}

% FBIST Pool 0 Inject
\subsubsection{FBIST Pool 0 Inject}
\begin{tabular}{ r | p{350px} }
  Mnemonic       & FBIST\_POOL\_0\_INJECT                  \\
  Address Offset & \texttt{0x00100}                        \\
  VHDL Name      &  fbist\_cfg.fbist\_cfg\_regs.reg\_40\_q \\ \hline

  Description &
  Error injections for Pool 0. \\
\end{tabular}
\\
\begin{tabularx}{\textwidth}{r|l|l|l|X}
  \hline
  Bit  & Field Mnemonic          & Type & Reset & Description \\ \hline

  31:4 & Reserved                & RO   & 0     &

  Reserved 0. \\

  3    & ADDRESS\_INJECT\_ENABLE & RW   & 0     &

  Set to 1 to force OpenCAPI PA bit 0 to flip on the next FBIST read
  or write command dispatched. This will lead to a misaligned address
  sent to the endpoint, and a fail response. Toggle to reinject on
  future commands. \\

  2    & ADDRESS\_INJECT\_DONE   & RWX  & 0     &

  Set to 1 to when an address error was successfully injected. Write
  to 0 to clear. Not automatically cleared by hardware. \\

  1    & DATA\_INJECT\_ENABLE    & RW   & 0     &

  Set to 1 to force OpenCAPI Data bit 0 to flip on the next FBIST
  write command dispatched. This will lead to a miscompare when
  checking subsequent reads to the same location. Toggle to reinject
  on future commands. \\

  0    & DATA\_INJECT\_ENABLE    & RWX  & 0     &

  Set to 1 to when a data error was successfully injected. Write to 0
  to clear. Not automatically cleared by hardware. \\
\end{tabularx}
