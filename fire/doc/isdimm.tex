\section{Dedicated DDR4 ISDIMM} \label{section_isdimm}

Connected to the FPGA there is a single DDR4 connector. This provides
DRAM general use storage for any purpose in the FPGA design, such as
memory for a MicroBlaze application. This memory is accessible as part
of the overall memory space of the system. Because this address space
is located with the address space of the configuration space on the
FPGA, as well as all memory and registers on attached OCMB, care must
be taken by application programmers to maintain in the DDR4 address
space, as there is no protection. The DDR4 memory space is also shared
unprotected among all applications.

The offset and size of this memory is TBD. While different size
memories can be installed on the board, this requires a different FPGA
image to match the new configuration. Currently, the memory controller
logic does not support A17 (address bit 17) or C2 (chip id bit 2), and
targets a 72-bit DIMM with x4 DRAMs.

The FPGA logic is implemented with Xilinx IP consisting of a DDR4
memory controller and a DDR4 PHY, detailed in \textit{UltraScale
Architecture-Based FPGAs Memory IP v1.4}
(PG150)\footnote{\url{https://www.xilinx.com/support/documentation/ip\_documentation/ultrascale\_memory\_ip/v1\_4/pg150-ultrascale-memory-ip.pdf}}.
This memory is not high performance, but is high capacity. The core is
attached to the main AXI4 Interconnect, and can be addressed similar
to other memory attached to that interconnect. The monitor and control
interface for the ECC is connected via a separate AXI4-Lite port.

This logic and port is optional, only needed if FPGA logic or
applications require it.
